%% Generated by Sphinx.
\def\sphinxdocclass{report}
\documentclass[a4paper,10pt,english]{sphinxmanual}
\ifdefined\pdfpxdimen
   \let\sphinxpxdimen\pdfpxdimen\else\newdimen\sphinxpxdimen
\fi \sphinxpxdimen=.75bp\relax
\ifdefined\pdfimageresolution
    \pdfimageresolution= \numexpr \dimexpr1in\relax/\sphinxpxdimen\relax
\fi
%% let collapsible pdf bookmarks panel have high depth per default
\PassOptionsToPackage{bookmarksdepth=5}{hyperref}

\PassOptionsToPackage{warn}{textcomp}
\usepackage[utf8]{inputenc}
\ifdefined\DeclareUnicodeCharacter
% support both utf8 and utf8x syntaxes
  \ifdefined\DeclareUnicodeCharacterAsOptional
    \def\sphinxDUC#1{\DeclareUnicodeCharacter{"#1}}
  \else
    \let\sphinxDUC\DeclareUnicodeCharacter
  \fi
  \sphinxDUC{00A0}{\nobreakspace}
  \sphinxDUC{2500}{\sphinxunichar{2500}}
  \sphinxDUC{2502}{\sphinxunichar{2502}}
  \sphinxDUC{2514}{\sphinxunichar{2514}}
  \sphinxDUC{251C}{\sphinxunichar{251C}}
  \sphinxDUC{2572}{\textbackslash}
\fi
\usepackage{cmap}
\usepackage[T1]{fontenc}
\usepackage{amsmath,amssymb,amstext}
\usepackage{babel}



\usepackage{tgtermes}
\usepackage{tgheros}
\renewcommand{\ttdefault}{txtt}



\usepackage[Bjarne]{fncychap}
\usepackage[,numfigreset=1,mathnumfig]{sphinx}

\fvset{fontsize=auto}
\usepackage{geometry}


% Include hyperref last.
\usepackage{hyperref}
% Fix anchor placement for figures with captions.
\usepackage{hypcap}% it must be loaded after hyperref.
% Set up styles of URL: it should be placed after hyperref.
\urlstyle{same}


\usepackage{sphinxmessages}
\setcounter{tocdepth}{2}



\title{PIDINST Documentation}
\date{7 February 2024}
\release{1.0b2.dev132+gb154c75}
\author{RDA Persistent Identification of Instruments WG}
\newcommand{\sphinxlogo}{\vbox{}}
\renewcommand{\releasename}{Release}
\makeindex
\begin{document}

\ifdefined\shorthandoff
  \ifnum\catcode`\=\string=\active\shorthandoff{=}\fi
  \ifnum\catcode`\"=\active\shorthandoff{"}\fi
\fi

\pagestyle{empty}
\sphinxmaketitle
\pagestyle{plain}
\sphinxtableofcontents
\pagestyle{normal}
\phantomsection\label{\detokenize{index::doc}}
\noindent{\hspace*{\fill}\sphinxincludegraphics[width=100\sphinxpxdimen]{{pidinst-logo}.pdf}}



\sphinxAtStartPar
The \sphinxhref{https://www.rd-alliance.org/groups/persistent-identification-instruments-wg}{Persistent Identification of Instruments WG (PIDINST)} seeks to explore a community\sphinxhyphen{}driven solution for globally
unique identification of measuring instruments operated in the
sciences.

\sphinxAtStartPar
Measuring instruments, such as sensors used in environmental science,
DNA sequencers used in life sciences or laboratory engines used for
medical domains, are widespread in most fields of applied sciences.
The ability to link an active instrument (instance) with an instrument
type and with the broader context in which the instrument operates,
including generated data, other instruments and platforms, people and
manufacturers, etc., is critical, especially for automated processing
of such contextual information and for the interpretation of generated
data.

\sphinxAtStartPar
PIDINST is a working group in the \sphinxhref{https://www.rd-alliance.org/}{Research Data Alliance (RDA)}.  It aims to establish a cross\sphinxhyphen{}discipline, operational
solution for the unique and lasting identification of measuring
instruments actively operated in the sciences.

\sphinxAtStartPar
The group produced the following outputs:
\begin{itemize}
\item {} 
\sphinxAtStartPar
Stocker, M, Darroch, L, Krahl, R, Habermann, T, Devaraju, A,
Schwardmann, U, D’Onofrio, C and Häggström, I.  2020.  Persistent
Identification of Instruments.  Data Science Journal, 19: 18,
pp. 1\textendash{}12.  DOI: \sphinxurl{https://doi.org/10.5334/dsj-2020-018} 
This paper provides an overview of the work of PIDINST.

\item {} 
\sphinxAtStartPar
Krahl, R., Darroch, L., Huber, R., Devaraju, A., Klump, J.,
Habermann, T., Stocker, M., \& The Research Data Alliance Persistent
Identification of Instruments Working Group members (2022).
Metadata Schema for the Persistent Identification of Instruments
(1.0).  Research Data Alliance.
\sphinxurl{https://doi.org/10.15497/RDA00070} 
This RDA recommendation defines the Metadata Schema.

\item {} 
\sphinxAtStartPar
{\hyperref[\detokenize{white-paper/index:white-paper}]{\sphinxcrossref{\DUrole{std,std-ref}{PIDINST White Paper}}}} 
This white paper provides recommendations for the use of instrument
PIDs and gives technical details that go beyond the overview
provided in the Data Science Journal paper.  It is expected to
evolve with new user requirements and working group activities.

\item {} 
\sphinxAtStartPar
{\hyperref[\detokenize{epic-cookbook/index:epic-cookbook}]{\sphinxcrossref{\DUrole{std,std-ref}{ePIC Cookbook}}}} 
Detailed instructions on how to create instrument PIDs using the
ePIC infrastructure.

\item {} 
\sphinxAtStartPar
{\hyperref[\detokenize{datacite-cookbook/index:datacite-cookbook}]{\sphinxcrossref{\DUrole{std,std-ref}{DataCite Cookbook}}}} 
Detailed instructions on how to create DataCite DOIs for instruments.

\end{itemize}

\sphinxhref{https://creativecommons.org/licenses/by/4.0/}{{\hspace*{\fill}\sphinxincludegraphics[width=88\sphinxpxdimen]{{cc-by}.pdf}}}

\sphinxAtStartPar
The content of this site is licensed under a \sphinxhref{https://creativecommons.org/licenses/by/4.0/}{Creative Commons
Attribution 4.0 International License}.


\chapter{PIDINST White Paper}
\label{\detokenize{white-paper/index:pidinst-white-paper}}\label{\detokenize{white-paper/index:white-paper}}\label{\detokenize{white-paper/index::doc}}

\begin{savenotes}\sphinxattablestart
\centering
\begin{tabular}[t]{|*{2}{\X{1}{2}|}}
\hline

\sphinxAtStartPar
Document type
&
\begin{DUlineblock}{0em}
\item[] Research Data Alliance (RDA)
\item[] Persistent Identification of Instruments (PIDINST)
\item[] working group output report
\end{DUlineblock}
\\
\hline
\sphinxAtStartPar
Version
&
\sphinxAtStartPar
1.0b2.dev132+gb154c75
\\
\hline
\sphinxAtStartPar
Date
&
\sphinxAtStartPar
7 February 2024
\\
\hline
\end{tabular}
\par
\sphinxattableend\end{savenotes}


\section{Table of contents}
\label{\detokenize{white-paper/index:table-of-contents}}

\subsection{Overview}
\label{\detokenize{white-paper/overview:overview}}\label{\detokenize{white-paper/overview::doc}}
\sphinxAtStartPar
This white paper provides recommendations for the use of instrument
Persistent Identifiers (PIDs) for institutional instrument providers,
such as bodies who fund, maintain or provide instruments, or are
responsible for maintaining information about or data from instruments.
Institutional instrument providers will be responsible for maintaining
and publishing instrument PIDs locally, potentially on behalf of other
institutional instrument providers. These recommendations are compiled
through the activities of the Research Data Alliance (RDA) \sphinxhref{https://www.rd-alliance.org/groups/persistent-identification-instruments-wg}{Persistent
Identification of Instruments (PIDINST) Working Group}.%
\begin{footnote}[1]\sphinxAtStartFootnote
Stocker, M, Darroch, L, Krahl, R, Habermann, T, Devaraju, A,
Schwardmann, U, D’Onofrio, C and Häggström, I. 2020. Persistent
Identification of Instruments. Data Science Journal, 19: 18, pp.
1\textendash{}12. DOI: \sphinxurl{https://doi.org/10.5334/dsj-2020-018})
%
\end{footnote} This document is expected to evolve with
new user requirements and working group activities.


\subsection{Instrument PIDs}
\label{\detokenize{white-paper/instrument-pids:instrument-pids}}\label{\detokenize{white-paper/instrument-pids::doc}}
\sphinxAtStartPar
The PIDINST PID is used to identify measuring instruments, defined by
the Joint Committee for Guides in Metrology (JCGM) as “device used for
making measurements, alone or in conjunction with one or more
supplementary devices” (VIM, 2012). It is used to identify the devices
themselves (instances), the real\sphinxhyphen{}world assets with instantaneous
capabilities and configurations, rather than the identification of
material instrument designs (models).


\subsection{Publishing new instrument PIDs}
\label{\detokenize{white-paper/publishing:publishing-new-instrument-pids}}\label{\detokenize{white-paper/publishing::doc}}
\sphinxAtStartPar
To create new PIDs and assign them to instruments, institutional
instrument providers will submit a metadata record following the PIDINST
metadata schema and a URL for the landing page of the instrument to a
PID provider that is compliant with RDA PIDINST recommendations. Thus,
it is necessary to become a member of the PID provider to publish PIDs
directly, or work with one of their current members or registration
repositories to publish PIDs on behalf of the institutional instrument
provider. Current PID providers known to be suitable with RDA PIDINST
are \sphinxhref{https://www.pidconsortium.net/}{ePIC} and \sphinxhref{https://datacite.org/}{DataCite}, but PIDINST is not limited to these
providers; others may implement the schema, too.


\subsection{PIDINST metadata schema}
\label{\detokenize{white-paper/metadata-schema:pidinst-metadata-schema}}\label{\detokenize{white-paper/metadata-schema:id1}}\label{\detokenize{white-paper/metadata-schema::doc}}
\sphinxAtStartPar
The metadata that is to be registered with an instrument PID needs to
contain enough information to unambiguously identify the instrument
across networks and infrastructures.  It should also allow linking
related resources to the instrument, thus providing a means to
aggregate information about the instrument.  The PIDINST working group
defined a schema of metadata that can be registered alongside
instrument PIDs at PID providers to help meet these criteria.  Version
1.0 has been endorsed as an RDA recommendation.%
\begin{footnote}[1]\sphinxAtStartFootnote
Krahl, R., Darroch, L., Huber, R., Devaraju, A., Klump, J., Habermann, T.,
Stocker, M., \& The Research Data Alliance Persistent Identification of
Instruments Working Group members (2022). Metadata Schema for the
Persistent Identification of Instruments (1.0). Research Data Alliance.
DOI: \sphinxurl{https://doi.org/10.15497/RDA00070}
%
\end{footnote}

\sphinxAtStartPar
Currently, two variants of the metadata schema exist.  The original
\sphinxhref{https://github.com/rdawg-pidinst/schema/blob/master/schema.rst}{PIDINST schema}, based on the evaluation of use cases collected by
the working group, is used for prototypical implementation of metadata
properties in the ePIC infrastructure.  A second variant provides a
\sphinxhref{https://github.com/rdawg-pidinst/schema/blob/master/schema-datacite.rst}{mapping between PIDINST metadata properties and DataCite Metadata
Schema 4.3}.  In the following, we
describe the properties in the original PIDINST schema and discuss
their semantics:
\begin{description}
\item[{\sphinxtitleref{Identifier}}] \leavevmode
\sphinxAtStartPar
The PID of the instrument.  The subproperty \sphinxtitleref{identifierType}
contains the type of the PID, e.g. \sphinxtitleref{Handle} or \sphinxtitleref{DOI} in the case of
an ePIC Handle or a DataCite DOI respectively.

\item[{\sphinxtitleref{SchemaVersion}}] \leavevmode
\sphinxAtStartPar
The version number of the PIDINST schema used to create a record.

\item[{\sphinxtitleref{LandingPage}}] \leavevmode
\sphinxAtStartPar
The URL of the landing page that the PID resolves to.

\item[{\sphinxtitleref{Name}}] \leavevmode
\sphinxAtStartPar
The name by which this instrument is known.  It should preferably be
meaningful and unique within the organization that manages it.

\item[{\sphinxtitleref{Owner}}] \leavevmode
\sphinxAtStartPar
The organization or individual that manages the instrument.  This
may or may not be the legal owner.  It could also be an organization
that hosts or operates the instrument, manages its deployment, or
provides access to it.  In case of doubt, it would be the instance
that a potential user would reach out to in order to get access.
There may be more then one owner registered in the metadata.

\sphinxAtStartPar
\sphinxtitleref{Owner} is a complex property having at least the subproperty
\sphinxtitleref{ownerName} and optionally a contact address in \sphinxtitleref{ownerContact} and a
common identifier of the owner in \sphinxtitleref{ownerIdentifier} and its type in
\sphinxtitleref{ownerIdentifierType}.

\item[{\sphinxtitleref{Manufacturer}}] \leavevmode
\sphinxAtStartPar
The organization or individual that built the instrument.  In the
case of an off the shelf product, this will probably be a commercial
company that put the instrument on the market.  In the case of an
custom built instrument, the manufacturer may be the same as the
owner.  In the latter case, they would be registered as both, owner
and manufacturer.  In case of doubt, the manufacturer would be the
instance that defined the technical specification of the instrument.
Again, there may be more then one manufacturer registered in the
metadata.

\sphinxAtStartPar
In the same way as \sphinxtitleref{Owner}, \sphinxtitleref{Manufacturer} is a complex property
with subproperties \sphinxtitleref{manufacturerName}, \sphinxtitleref{manufacturerIdentifier} and
\sphinxtitleref{manufacturerIdentifierType}.

\item[{\sphinxtitleref{Model}}] \leavevmode
\sphinxAtStartPar
The name of the model or type of the instrument.  In the case of an
off the shelf product, this may be a brand name attributed by the
manufacturer.  In the case of a custom built instrument, it may not
have a model name.  Hence this property is not mandatory, but
recommended if the value can be obtained.  \sphinxtitleref{Model} has optional
subproperties \sphinxtitleref{modelIdentifier} and \sphinxtitleref{modelIdentifierType} to be used
if an identifier for the model is known.

\item[{\sphinxtitleref{Description}}] \leavevmode
\sphinxAtStartPar
A textual description of the device and its capabilities.  This is
mostly targeted to a human reader and should provide a notion of
what this instrument is and what it can do.

\item[{\sphinxtitleref{InstrumentType}}] \leavevmode
\sphinxAtStartPar
A classification of the type of the instrument.  Hierarchical
classification enables grouping of instrument records.

\sphinxAtStartPar
In the same way as \sphinxtitleref{Owner}, \sphinxtitleref{Manufacturer} and \sphinxtitleref{Model},
\sphinxtitleref{InstrumentType} is a complex property with subproperties
\sphinxtitleref{instrumentTypeName}, \sphinxtitleref{instrumentTypeIdentifier} and
\sphinxtitleref{instrumentTypeIdentifierType}.

\item[{\sphinxtitleref{MeasuredVariable}}] \leavevmode
\sphinxAtStartPar
The variables or physical properties that the instrument measures or
observes.  Some communities have established terminologies to
identify measured variables that are specific to their respective
domain (see {\hyperref[\detokenize{white-paper/metadata-schema-recommendations:pidinst-metadata-schema-terminologies}]{\sphinxcrossref{\DUrole{std,std-ref}{Using common terminologies}}}}).  If such
a standard is applicable, it should be used for for
\sphinxtitleref{MeasuredVariable}.  Otherwise, a textual description may be used.

\item[{\sphinxtitleref{Date}}] \leavevmode
\sphinxAtStartPar
Relevant events pertaining to this instrument instance, such as when
it has started and ended to be in operation.  Each \sphinxtitleref{Date} need to
have a \sphinxtitleref{dateType} subproperty to specify the nature of the event.

\item[{\sphinxtitleref{RelatedIdentifier}}] \leavevmode
\sphinxAtStartPar
This can be used to establish links to related resources, such as
documents describing the instrument or external metadata records,
possibly using other metadata standards to provide more details
about the instrument.

\sphinxAtStartPar
Another application might be, if an instrument has been
substantially modified, it would make sense to issue a new PID for
the modified instrument with a new metadata record.  In this case
both PIDs should relate to each other to indicate that one is a new
version of the other.

\sphinxAtStartPar
Furthermore, in the case of a complex instrument, it can make sense
to issue PIDs for individual components, such as an individual
detector in a larger experimental station.  In this case, the
relation between the complex instrument and its components should be
established by creating links between the respective PIDs.

\sphinxAtStartPar
The links established using this property are particuarly useful as
they allow the automatic aggregation of a rich set of information
about the instrument.  Each \sphinxtitleref{RelatedIdentifier} needs to have
subproperties \sphinxtitleref{relatedIdentifierType} and \sphinxtitleref{relationType} to specify
the type of the related identifier and the type of the relation
respectively.

\item[{\sphinxtitleref{AlternateIdentifier}}] \leavevmode
\sphinxAtStartPar
If the instrument instance is also registered elsewhere, aside from
the persistent identifier, \sphinxtitleref{AlternateIdentifier} is the place to
store a reference to these register entries.  Common use cases are
the serial number attributed by the manufacturer or inventory number
used by the owner.  But also other instrument databases or access
portals may hold an entry for the instrument that should be
referenced from the PIDINST metadata.

\sphinxAtStartPar
The subproperty \sphinxtitleref{alternateIdentifierType} needs to specify the kind
of the alternate identifier.  Standardized values should be used
where applicable.  For serial and inventory numbers, the suggested
values are \sphinxtitleref{serialNumber} and \sphinxtitleref{inventoryNumber} respectively.

\end{description}


\subsection{Recommendations using the PIDINST metadata schema}
\label{\detokenize{white-paper/metadata-schema-recommendations:recommendations-using-the-pidinst-metadata-schema}}\label{\detokenize{white-paper/metadata-schema-recommendations:pidinst-metadata-schema-recommendations}}\label{\detokenize{white-paper/metadata-schema-recommendations::doc}}
\sphinxAtStartPar
We recommend that an instrument’s associated metadata is published in
a common language, specifically US English to enable its reuse.  In
the following section, we provide advanced recommendations on how to
specify the values in the PIDINST metadata and discuss special cases.


\subsubsection{Using common terminologies}
\label{\detokenize{white-paper/metadata-schema-recommendations:using-common-terminologies}}\label{\detokenize{white-paper/metadata-schema-recommendations:pidinst-metadata-schema-terminologies}}
\sphinxAtStartPar
Common terminologies such as controlled vocabularies, taxonomies or
ontologies, are sets of standardised terms that solve the problem of
ambiguities associated with metadata markup and enable records to be
shared and interpreted semantically by computers.  Many terminologies
exist, covering a broad spectrum of disciplines and their best
practices.  The PIDINST schema is designed to complement
multidisciplinary best practices for property values.  Many properties
allow for soft\sphinxhyphen{}typing (e.g., \sphinxstyleemphasis{ownerName}), giving users the ability to
use values of their choice, such as free text or domain\sphinxhyphen{}specific
terminologies.  Property attributes enable users and machines to
understand the context of the value (e.g., \sphinxstyleemphasis{ownerIdentifier},
\sphinxstyleemphasis{ownerIdentifierType}), again using free text or standardised
terminologies.  While free text is allowed, institutions should
consider using common terminologies where practical to enhance the
(semantic) interoperability of PID records, particularly where they
form part of domain\sphinxhyphen{}specific best practice.  For example, a
comprehensive set of terminologies that describe \sphinxstyleemphasis{instrumentType} (via
\sphinxstyleemphasis{instrumentTypeIdentifier}) or \sphinxstyleemphasis{Model} (via \sphinxstyleemphasis{modelIdentifier}) are
used widely in the Earth science marine domain
(\sphinxurl{http://vocab.nerc.ac.uk/collection/L22/current/},
\sphinxurl{http://vocab.nerc.ac.uk/collection/L05/current/}).
An example of the use of common terminologies in ePIC records is shown
in \hyperref[\detokenize{white-paper/metadata-schema-recommendations:tab-schema-handle-record}]{Table \ref{\detokenize{white-paper/metadata-schema-recommendations:tab-schema-handle-record}}}.


\begin{savenotes}\sphinxatlongtablestart\begin{longtable}[c]{|*{2}{\X{1}{2}|}}
\sphinxthelongtablecaptionisattop
\caption{Handle record of instrument identifier
       http://hdl.handle.net/21.T11998/0000\sphinxhyphen{}001A\sphinxhyphen{}3905\sphinxhyphen{}F displaying
       the use of common terminologies to identify instrument
       metadata compliant with the PIDINST schema as implemented
       by ePIC.  The terminologies used are published on the NERC
       Vocabulary Server (NVS).  The data for each
       metadata property is provided in JSON.  The Handle record
       can be viewed at
       http://hdl.handle.net/21.T11998/0000\sphinxhyphen{}001A\sphinxhyphen{}3905\sphinxhyphen{}F?noredirect\strut}\label{\detokenize{white-paper/metadata-schema-recommendations:tab-schema-handle-record}}\\*[\sphinxlongtablecapskipadjust]
\hline
\sphinxstyletheadfamily 
\sphinxAtStartPar
Type
&\sphinxstyletheadfamily 
\sphinxAtStartPar
Data
\\
\hline
\endfirsthead

\multicolumn{2}{c}%
{\makebox[0pt]{\sphinxtablecontinued{\tablename\ \thetable{} \textendash{} continued from previous page}}}\\
\hline
\sphinxstyletheadfamily 
\sphinxAtStartPar
Type
&\sphinxstyletheadfamily 
\sphinxAtStartPar
Data
\\
\hline
\endhead

\hline
\multicolumn{2}{r}{\makebox[0pt][r]{\sphinxtablecontinued{continues on next page}}}\\
\endfoot

\endlastfoot

\sphinxAtStartPar
URL
&
\begin{sphinxVerbatimintable}[commandchars=\\\{\}]
\PYG{l+s+s2}{\PYGZdq{}https://linkedsystems.uk/system/instance/TOOL0022\PYGZus{}2490/current/\PYGZdq{}}
\end{sphinxVerbatimintable}
\\
\hline
\begin{DUlineblock}{0em}
\item[] 21.T11148/8eb858ee0b12e8e463a5
\item[] (Identifier)
\end{DUlineblock}
&
\begin{sphinxVerbatimintable}[commandchars=\\\{\}]
\PYG{p}{\PYGZob{}}
  \PYG{n+nt}{\PYGZdq{}identifierValue\PYGZdq{}}\PYG{p}{:}\PYG{l+s+s2}{\PYGZdq{}http://hdl.handle.net/21.T11998/0000\PYGZhy{}001A\PYGZhy{}3905\PYGZhy{}F\PYGZdq{}}\PYG{p}{,}
  \PYG{n+nt}{\PYGZdq{}identifierType\PYGZdq{}}\PYG{p}{:}\PYG{l+s+s2}{\PYGZdq{}Handle\PYGZdq{}}
\PYG{p}{\PYGZcb{}}
\end{sphinxVerbatimintable}
\\
\hline
\begin{DUlineblock}{0em}
\item[] 21.T11148/aa24da8ba845c23ea75c
\item[] (SchemaVersion)
\end{DUlineblock}
&
\begin{sphinxVerbatimintable}[commandchars=\\\{\}]
\PYG{l+s+s2}{\PYGZdq{}1.0\PYGZdq{}}
\end{sphinxVerbatimintable}
\\
\hline
\begin{DUlineblock}{0em}
\item[] 21.T11148/e0efc41346cda4ba84ca
\item[] (LandingPage)
\end{DUlineblock}
&
\begin{sphinxVerbatimintable}[commandchars=\\\{\}]
\PYG{l+s+s2}{\PYGZdq{}https://linkedsystems.uk/system/instance/TOOL0022\PYGZus{}2490/current/\PYGZdq{}}
\end{sphinxVerbatimintable}
\\
\hline
\begin{DUlineblock}{0em}
\item[] 21.T11148/ab8d232261b9b60ba559
\item[] (Name)
\end{DUlineblock}
&
\begin{sphinxVerbatimintable}[commandchars=\\\{\}]
\PYG{l+s+s2}{\PYGZdq{}Sea\PYGZhy{}Bird SBE 37\PYGZhy{}IM MicroCAT C\PYGZhy{}T Sensor\PYGZdq{}}
\end{sphinxVerbatimintable}
\\
\hline
\begin{DUlineblock}{0em}
\item[] 21.T11148/4eaec4bc0f1df68ab2a7
\item[] (Owners)
\end{DUlineblock}
&
\begin{sphinxVerbatimintable}[commandchars=\\\{\}]
 \PYG{p}{[}
   \PYG{p}{\PYGZob{}}
     \PYG{n+nt}{\PYGZdq{}Owner\PYGZdq{}}\PYG{p}{:} \PYG{p}{\PYGZob{}}
       \PYG{n+nt}{\PYGZdq{}ownerName\PYGZdq{}}\PYG{p}{:}\PYG{l+s+s2}{\PYGZdq{}National Oceanography Centre\PYGZdq{}}\PYG{p}{,}
       \PYG{n+nt}{\PYGZdq{}ownerContact\PYGZdq{}}\PYG{p}{:}\PYG{l+s+s2}{\PYGZdq{}someone@example.org\PYGZdq{}}\PYG{p}{,}
       \PYG{n+nt}{\PYGZdq{}ownerIdentifier\PYGZdq{}}\PYG{p}{:} \PYG{p}{\PYGZob{}}
         \PYG{n+nt}{\PYGZdq{}ownerIdentifierValue\PYGZdq{}}\PYG{p}{:} \PYG{l+s+s2}{\PYGZdq{}http://vocab.nerc.ac.uk/collection/B75/current/ORG00009/\PYGZdq{}}\PYG{p}{,}
         \PYG{n+nt}{\PYGZdq{}ownerIdentifierType\PYGZdq{}}\PYG{p}{:}\PYG{l+s+s2}{\PYGZdq{}URL\PYGZdq{}}
      \PYG{p}{\PYGZcb{}}
    \PYG{p}{\PYGZcb{}}
  \PYG{p}{\PYGZcb{}}
\PYG{p}{]}
\end{sphinxVerbatimintable}
\\
\hline
\begin{DUlineblock}{0em}
\item[] 21.T11148/1f3e82ddf0697a497432
\item[] (Manufacturers)
\end{DUlineblock}
&
\begin{sphinxVerbatimintable}[commandchars=\\\{\}]
\PYG{p}{[}
  \PYG{p}{\PYGZob{}}
    \PYG{n+nt}{\PYGZdq{}Manufacturer\PYGZdq{}}\PYG{p}{:} \PYG{p}{\PYGZob{}}
      \PYG{n+nt}{\PYGZdq{}manufacturerName\PYGZdq{}}\PYG{p}{:}\PYG{l+s+s2}{\PYGZdq{}Sea\PYGZhy{}Bird Scientific\PYGZdq{}}\PYG{p}{,}
      \PYG{n+nt}{\PYGZdq{}manufacturerIdentifier\PYGZdq{}}\PYG{p}{:} \PYG{p}{\PYGZob{}}
        \PYG{n+nt}{\PYGZdq{}manufacturerIdentifierValue\PYGZdq{}}\PYG{p}{:} \PYG{l+s+s2}{\PYGZdq{}http://vocab.nerc.ac.uk/collection/L35/current/MAN0013/\PYGZdq{}}\PYG{p}{,}
        \PYG{n+nt}{\PYGZdq{}manufacturerIdentifierType\PYGZdq{}}\PYG{p}{:}\PYG{l+s+s2}{\PYGZdq{}URL\PYGZdq{}}
      \PYG{p}{\PYGZcb{}}
    \PYG{p}{\PYGZcb{}}
  \PYG{p}{\PYGZcb{}}
\PYG{p}{]}
\end{sphinxVerbatimintable}
\\
\hline
\begin{DUlineblock}{0em}
\item[] 21.T11148/c1a0ec5ad347427f25d6
\item[] (Model)
\end{DUlineblock}
&
\begin{sphinxVerbatimintable}[commandchars=\\\{\}]
 \PYG{p}{[}
   \PYG{p}{\PYGZob{}}
     \PYG{n+nt}{\PYGZdq{}modelName\PYGZdq{}}\PYG{p}{:}\PYG{l+s+s2}{\PYGZdq{}Sea\PYGZhy{}Bird SBE 37 MicroCat IM\PYGZhy{}CT with optional pressure (submersible) CTD sensor series\PYGZdq{}}\PYG{p}{,}
     \PYG{n+nt}{\PYGZdq{}modelIdentifier\PYGZdq{}}\PYG{p}{:} \PYG{p}{\PYGZob{}}
       \PYG{n+nt}{\PYGZdq{}modelIdentifierValue\PYGZdq{}}\PYG{p}{:} \PYG{l+s+s2}{\PYGZdq{}http://vocab.nerc.ac.uk/collection/L22/current/TOOL0022/\PYGZdq{}}\PYG{p}{,}
       \PYG{n+nt}{\PYGZdq{}modelIdentifierType\PYGZdq{}}\PYG{p}{:}\PYG{l+s+s2}{\PYGZdq{}URL\PYGZdq{}}
    \PYG{p}{\PYGZcb{}}
  \PYG{p}{\PYGZcb{}}
\PYG{p}{]}
\end{sphinxVerbatimintable}
\\
\hline
\begin{DUlineblock}{0em}
\item[] 21.T11148/f1627ce85386d8d75078
\item[] (Description)
\end{DUlineblock}
&
\begin{sphinxVerbatimintable}[commandchars=\\\{\}]
\PYG{l+s+s2}{\PYGZdq{}A high accuracy conductivity and temperature recorder with an optional}
\PYG{l+s+s2}{pressure sensor designed for deployment on moorings. The IM model has an}
\PYG{l+s+s2}{inductive modem for real\PYGZhy{}time data transmission plus internal flash memory}
\PYG{l+s+s2}{data storage.\PYGZdq{}}
\end{sphinxVerbatimintable}
\\
\hline
\begin{DUlineblock}{0em}
\item[] 21.T11148/c60c8da7fff2ef4f98ce
\item[] (InstrumentTypes)
\end{DUlineblock}
&
\begin{sphinxVerbatimintable}[commandchars=\\\{\}]
\PYG{p}{[}
  \PYG{p}{\PYGZob{}}
    \PYG{n+nt}{\PYGZdq{}InstrumentType\PYGZdq{}}\PYG{p}{:} \PYG{p}{\PYGZob{}}
      \PYG{n+nt}{\PYGZdq{}instrumentTypeName\PYGZdq{}}\PYG{p}{:}\PYG{l+s+s2}{\PYGZdq{}water temperature sensor\PYGZdq{}}\PYG{p}{,}
      \PYG{n+nt}{\PYGZdq{}instrumentTypeIdentifier\PYGZdq{}}\PYG{p}{:} \PYG{p}{\PYGZob{}}
        \PYG{n+nt}{\PYGZdq{}instrumentTypeIdentifierValue\PYGZdq{}}\PYG{p}{:}\PYG{l+s+s2}{\PYGZdq{}http://vocab.nerc.ac.uk/collection/L05/current/134/\PYGZdq{}}\PYG{p}{,}
        \PYG{n+nt}{\PYGZdq{}instrumentTypeIdentifierType\PYGZdq{}}\PYG{p}{:}\PYG{l+s+s2}{\PYGZdq{}URL\PYGZdq{}}
      \PYG{p}{\PYGZcb{}}
    \PYG{p}{\PYGZcb{}}
  \PYG{p}{\PYGZcb{}}\PYG{p}{,}
  \PYG{p}{\PYGZob{}}
    \PYG{n+nt}{\PYGZdq{}InstrumentType\PYGZdq{}}\PYG{p}{:} \PYG{p}{\PYGZob{}}
      \PYG{n+nt}{\PYGZdq{}instrumentTypeName\PYGZdq{}}\PYG{p}{:}\PYG{l+s+s2}{\PYGZdq{}salinity sensor\PYGZdq{}}\PYG{p}{,}
      \PYG{n+nt}{\PYGZdq{}InstrumentTypeIdentifier\PYGZdq{}}\PYG{p}{:}\PYG{p}{\PYGZob{}}
        \PYG{n+nt}{\PYGZdq{}instrumentTypeIdentifierValue\PYGZdq{}}\PYG{p}{:}\PYG{l+s+s2}{\PYGZdq{}http://vocab.nerc.ac.uk/collection/L05/current/350/\PYGZdq{}}\PYG{p}{,}
        \PYG{n+nt}{\PYGZdq{}instrumentTypeIdentifierType\PYGZdq{}}\PYG{p}{:}\PYG{l+s+s2}{\PYGZdq{}URL\PYGZdq{}}
      \PYG{p}{\PYGZcb{}}
    \PYG{p}{\PYGZcb{}}
  \PYG{p}{\PYGZcb{}}
\PYG{p}{]}
\end{sphinxVerbatimintable}
\\
\hline
\begin{DUlineblock}{0em}
\item[] 21.T11148/72928b84e060d491ee41
\item[] (MeasuredVariables)
\end{DUlineblock}
&
\begin{sphinxVerbatimintable}[commandchars=\\\{\}]
\PYG{p}{[}
  \PYG{p}{\PYGZob{}}
    \PYG{n+nt}{\PYGZdq{}MeasuredVariable\PYGZdq{}}\PYG{p}{:} \PYG{l+s+s2}{\PYGZdq{}http://vocab.nerc.ac.uk/collection/P01/current/CNDCPR01/\PYGZdq{}}
  \PYG{p}{\PYGZcb{}}\PYG{p}{,}
  \PYG{p}{\PYGZob{}}
    \PYG{n+nt}{\PYGZdq{}MeasuredVariable\PYGZdq{}}\PYG{p}{:} \PYG{l+s+s2}{\PYGZdq{}http://vocab.nerc.ac.uk/collection/P01/current/PSALPR01/\PYGZdq{}}
  \PYG{p}{\PYGZcb{}}\PYG{p}{,}
  \PYG{p}{\PYGZob{}}
    \PYG{n+nt}{\PYGZdq{}MeasuredVariable\PYGZdq{}}\PYG{p}{:} \PYG{l+s+s2}{\PYGZdq{}http://vocab.nerc.ac.uk/collection/P01/current/TEMPPR01/\PYGZdq{}}
  \PYG{p}{\PYGZcb{}}\PYG{p}{,}
  \PYG{p}{\PYGZob{}}
    \PYG{n+nt}{\PYGZdq{}MeasuredVariable\PYGZdq{}}\PYG{p}{:} \PYG{l+s+s2}{\PYGZdq{}http://vocab.nerc.ac.uk/collection/P01/current/PREXMCAT/\PYGZdq{}}
  \PYG{p}{\PYGZcb{}}
\PYG{p}{]}
\end{sphinxVerbatimintable}
\\
\hline
\begin{DUlineblock}{0em}
\item[] 21.T11148/22c62082a4d2d9ae2602
\item[] (Dates)
\end{DUlineblock}
&
\begin{sphinxVerbatimintable}[commandchars=\\\{\}]
\PYG{p}{[}
  \PYG{p}{\PYGZob{}}
    \PYG{n+nt}{\PYGZdq{}date\PYGZdq{}}\PYG{p}{:} \PYG{p}{\PYGZob{}}
      \PYG{n+nt}{\PYGZdq{}dateValue\PYGZdq{}}\PYG{p}{:}\PYG{l+s+s2}{\PYGZdq{}1999\PYGZhy{}11\PYGZhy{}01\PYGZdq{}}\PYG{p}{,}
      \PYG{n+nt}{\PYGZdq{}dateType\PYGZdq{}}\PYG{p}{:}\PYG{l+s+s2}{\PYGZdq{}Commissioned\PYGZdq{}}
    \PYG{p}{\PYGZcb{}}
  \PYG{p}{\PYGZcb{}}
\PYG{p}{]}
\end{sphinxVerbatimintable}
\\
\hline
\begin{DUlineblock}{0em}
\item[] 21.T11148/eb3c713572f681e6c4c3
\item[] (AlternateIdentifiers)
\end{DUlineblock}
&
\begin{sphinxVerbatimintable}[commandchars=\\\{\}]
\PYG{p}{[}
  \PYG{p}{\PYGZob{}}
    \PYG{n+nt}{\PYGZdq{}AlternateIdentifier\PYGZdq{}}\PYG{p}{:} \PYG{p}{\PYGZob{}}
      \PYG{n+nt}{\PYGZdq{}alternateIdentifierValue\PYGZdq{}}\PYG{p}{:}\PYG{l+s+s2}{\PYGZdq{}2490\PYGZdq{}}\PYG{p}{,}
      \PYG{n+nt}{\PYGZdq{}alternateIdentifierType\PYGZdq{}}\PYG{p}{:}\PYG{l+s+s2}{\PYGZdq{}serialNumber\PYGZdq{}}
    \PYG{p}{\PYGZcb{}}
  \PYG{p}{\PYGZcb{}}
\PYG{p}{]}
\end{sphinxVerbatimintable}
\\
\hline
\begin{DUlineblock}{0em}
\item[] 21.T11148/178fb558abc755ca7046
\item[] (RelatedIdentifiers)
\end{DUlineblock}
&
\begin{sphinxVerbatimintable}[commandchars=\\\{\}]
 \PYG{p}{[}
   \PYG{p}{\PYGZob{}}
     \PYG{n+nt}{\PYGZdq{}RelatedIdentifier\PYGZdq{}}\PYG{p}{:} \PYG{p}{\PYGZob{}}
       \PYG{n+nt}{\PYGZdq{}relatedIdentifierValue\PYGZdq{}}\PYG{p}{:}
         \PYG{l+s+s2}{\PYGZdq{}https://www.bodc.ac.uk/data/documents/nodb/pdf/37imbrochurejul08.pdf\PYGZdq{}}\PYG{p}{,}
       \PYG{n+nt}{\PYGZdq{}relatedIdentifierType\PYGZdq{}}\PYG{p}{:} \PYG{l+s+s2}{\PYGZdq{}URL\PYGZdq{}}\PYG{p}{,}
       \PYG{n+nt}{\PYGZdq{}relationType\PYGZdq{}}\PYG{p}{:}\PYG{l+s+s2}{\PYGZdq{}IsDescribedBy \PYGZdq{}}
    \PYG{p}{\PYGZcb{}}
  \PYG{p}{\PYGZcb{}}
\PYG{p}{]}
\end{sphinxVerbatimintable}
\\
\hline
\end{longtable}\sphinxatlongtableend\end{savenotes}


\subsubsection{Using other PIDs}
\label{\detokenize{white-paper/metadata-schema-recommendations:using-other-pids}}
\sphinxAtStartPar
The PIDINST metadata may contain references to related entities at
various places.  Obviously, these references should preferably use
persistent identifiers whenever applicable.  Different types of PIDs
are recommended depending on the nature of the referenced entity.  The
most common cases are:
\begin{itemize}
\item {} 
\sphinxAtStartPar
other instruments may be referenced in several cases.  The most
common PID types are Handles and DOIs here.

\item {} 
\sphinxAtStartPar
organizations that may appear as owner or manufacturer may be
referenced using a \sphinxhref{https://ror.org/}{ROR}.

\item {} 
\sphinxAtStartPar
the most common PID for individuals that may appear as owner or
manufacturer is the \sphinxhref{https://orcid.org/}{ORCID} iD.

\item {} 
\sphinxAtStartPar
the \sphinxhref{https://www.rrids.org/}{RRID} is common in the biological sciences and may be used to
reference a class of instruments, see next subsection.

\end{itemize}


\paragraph{RRIDs}
\label{\detokenize{white-paper/metadata-schema-recommendations:rrids}}
\sphinxAtStartPar
In a similar way to common terminologies, persistent identifiers have
been created to help users classify and accurately describe physical
objects.  The research resource identifier (RRID) can be used to
identify classes of instruments (models) and is thus related to
PIDINST, which identifies instrument instances.%
\begin{footnote}[1]\sphinxAtStartFootnote
Bandrowski A, Brush M, Grethe JS, Haendel MA, Kennedy DN, Hill S, Hof
PR, Martone ME, Pols M, Tan SC, Washington N, Zudilova\sphinxhyphen{}Seinstra E,
Vasilevsky N. \sphinxhref{https://pubmed.ncbi.nlm.nih.gov/26599696/}{The Resource Identification Initiative: A Cultural
Shift in Publishing.} J
Comp Neurol. 2016 Jan 1;524(1):8\sphinxhyphen{}22.
\sphinxurl{https://doi.org/10.1002/cne.23913}
%
\end{footnote}
This work is undertaken by the \sphinxhref{http://myweb.fsu.edu/aglerum/usedit/usedit-about.html}{UsedIT} group, which is extending the
RRID to instrument classes that could be used to describe the \sphinxstyleemphasis{Model}
(via \sphinxstyleemphasis{modelIdentifier}) property (\hyperref[\detokenize{white-paper/metadata-schema-recommendations:tab-schema-use-rrid}]{Table \ref{\detokenize{white-paper/metadata-schema-recommendations:tab-schema-use-rrid}}}).
RRIDs are not described in detail here, but it is envisioned that the
RRID metadata schema, which was described in detail
previously,%
\begin{footnote}[2]\sphinxAtStartFootnote
Bandrowski AE, Cachat J, Li Y, Müller HM, Sternberg PW, Ciccarese P,
Clark T, Marenco L, Wang R, Astakhov V, Grethe JS, Martone ME. A
hybrid human and machine resource curation pipeline for the
Neuroscience Information Framework. Database (Oxford). 2012 Mar
20;2012:bas005. \sphinxurl{https://doi.org/10.1093/database/bas005}
%
\end{footnote} and extended by UsedIT, will be
interoperable with instrument instance (PIDINST) PIDs.  This
interoperability should enable any project to quickly download data
about the model to consistently fill mapped fields.

\sphinxAtStartPar
Why RRIDs? RRIDs are currently used in about 1000 journals to tag
classes of research resources (including reagents like antibodies or
plasmids, organisms, cell lines, and a relatively broad category of
“tools” which includes software tools and services such as university
core facilities, but recently has been extended to physical tools such
as models of sequencers or microscopes).  Because RRIDs were created
as an agreement between a group of biological journals and the
National Institutes of Health, they are most commonly found and linked
in the biological sciences literature (e.g., Cell, eLife), they are
part of the JATS NISO standard, STAR Methods, and the MDAR
pan\sphinxhyphen{}publisher reproducibility checklist, resolved by identifiers.org
and the n2t resolver and echoed by some of the major reagent providers
(e.g., Thermo Fisher, Addgene, and the MMRRC mouse repository).


\begin{savenotes}\sphinxattablestart
\centering
\sphinxcapstartof{table}
\sphinxthecaptionisattop
\sphinxcaption{Example showing the use of RRIDs in the PIDINST metadata schema.}\label{\detokenize{white-paper/metadata-schema-recommendations:tab-schema-use-rrid}}
\sphinxaftertopcaption
\begin{tabulary}{\linewidth}[t]{|T|T|T|T|T|T|}
\hline
\sphinxstyletheadfamily 
\sphinxAtStartPar
ID
&\sphinxstyletheadfamily 
\sphinxAtStartPar
Property
&\sphinxstyletheadfamily 
\sphinxAtStartPar
Obligation
&\sphinxstyletheadfamily 
\sphinxAtStartPar
Occ.
&\sphinxstyletheadfamily 
\sphinxAtStartPar
Definition
&\sphinxstyletheadfamily 
\sphinxAtStartPar
Allowed values, constraints, remarks
\\
\hline
\sphinxAtStartPar
6
&
\sphinxAtStartPar
Model
&
\sphinxAtStartPar
R
&
\sphinxAtStartPar
0\sphinxhyphen{}1
&
\sphinxAtStartPar
Name of the model or type of device as attributed
by the manufacturer
&
\sphinxAtStartPar
Element
\\
\hline
\sphinxAtStartPar
6.1
&
\sphinxAtStartPar
modelName
&
\sphinxAtStartPar
R
&
\sphinxAtStartPar
1
&
\sphinxAtStartPar
Full name of the model
&
\sphinxAtStartPar
Name field from RRID

\sphinxAtStartPar
E.g.

\sphinxAtStartPar
‘Illumina HiSeq 3000/HiSeq 4000 System’
\\
\hline
\sphinxAtStartPar
6.2
&
\sphinxAtStartPar
modelIdentifier
&
\sphinxAtStartPar
O
&
\sphinxAtStartPar
0\sphinxhyphen{}1
&
\sphinxAtStartPar
Persistent identifier of the model
&
\sphinxAtStartPar
RRID identifier

\sphinxAtStartPar
E.g.

\sphinxAtStartPar
‘RRID:SCR\_016386’
\\
\hline
\sphinxAtStartPar
6.2.1
&
\sphinxAtStartPar
modelIdentifierType
&
\sphinxAtStartPar
O
&
\sphinxAtStartPar
1
&
\sphinxAtStartPar
Type of the identifier
&
\sphinxAtStartPar
Free text; must be identifier type

\sphinxAtStartPar
E.g. ‘RRID’
\\
\hline
\end{tabulary}
\par
\sphinxattableend\end{savenotes}


\subsubsection{Dealing with unknown information}
\label{\detokenize{white-paper/metadata-schema-recommendations:dealing-with-unknown-information}}
\sphinxAtStartPar
There are situations where it is not possible or not appropriate to
provide some piece of information that should normally be present in
the metadata.  This may for instance happen, if this information is
simply unknown, if a property has not or not yet been assigned a
value, or if it is not appropriate to disclose some piece of
information.  As an example for the latter case, consider a person
that contributes measurements to a citizen science project, but who
prefers to remain anonymous for privacy reasons.  That person might
not want to be named as the owner of the instrument taking the data.

\sphinxAtStartPar
In all these cases it is still useful to make it at least explicit
that this information has not been omitted inadvertently and also to
give a reason why it is missing.  For this purpose, PIDINST adopts the
\sphinxstyleemphasis{standard values for unknown information} from DataCite, see Appendix
3 in the DataCite Metadata Schema Documentation.%
\begin{footnote}[3]\sphinxAtStartFootnote
DataCite Metadata Working Group (2019).  DataCite Metadata Schema
Documentation for the Publication and Citation of Research Data.
Version 4.3.  DataCite e.V.  \sphinxurl{https://doi.org/10.14454/7xq3-zf69}
%
\end{footnote}
\sphinxSetupCaptionForVerbatim{Encoding unknown values in the instrument PID metadata using XML}
\def\sphinxLiteralBlockLabel{\label{\detokenize{white-paper/metadata-schema-recommendations:snip-schema-unknown-xml}}}
\begin{sphinxVerbatim}[commandchars=\\\{\}]
  \PYG{n+nt}{\PYGZlt{}name}\PYG{n+nt}{\PYGZgt{}}:tba\PYG{n+nt}{\PYGZlt{}/name\PYGZgt{}}
  \PYG{n+nt}{\PYGZlt{}owners}\PYG{n+nt}{\PYGZgt{}}
     \PYG{n+nt}{\PYGZlt{}owner}\PYG{n+nt}{\PYGZgt{}}
        \PYG{n+nt}{\PYGZlt{}ownerName}\PYG{n+nt}{\PYGZgt{}}:unal\PYG{n+nt}{\PYGZlt{}/ownerName\PYGZgt{}}
     \PYG{n+nt}{\PYGZlt{}/owner\PYGZgt{}}
  \PYG{n+nt}{\PYGZlt{}/owners\PYGZgt{}}
  \PYG{n+nt}{\PYGZlt{}manufacturers}\PYG{n+nt}{\PYGZgt{}}
     \PYG{n+nt}{\PYGZlt{}manufacturer}\PYG{n+nt}{\PYGZgt{}}
        \PYG{n+nt}{\PYGZlt{}manufacturerName}\PYG{n+nt}{\PYGZgt{}}:unav\PYG{n+nt}{\PYGZlt{}/manufacturerName\PYGZgt{}}
     \PYG{n+nt}{\PYGZlt{}/manufacturer\PYGZgt{}}
  \PYG{n+nt}{\PYGZlt{}/manufacturers\PYGZgt{}}
\end{sphinxVerbatim}

\sphinxAtStartPar
\hyperref[\detokenize{white-paper/metadata-schema-recommendations:snip-schema-unknown-xml}]{Snippet \ref{\detokenize{white-paper/metadata-schema-recommendations:snip-schema-unknown-xml}}} demonstrates the use of standard
values for unknown information in the metadata of an instrument PID.
It shows an instrument that has not yet been assigned a name, e.g. it
may be assumed that the metadata record will be updated at a later
point in time including a name.  The owner of the instrument is
refused to be disclosed and the manufacturer is not known.


\subsection{Registration}
\label{\detokenize{white-paper/registration:registration}}\label{\detokenize{white-paper/registration::doc}}

\subsubsection{Central registration at PID providers}
\label{\detokenize{white-paper/registration:central-registration-at-pid-providers}}
\sphinxAtStartPar
The following resources (\hyperref[\detokenize{white-paper/registration:tab-register-guidance}]{Table \ref{\detokenize{white-paper/registration:tab-register-guidance}}}) provide
technical guidance for institutions to publish and manage PID records
at PID providers compliant with RDA PIDINST recommendations.


\begin{savenotes}\sphinxattablestart
\centering
\sphinxcapstartof{table}
\sphinxthecaptionisattop
\sphinxcaption{Technical guidance for publishing and managing instrument
       PIDs at PID providers compliant with RDA PIDINST
       recommendations. The table provides links to the relevant
       metadata schema that accompanies PID records at PID
       providers.}\label{\detokenize{white-paper/registration:tab-register-guidance}}
\sphinxaftertopcaption
\begin{tabulary}{\linewidth}[t]{|T|T|T|}
\hline
\sphinxstyletheadfamily 
\sphinxAtStartPar
PID provider
&\sphinxstyletheadfamily 
\sphinxAtStartPar
Technical resource
&\sphinxstyletheadfamily 
\sphinxAtStartPar
Metadata schema
\\
\hline
\sphinxAtStartPar
ePIC
&
\sphinxAtStartPar
{\hyperref[\detokenize{epic-cookbook/index:epic-cookbook}]{\sphinxcrossref{\DUrole{std,std-ref}{ePIC Cookbook}}}}
&
\sphinxAtStartPar
\sphinxhref{https://github.com/rdawg-pidinst/schema/blob/master/schema.rst}{PIDINST}
\\
\hline
\sphinxAtStartPar
DataCite
&
\sphinxAtStartPar
{\hyperref[\detokenize{datacite-cookbook/index:datacite-cookbook}]{\sphinxcrossref{\DUrole{std,std-ref}{DataCite Cookbook}}}}
&
\sphinxAtStartPar
\sphinxhref{https://github.com/rdawg-pidinst/schema/blob/master/schema-datacite.rst}{PIDINST to DataCite}
\\
\hline
\end{tabulary}
\par
\sphinxattableend\end{savenotes}


\subsubsection{Local registration at institutional instrument providers}
\label{\detokenize{white-paper/registration:local-registration-at-institutional-instrument-providers}}
\sphinxAtStartPar
In order to register instrument PIDs at a provider service,
institutional instrument providers must publish a landing page for each
instrument PID to resolve to. These publications might be encoded using
standard markup languages (e.g. HTML), structured, machine\sphinxhyphen{}actionable
web resources (e.g. World Wide Consortium’s (W3C) Linked Data), or
specialist standards for describing instruments and their inherited
properties and processes (e.g. Open Geospatial Consortium’s (OGC)
SensorML, W3C Semantic Sensor Network (SSN) ontology). Whichever method
of publication is used, it is necessary to ensure there is enough
metadata on landing pages to unambiguously identify the instrument (see
{\hyperref[\detokenize{white-paper/landing-page-content:landing-page-content}]{\sphinxcrossref{\DUrole{std,std-ref}{Landing page content}}}}). The URL address is also used to populate
the \sphinxstyleemphasis{LandingPage} property of the \sphinxhref{https://github.com/rdawg-pidinst/schema/blob/master/schema.rst}{PIDINST schema}, adding this
locator to the PID’s metadata record.


\subsection{Dealing with duplication}
\label{\detokenize{white-paper/duplication:dealing-with-duplication}}\label{\detokenize{white-paper/duplication::doc}}
\sphinxAtStartPar
Duplication between identifier records is not a new problem and is
common to many applications (e.g. bibliographic, medical records).
While PIDINST identifiers are considered globally persistent it is
accepted that duplication may occur particularly where institutions
loan instruments to other organisations or provide access to
specialised facilities (e.g. large scale synchrotrons, medical
laboratories, computational facilities).  Such duplication may lead to
inaccurate statistics or reporting about instrument assets.

\sphinxAtStartPar
It is recommended that owners of instruments try to employ workflows
and procedures that avoid duplication in the first instance.  Where
this has not been possible, it is recommended that instrument owners
employ deduplication, effectively merging duplicate records into one
representative record by ensuring links between them.  This can be
achieved using the PIDINST metadata schema \sphinxstyleemphasis{relatedIdentifier}
property with a \sphinxstyleemphasis{relationType} attribute \sphinxstyleemphasis{IsIdenticalTo} as shown in
\hyperref[\detokenize{white-paper/duplication:snip-dub-merge-xml}]{Snippet \ref{\detokenize{white-paper/duplication:snip-dub-merge-xml}}} and \hyperref[\detokenize{white-paper/duplication:snip-dub-merge-json}]{\ref{\detokenize{white-paper/duplication:snip-dub-merge-json}}}.
\sphinxSetupCaptionForVerbatim{Merging duplicate instrument PID records using XML}
\def\sphinxLiteralBlockLabel{\label{\detokenize{white-paper/duplication:snip-dub-merge-xml}}}
\begin{sphinxVerbatim}[commandchars=\\\{\}]
  \PYG{n+nt}{\PYGZlt{}relatedIdentifiers}\PYG{n+nt}{\PYGZgt{}}
     \PYG{n+nt}{\PYGZlt{}relatedIdentifier} \PYG{n+na}{relatedIdentifierType=}\PYG{l+s}{\PYGZdq{}DOI\PYGZdq{}} \PYG{n+na}{relationType=}\PYG{l+s}{\PYGZdq{}IsIdenticalTo\PYGZdq{}}\PYG{n+nt}{\PYGZgt{}}10.4232/10.CPoS\PYGZhy{}2013\PYGZhy{}02en\PYG{n+nt}{\PYGZlt{}/relatedIdentifier\PYGZgt{}}
  \PYG{n+nt}{\PYGZlt{}/relatedIdentifiers\PYGZgt{}}
\end{sphinxVerbatim}
\sphinxSetupCaptionForVerbatim{Merging duplicate instrument PID records using JSON}
\def\sphinxLiteralBlockLabel{\label{\detokenize{white-paper/duplication:snip-dub-merge-json}}}
\begin{sphinxVerbatim}[commandchars=\\\{\}]
\PYG{p}{[}\PYG{p}{\PYGZob{}}
  \PYG{n+nt}{\PYGZdq{}RelatedIdentifier\PYGZdq{}}\PYG{p}{:}\PYG{p}{\PYGZob{}}
    \PYG{n+nt}{\PYGZdq{}RelatedIdentifierValue\PYGZdq{}}\PYG{p}{:}\PYG{l+s+s2}{\PYGZdq{}10.4232/10.CPoS\PYGZhy{}2013\PYGZhy{}02en\PYGZdq{}}\PYG{p}{,}
    \PYG{n+nt}{\PYGZdq{}RelatedIdentifierType\PYGZdq{}}\PYG{p}{:} \PYG{l+s+s2}{\PYGZdq{}DOI\PYGZdq{}}\PYG{p}{,}
    \PYG{n+nt}{\PYGZdq{}relationType\PYGZdq{}}\PYG{p}{:}\PYG{l+s+s2}{\PYGZdq{}IsIdenticalTo\PYGZdq{}}
  \PYG{p}{\PYGZcb{}}
\PYG{p}{\PYGZcb{}}\PYG{p}{]}
\end{sphinxVerbatim}

\sphinxAtStartPar
Recent advances in technologies are expanding to algorithms that
automatically detect and resolve deduplication.  While such
methodologies have been known to improve the efficiency of detection
in large collections such as Google Scholar or \sphinxhref{https://graph.openaire.eu}{OpenAIRE Graph},
algorithms may be limited by heterogeneous representations for
example, by the use of differing semantics.  While automatic detection
is encouraged, the PIDINST schema is designed to complement
multidisciplinary best practices for property values and many
properties allow for soft\sphinxhyphen{}typing, giving users the ability to use
values of their choice, such as free text or domain\sphinxhyphen{}specific
standards.


\subsection{Linking physical objects}
\label{\detokenize{white-paper/physical-objects:linking-physical-objects}}\label{\detokenize{white-paper/physical-objects::doc}}
\sphinxAtStartPar
Instruments and their individual configuration represent the major
reference for the origin of a broad spectrum of data. As such, both
become part of the Internet of Things (IoT) and therefore it is of key
importance for related identification mechanisms to enable physical
access to these objects in addition to their digital representations or
catalogue metadata. Thus, to ultimately allow the “mapping the real
world into the virtual world”.%
\begin{footnote}[1]\sphinxAtStartFootnote
Atzori, Luigi \& Iera, Antonio \& Morabito, Giacomo. (2010). The
Internet of Things: A Survey. Computer Networks. 2787\sphinxhyphen{}2805.
10.1016/j.comnet.2010.05.010.
%
\end{footnote} This kind of access is
essential to reproduce science as it allows us to compare experimental
setup and to repeat analyses.

\sphinxAtStartPar
The most trivial but failsafe method to link physical objects with their
virtual representation would be to permanently label an instrument by
writing or engraving its PID onto it or its container along with its
inventory number and serial number. Because space for labels is limited
on smaller sensors, modern QR tags or barcodes may be more convenient as
they offer the possibility to encode any identifying information in a
machine readable way. A recommended way would be to use QR codes to
embed a PID’s actionable URIs (\hyperref[\detokenize{white-paper/physical-objects:fig-objects-qr}]{Figure \ref{\detokenize{white-paper/physical-objects:fig-objects-qr}}}). Ideally such
a QR badge additionally displays the PID as well as the inventory
number and serial number in a human readable way. Some QR code
generators now allow users to integrate images like organisation logos
or track scanning activity such as the GPS position when the label is
scanned.

\sphinxAtStartPar
In case neither labelling of physical objects with barcodes or PID
strings is possible, linking of instruments with their digital
representation can be maintained by providing appropriate metadata
records. For instruments such linking can be achieved by capturing
identifiers which uniquely identify an instrument such as serial number
or inventory number.

\sphinxAtStartPar
While PIDINST schema metadata does not provide explicit fields for
serial numbers or inventory numbers, it currently offers a generic way
to capture any kind of identifier which can be used for this purpose.
\sphinxstyleemphasis{AlternateIdentifier} can be used to record any identifier string and
\sphinxstyleemphasis{alternateIdentifierType} to specify an identifier type
(\hyperref[\detokenize{white-paper/physical-objects:snip-objects-serial}]{Snippet \ref{\detokenize{white-paper/physical-objects:snip-objects-serial}}}). PIDINST schema recommends the use of
the terms \sphinxstyleemphasis{serialNumber} and \sphinxstyleemphasis{inventoryNumber.} There is on\sphinxhyphen{}going
discussion regarding the use of explicit fields for these properties
in PIDINST.

\begin{figure}[htbp]
\centering
\capstart

\noindent\sphinxincludegraphics{{image4}.png}
\caption{An example of a webpage QR code that includes an organisation logo
and re\sphinxhyphen{}directs the scanner to the PID URL
(\sphinxurl{http://hdl.handle.net/21.T11998/0000-001A-3905-F}).}\label{\detokenize{white-paper/physical-objects:fig-objects-qr}}\end{figure}
\sphinxSetupCaptionForVerbatim{An instrument serial number expressed in XML}
\def\sphinxLiteralBlockLabel{\label{\detokenize{white-paper/physical-objects:snip-objects-serial}}}
\begin{sphinxVerbatim}[commandchars=\\\{\}]
  \PYG{n+nt}{\PYGZlt{}AlternateIdentifiers}\PYG{n+nt}{\PYGZgt{}}
     \PYG{n+nt}{\PYGZlt{}AlternateIdentifier} \PYG{n+na}{alternateIdentifierType=}\PYG{l+s}{\PYGZdq{}serialNumber\PYGZdq{}}\PYG{n+nt}{\PYGZgt{}}7351\PYGZhy{}349l\PYGZhy{}mn24\PYGZhy{}019f\PYG{n+nt}{\PYGZlt{}/AlternateIdentifier\PYGZgt{}}
  \PYG{n+nt}{\PYGZlt{}/AlternateIdentifiers\PYGZgt{}}
\end{sphinxVerbatim}

\sphinxAtStartPar
Besides storing e.g. serial numbers in PIDINST schema metadata, it is
highly recommended to store the instrument PID within an institutional
sensor management or inventory system immediately after PID
registration. This ensures the maintenance of links between physical
objects and their virtual representation at both endpoints, the
institutional sensor management system as well as the PID registry, and
will ensure the persistence of object linking in case of failures on
either side.


\subsection{When to create a new PID?}
\label{\detokenize{white-paper/create-new-pid:when-to-create-a-new-pid}}\label{\detokenize{white-paper/create-new-pid::doc}}
\sphinxAtStartPar
Instruments can be changed or modified over time. For example, when a
component is changed or an instrument is upgraded to meet new
requirements in measurement capability. Defining the exact moment when a
new PID should be created is challenging because different stakeholders
will have different reasons for each evolution. Indeed the PIDINST WG
has not been able to settle on a definitive answer. Thus to accommodate
varying stakeholder needs, it is recommended that a PID will evolve when
there is a significant change in context that is important to an
institutional instrument provider. Significant changes might include
when an instrument is cited in the literature and changes, there is a
need to preserve the instrument history, major changes in measurement
capability that affect automated workflows such as quality control, or
modifications to an instrument’s firmware etc. Whatever the reason an
institution chooses to create new PIDs, it is recommended that
instrument providers identify the succession in the PIDINST metadata
schema using the \sphinxstyleemphasis{relatedIdentifier} property with a \sphinxstyleemphasis{relationType}
attribute \sphinxstyleemphasis{IsNewVersionOf} for the new PID and, \sphinxstyleemphasis{IsPreviousVersionOf}
for the superceded PID as shown in
\hyperref[\detokenize{white-paper/create-new-pid:snip-create-superseding-xml}]{Snippet \ref{\detokenize{white-paper/create-new-pid:snip-create-superseding-xml}}},
\hyperref[\detokenize{white-paper/create-new-pid:snip-create-superseded-xml}]{\ref{\detokenize{white-paper/create-new-pid:snip-create-superseded-xml}}},
\hyperref[\detokenize{white-paper/create-new-pid:snip-create-superseding-json}]{\ref{\detokenize{white-paper/create-new-pid:snip-create-superseding-json}}}, and
\hyperref[\detokenize{white-paper/create-new-pid:snip-create-superseded-json}]{\ref{\detokenize{white-paper/create-new-pid:snip-create-superseded-json}}}.
\sphinxSetupCaptionForVerbatim{The use of the relatedIdentifier property to represent
superseding PID records in XML}
\def\sphinxLiteralBlockLabel{\label{\detokenize{white-paper/create-new-pid:snip-create-superseding-xml}}}
\begin{sphinxVerbatim}[commandchars=\\\{\}]
  \PYG{n+nt}{\PYGZlt{}relatedIdentifiers}\PYG{n+nt}{\PYGZgt{}}
     \PYG{n+nt}{\PYGZlt{}relatedIdentifier} \PYG{n+na}{relatedIdentifierType=}\PYG{l+s}{\PYGZdq{}DOI\PYGZdq{}} \PYG{n+na}{relationType=}\PYG{l+s}{\PYGZdq{}IsNewVersionOf\PYGZdq{}}\PYG{n+nt}{\PYGZgt{}}10.4232/10.CPoS\PYGZhy{}2013\PYGZhy{}02en\PYG{n+nt}{\PYGZlt{}/relatedIdentifier\PYGZgt{}}
  \PYG{n+nt}{\PYGZlt{}/relatedIdentifiers\PYGZgt{}}
\end{sphinxVerbatim}
\sphinxSetupCaptionForVerbatim{The use of the relatedIdentifier property to represent
superseded PID records in XML}
\def\sphinxLiteralBlockLabel{\label{\detokenize{white-paper/create-new-pid:snip-create-superseded-xml}}}
\begin{sphinxVerbatim}[commandchars=\\\{\}]
  \PYG{n+nt}{\PYGZlt{}relatedIdentifiers}\PYG{n+nt}{\PYGZgt{}}
     \PYG{n+nt}{\PYGZlt{}relatedIdentifier} \PYG{n+na}{relatedIdentifierType=}\PYG{l+s}{\PYGZdq{}DOI\PYGZdq{}} \PYG{n+na}{relationType=}\PYG{l+s}{\PYGZdq{}IsPreviousVersionOf\PYGZdq{}}\PYG{n+nt}{\PYGZgt{}}http://hdl.handle.net/21.T11998/0000\PYGZhy{}001A\PYGZhy{}3905\PYGZhy{}F\PYG{n+nt}{\PYGZlt{}/relatedIdentifier\PYGZgt{}}
  \PYG{n+nt}{\PYGZlt{}/relatedIdentifiers\PYGZgt{}}
\end{sphinxVerbatim}
\sphinxSetupCaptionForVerbatim{The use of the relatedIdentifier property to represent
superseding PID records in JSON}
\def\sphinxLiteralBlockLabel{\label{\detokenize{white-paper/create-new-pid:snip-create-superseding-json}}}
\begin{sphinxVerbatim}[commandchars=\\\{\}]
  \PYG{p}{[}\PYG{p}{\PYGZob{}}
    \PYG{n+nt}{\PYGZdq{}RelatedIdentifier\PYGZdq{}}\PYG{p}{:}\PYG{p}{\PYGZob{}}
      \PYG{n+nt}{\PYGZdq{}RelatedIdentifierValue\PYGZdq{}}\PYG{p}{:}\PYG{l+s+s2}{\PYGZdq{}10.4232/10.CPoS\PYGZhy{}2013\PYGZhy{}02en\PYGZdq{}}\PYG{p}{,}
      \PYG{n+nt}{\PYGZdq{}RelatedIdentifierType\PYGZdq{}}\PYG{p}{:} \PYG{l+s+s2}{\PYGZdq{}DOI\PYGZdq{}}\PYG{p}{,}
      \PYG{n+nt}{\PYGZdq{}relationType\PYGZdq{}}\PYG{p}{:}\PYG{l+s+s2}{\PYGZdq{}IsNewVersionOf\PYGZdq{}}
    \PYG{p}{\PYGZcb{}}
  \PYG{p}{\PYGZcb{}}\PYG{p}{]}
\end{sphinxVerbatim}
\sphinxSetupCaptionForVerbatim{The use of the relatedIdentifier property to represent
superseded PID records in JSON}
\def\sphinxLiteralBlockLabel{\label{\detokenize{white-paper/create-new-pid:snip-create-superseded-json}}}
\begin{sphinxVerbatim}[commandchars=\\\{\}]
  \PYG{p}{[}\PYG{p}{\PYGZob{}}
    \PYG{n+nt}{\PYGZdq{}RelatedIdentifier\PYGZdq{}}\PYG{p}{:}\PYG{p}{\PYGZob{}}
      \PYG{n+nt}{\PYGZdq{}RelatedIdentifierValue\PYGZdq{}}\PYG{p}{:}\PYG{l+s+s2}{\PYGZdq{}http://hdl.handle.net/21.T11998/0000\PYGZhy{}001A\PYGZhy{}3905\PYGZhy{}F\PYGZdq{}}\PYG{p}{,}
      \PYG{n+nt}{\PYGZdq{}RelatedIdentifierType\PYGZdq{}}\PYG{p}{:} \PYG{l+s+s2}{\PYGZdq{}DOI\PYGZdq{}}\PYG{p}{,}
      \PYG{n+nt}{\PYGZdq{}relationType\PYGZdq{}}\PYG{p}{:}\PYG{l+s+s2}{\PYGZdq{}IsPreviousVersionOf\PYGZdq{}}
    \PYG{p}{\PYGZcb{}}
  \PYG{p}{\PYGZcb{}}\PYG{p}{]}
\end{sphinxVerbatim}


\subsection{Landing page content}
\label{\detokenize{white-paper/landing-page-content:landing-page-content}}\label{\detokenize{white-paper/landing-page-content:id1}}\label{\detokenize{white-paper/landing-page-content::doc}}
\sphinxAtStartPar
It is recommended that instrument providers use enough information
(metadata) on landing pages to unambiguously identify the instrument.
Ideally, landing pages should include the metadata specified in the
schema for PID providers and use common terminology where practical to
aid interoperability (see {\hyperref[\detokenize{white-paper/metadata-schema-recommendations:pidinst-metadata-schema-terminologies}]{\sphinxcrossref{\DUrole{std,std-ref}{Using common terminologies}}}}).
Institutions should also consider providing links to the metadata record
that accompanies PIDs published at PID providers to aid metadata
exchange (e.g. DataCite XML).

\sphinxAtStartPar
\hyperref[\detokenize{white-paper/landing-page-content:tab-landing-content-inst}]{Tables \ref{\detokenize{white-paper/landing-page-content:tab-landing-content-inst}}} and
\hyperref[\detokenize{white-paper/landing-page-content:tab-landing-content-events}]{\ref{\detokenize{white-paper/landing-page-content:tab-landing-content-events}}} provide recommendations for
some additional, more descriptive metadata that can be published on
landing pages. Together with the PIDINST metadata schema, they are
designed to complement the administration and discovery of
instruments; to enable users to put data into context; and to automate
instrument metadata into data workflows.


\begin{savenotes}\sphinxattablestart
\centering
\sphinxcapstartof{table}
\sphinxthecaptionisattop
\sphinxcaption{Descriptive landing page metadata describing measuring
       instruments. To be used in conjunction with the core
       instrument metadata used in the PIDINST schema.}\label{\detokenize{white-paper/landing-page-content:tab-landing-content-inst}}
\sphinxaftertopcaption
\begin{tabulary}{\linewidth}[t]{|T|T|}
\hline
\sphinxstyletheadfamily 
\sphinxAtStartPar
\sphinxstylestrong{Metadata type}
&\sphinxstyletheadfamily 
\sphinxAtStartPar
\sphinxstylestrong{Comments}
\\
\hline
\sphinxAtStartPar
Model version
&
\sphinxAtStartPar
A variant of an instrument model. While the
design of an instrument remains largely the
same, variants are available with minor changes
to suit different applications. For example, an
instrument may be available with different
housing material from the standard design,
allowing the instrument to be used in more
dynamic environments such as extreme pressures
or weather conditions.
\\
\hline
\sphinxAtStartPar
Documents
&
\sphinxAtStartPar
Descriptive or supporting documentation such as
manuals, data sheets, scientific references
etc.
\\
\hline
\sphinxAtStartPar
Classifications
&
\sphinxAtStartPar
Properties that categorise instruments. In
addition to instrument type, these properties
can describe aspects such as the intended
applications, operating principles, whether the
instrument is a compound instrument or a
component etc.
\\
\hline
\end{tabulary}
\par
\sphinxattableend\end{savenotes}


\begin{savenotes}\sphinxattablestart
\centering
\sphinxcapstartof{table}
\sphinxthecaptionisattop
\sphinxcaption{Descriptive, landing page metadata that describes the
       history of events, operations or changes during the
       lifetime of an instrument. This kind of metadata should be
       associated to dates and ideally accompanied by comments. To
       be used in conjunction with the core instrument metadata
       used in the PIDINST schema.}\label{\detokenize{white-paper/landing-page-content:tab-landing-content-events}}
\sphinxaftertopcaption
\begin{tabulary}{\linewidth}[t]{|T|T|}
\hline
\sphinxstyletheadfamily 
\sphinxAtStartPar
\sphinxstylestrong{Metadata type}
&\sphinxstyletheadfamily 
\sphinxAtStartPar
\sphinxstylestrong{Comments}
\\
\hline
\sphinxAtStartPar
Calibrations
&
\sphinxAtStartPar
Many instruments are calibrated to convert raw
outputs to meaningful units or to correct for
data uncertainty. It is highly recommended to
store the calibration date and type. It may
also be useful to store the coefficients,
algorithm used and calibration certificates.
\\
\hline
\sphinxAtStartPar
Capabilities
&
\sphinxAtStartPar
Capabilities are properties that further
quantify or qualify an instrument’s outputs
(e.g. detection limits, accuracy, precision,
operating ranges etc.).
\\
\hline
\sphinxAtStartPar
Characteristics
&
\sphinxAtStartPar
Properties that describe features and
qualities belonging to an instrument. (e.g.
weight, size, housing material, components,
firmware etc.).
\\
\hline
\sphinxAtStartPar
Servicing
&
\sphinxAtStartPar
Descriptions of maintenance procedures carried
out on the instrument.
\\
\hline
\sphinxAtStartPar
Funding references
&
\sphinxAtStartPar
Identifiers or names of funding resources
\\
\hline
\sphinxAtStartPar
Ownership dates
&
\sphinxAtStartPar
Ownership start and end dates
\\
\hline
\end{tabulary}
\par
\sphinxattableend\end{savenotes}


\subsection{Landing page encoding}
\label{\detokenize{white-paper/landing-page-encoding:landing-page-encoding}}\label{\detokenize{white-paper/landing-page-encoding:id1}}\label{\detokenize{white-paper/landing-page-encoding::doc}}
\sphinxAtStartPar
Landing page web resources can be written in any format (e.g. HTML,
XML). Although not obligatory, ideally resources should be encoded in
formats that not only improve syntactic interpretation of information
but semantic understanding of the information. In other words, machines
can not only read but understand the meaning of the information
presented in web resources, enhancing interoperability and integration
between systems. Below are some examples of landing page encodings.


\subsubsection{Examples}
\label{\detokenize{white-paper/landing-page-encoding:examples}}

\paragraph{JSON\sphinxhyphen{}LD}
\label{\detokenize{white-paper/landing-page-encoding:json-ld}}
\sphinxAtStartPar
There is a strong relation between PIDs with values of types that are
defined in a data type registry (DTR) as for instance in the
example in \hyperref[\detokenize{white-paper/metadata-schema-recommendations:tab-schema-handle-record}]{Table \ref{\detokenize{white-paper/metadata-schema-recommendations:tab-schema-handle-record}}} and linked data. First
of all a PID with a type value is a triple where the PID plays the
role of the subject, the type definition is the predicate and the
value is the object. Secondly the type definition itself can refer to
sub types also defined in a DTR. If this construction of types out of
other types is done in a consistent and machine actionable way, as it
is done for instance in the ePC DTR, these subtypes may be referred by
human readable names. The names are disambiguated by the type
definition, because each subtype used in a type is identified by a
PID. Such PIDs with types defined upon sub types span a graph of
metadata around the PID. PIDs with types are in other words a specific
representation of linked data.

\sphinxAtStartPar
It is therefore obvious to ask for other, more a common linked data
representation like RDF or JSON\sphinxhyphen{}LD of such PIDs with types. Such a
conversion can be done by a simple backtracking algorithm that crawls
from the PID through all its type and subtypes definitions to identify
the used names by the type PIDs and to collect this information for the
LD representation. This way the whole graph is explored and this graph
can be mapped into a linked data representation. In the following a
respective representation in JSON\sphinxhyphen{}LD of the schema example shown in
\hyperref[\detokenize{white-paper/metadata-schema-recommendations:tab-schema-handle-record}]{Table \ref{\detokenize{white-paper/metadata-schema-recommendations:tab-schema-handle-record}}} is shown in
\hyperref[\detokenize{white-paper/landing-page-encoding:snip-landing-encoding-json-ld}]{Snippet \ref{\detokenize{white-paper/landing-page-encoding:snip-landing-encoding-json-ld}}}.
\sphinxSetupCaptionForVerbatim{representation in JSON\sphinxhyphen{}LD of the example of
\hyperref[\detokenize{white-paper/metadata-schema-recommendations:tab-schema-handle-record}]{Table \ref{\detokenize{white-paper/metadata-schema-recommendations:tab-schema-handle-record}}}.}
\def\sphinxLiteralBlockLabel{\label{\detokenize{white-paper/landing-page-encoding:snip-landing-encoding-json-ld}}}
\begin{sphinxVerbatim}[commandchars=\\\{\}]
  \PYG{p}{\PYGZob{}}
    \PYG{n+nt}{\PYGZdq{}@context\PYGZdq{}}\PYG{p}{:} \PYG{p}{\PYGZob{}}
      \PYG{n+nt}{\PYGZdq{}ARK\PYGZhy{}Identifier\PYGZdq{}}\PYG{p}{:} \PYG{l+s+s2}{\PYGZdq{}dti:21.T11148/7af6f46512fb4c190d01\PYGZdq{}}\PYG{p}{,}
      \PYG{n+nt}{\PYGZdq{}AlternateIdentifier\PYGZdq{}}\PYG{p}{:} \PYG{l+s+s2}{\PYGZdq{}dti:21.T11148/d87a75c52c68b06e9a18\PYGZdq{}}\PYG{p}{,}
      \PYG{n+nt}{\PYGZdq{}AlternateIdentifiers\PYGZdq{}}\PYG{p}{:} \PYG{l+s+s2}{\PYGZdq{}dti:21.T11148/eb3c713572f681e6c4c3\PYGZdq{}}\PYG{p}{,}
      \PYG{n+nt}{\PYGZdq{}Bibcode\PYGZhy{}Identifier\PYGZdq{}}\PYG{p}{:} \PYG{l+s+s2}{\PYGZdq{}dti:21.T11148/6c2fc7682e48ac7160b5\PYGZdq{}}\PYG{p}{,}
      \PYG{n+nt}{\PYGZdq{}DOI\PYGZhy{}Identifier\PYGZhy{}General\PYGZdq{}}\PYG{p}{:} \PYG{l+s+s2}{\PYGZdq{}dti:21.T11148/d93427e3c56173e9dc08\PYGZdq{}}\PYG{p}{,}
      \PYG{n+nt}{\PYGZdq{}Date\PYGZdq{}}\PYG{p}{:} \PYG{l+s+s2}{\PYGZdq{}dti:21.T11148/eb9a4bc1c0c153e4e4b0\PYGZdq{}}\PYG{p}{,}
      \PYG{n+nt}{\PYGZdq{}Dates\PYGZdq{}}\PYG{p}{:} \PYG{l+s+s2}{\PYGZdq{}dti:21.T11148/22c62082a4d2d9ae2602\PYGZdq{}}\PYG{p}{,}
      \PYG{n+nt}{\PYGZdq{}Description\PYGZdq{}}\PYG{p}{:} \PYG{l+s+s2}{\PYGZdq{}dti:21.T11148/f1627ce85386d8d75078\PYGZdq{}}\PYG{p}{,}
      \PYG{n+nt}{\PYGZdq{}Handle\PYGZhy{}Identifier\PYGZhy{}ASCII\PYGZdq{}}\PYG{p}{:} \PYG{l+s+s2}{\PYGZdq{}dti:21.T11148/3626040cadcac1571685\PYGZdq{}}\PYG{p}{,}
      \PYG{n+nt}{\PYGZdq{}ISAN\PYGZhy{}Identifier\PYGZdq{}}\PYG{p}{:} \PYG{l+s+s2}{\PYGZdq{}dti:21.T11148/48cfce4482166a103c50\PYGZdq{}}\PYG{p}{,}
      \PYG{n+nt}{\PYGZdq{}ISBN\PYGZhy{}Identifier\PYGZdq{}}\PYG{p}{:} \PYG{l+s+s2}{\PYGZdq{}dti:21.T11148/2ff8ad6cdd4e46622944\PYGZdq{}}\PYG{p}{,}
      \PYG{n+nt}{\PYGZdq{}ISNI\PYGZhy{}Identifier\PYGZdq{}}\PYG{p}{:} \PYG{l+s+s2}{\PYGZdq{}dti:21.T11148/cff32964e132c14fc56f\PYGZdq{}}\PYG{p}{,}
      \PYG{n+nt}{\PYGZdq{}ISRC\PYGZhy{}Identifier\PYGZdq{}}\PYG{p}{:} \PYG{l+s+s2}{\PYGZdq{}dti:21.T11148/2719170925ff2bfb5157\PYGZdq{}}\PYG{p}{,}
      \PYG{n+nt}{\PYGZdq{}ISSN\PYGZhy{}Identifier\PYGZdq{}}\PYG{p}{:} \PYG{l+s+s2}{\PYGZdq{}dti:21.T11148/7e689432354610f388c0\PYGZdq{}}\PYG{p}{,}
      \PYG{n+nt}{\PYGZdq{}ISTC\PYGZhy{}Identifier\PYGZdq{}}\PYG{p}{:} \PYG{l+s+s2}{\PYGZdq{}dti:21.T11148/1f0df9ef66774b2e2aa1\PYGZdq{}}\PYG{p}{,}
      \PYG{n+nt}{\PYGZdq{}ISWC\PYGZhy{}Identifier\PYGZdq{}}\PYG{p}{:} \PYG{l+s+s2}{\PYGZdq{}dti:21.T11148/698fba7e1c659fcfdcdd\PYGZdq{}}\PYG{p}{,}
      \PYG{n+nt}{\PYGZdq{}InstrumentType\PYGZdq{}}\PYG{p}{:} \PYG{l+s+s2}{\PYGZdq{}dti:21.T11148/f76ad9d0324302fc47dd\PYGZdq{}}\PYG{p}{,}
      \PYG{n+nt}{\PYGZdq{}InstrumentTypes\PYGZdq{}}\PYG{p}{:} \PYG{l+s+s2}{\PYGZdq{}dti:21.T11148/c60c8da7fff2ef4f98ce\PYGZdq{}}\PYG{p}{,}
      \PYG{n+nt}{\PYGZdq{}LandingPage\PYGZdq{}}\PYG{p}{:} \PYG{l+s+s2}{\PYGZdq{}dti:21.T11148/e0efc41346cda4ba84ca\PYGZdq{}}\PYG{p}{,}
      \PYG{n+nt}{\PYGZdq{}Manufacturer\PYGZdq{}}\PYG{p}{:} \PYG{l+s+s2}{\PYGZdq{}dti:21.T11148/7adfcd13b3b01de0d875\PYGZdq{}}\PYG{p}{,}
      \PYG{n+nt}{\PYGZdq{}Manufacturers\PYGZdq{}}\PYG{p}{:} \PYG{l+s+s2}{\PYGZdq{}dti:21.T11148/1f3e82ddf0697a497432\PYGZdq{}}\PYG{p}{,}
      \PYG{n+nt}{\PYGZdq{}MeasuredVariable\PYGZdq{}}\PYG{p}{:} \PYG{l+s+s2}{\PYGZdq{}dti:21.T11148/f1627ce85386d8d75078\PYGZdq{}}\PYG{p}{,}
      \PYG{n+nt}{\PYGZdq{}MeasuredVariables\PYGZdq{}}\PYG{p}{:} \PYG{l+s+s2}{\PYGZdq{}dti:21.T11148/72928b84e060d491ee41\PYGZdq{}}\PYG{p}{,}
      \PYG{n+nt}{\PYGZdq{}Model\PYGZdq{}}\PYG{p}{:} \PYG{l+s+s2}{\PYGZdq{}dti:21.T11148/c1a0ec5ad347427f25d6\PYGZdq{}}\PYG{p}{,}
      \PYG{n+nt}{\PYGZdq{}Name\PYGZdq{}}\PYG{p}{:} \PYG{l+s+s2}{\PYGZdq{}dti:21.T11148/ab8d232261b9b60ba559\PYGZdq{}}\PYG{p}{,}
      \PYG{n+nt}{\PYGZdq{}Owner\PYGZdq{}}\PYG{p}{:} \PYG{l+s+s2}{\PYGZdq{}dti:21.T11148/89ff31225c5f042fff61\PYGZdq{}}\PYG{p}{,}
      \PYG{n+nt}{\PYGZdq{}Owners\PYGZdq{}}\PYG{p}{:} \PYG{l+s+s2}{\PYGZdq{}dti:21.T11148/4eaec4bc0f1df68ab2a7\PYGZdq{}}\PYG{p}{,}
      \PYG{n+nt}{\PYGZdq{}PMCID\PYGZhy{}Identifier\PYGZdq{}}\PYG{p}{:} \PYG{l+s+s2}{\PYGZdq{}dti:21.T11148/e94bec7d7f1c63dd00cd\PYGZdq{}}\PYG{p}{,}
      \PYG{n+nt}{\PYGZdq{}PMID\PYGZhy{}Identifier\PYGZdq{}}\PYG{p}{:} \PYG{l+s+s2}{\PYGZdq{}dti:21.T11148/234c084bac48480bfe5d\PYGZdq{}}\PYG{p}{,}
      \PYG{n+nt}{\PYGZdq{}RelatedIdentifier\PYGZdq{}}\PYG{p}{:} \PYG{l+s+s2}{\PYGZdq{}dti:21.T11148/ec9f00af0761a065dbd0\PYGZdq{}}\PYG{p}{,}
      \PYG{n+nt}{\PYGZdq{}RelatedIdentifiers\PYGZdq{}}\PYG{p}{:} \PYG{l+s+s2}{\PYGZdq{}dti:21.T11148/178fb558abc755ca7046\PYGZdq{}}\PYG{p}{,}
      \PYG{n+nt}{\PYGZdq{}SchemaVersion\PYGZdq{}}\PYG{p}{:} \PYG{l+s+s2}{\PYGZdq{}dti:21.T11148/aa24da8ba845c23ea75c\PYGZdq{}}\PYG{p}{,}
      \PYG{n+nt}{\PYGZdq{}URN\PYGZhy{}Identifier\PYGZdq{}}\PYG{p}{:} \PYG{l+s+s2}{\PYGZdq{}dti:21.T11148/d22b6854df3503df7831\PYGZdq{}}\PYG{p}{,}
      \PYG{n+nt}{\PYGZdq{}alternateIdentifierName\PYGZdq{}}\PYG{p}{:} \PYG{l+s+s2}{\PYGZdq{}dti:21.T11148/f1627ce85386d8d75078\PYGZdq{}}\PYG{p}{,}
      \PYG{n+nt}{\PYGZdq{}alternateIdentifierType\PYGZdq{}}\PYG{p}{:} \PYG{l+s+s2}{\PYGZdq{}dti:21.T11148/015dc79a77940fb65aa4\PYGZdq{}}\PYG{p}{,}
      \PYG{n+nt}{\PYGZdq{}alternateIdentifierValue\PYGZdq{}}\PYG{p}{:} \PYG{l+s+s2}{\PYGZdq{}dti:21.T11148/f1627ce85386d8d75078\PYGZdq{}}\PYG{p}{,}
      \PYG{n+nt}{\PYGZdq{}arXiv\PYGZhy{}Identifier\PYGZdq{}}\PYG{p}{:} \PYG{l+s+s2}{\PYGZdq{}dti:21.T11148/d66f8368c3d305941a2e\PYGZdq{}}\PYG{p}{,}
      \PYG{n+nt}{\PYGZdq{}date\PYGZdq{}}\PYG{p}{:} \PYG{l+s+s2}{\PYGZdq{}dti:21.T11148/eb9a4bc1c0c153e4e4b0\PYGZdq{}}\PYG{p}{,}
      \PYG{n+nt}{\PYGZdq{}dateType\PYGZdq{}}\PYG{p}{:} \PYG{l+s+s2}{\PYGZdq{}dti:21.T11148/2f0e608b621a5a97e0d9\PYGZdq{}}\PYG{p}{,}
      \PYG{n+nt}{\PYGZdq{}dateValue\PYGZdq{}}\PYG{p}{:} \PYG{l+s+s2}{\PYGZdq{}dti:21.T11148/be707495360a234ef049\PYGZdq{}}\PYG{p}{,}
      \PYG{n+nt}{\PYGZdq{}dti\PYGZdq{}}\PYG{p}{:} \PYG{l+s+s2}{\PYGZdq{}http://hdl.handle.net/\PYGZdq{}}\PYG{p}{,}
      \PYG{n+nt}{\PYGZdq{}identifier\PYGZhy{}general\PYGZhy{}with\PYGZhy{}type\PYGZdq{}}\PYG{p}{:} \PYG{l+s+s2}{\PYGZdq{}dti:21.T11148/8eb858ee0b12e8e463a5\PYGZdq{}}\PYG{p}{,}
      \PYG{n+nt}{\PYGZdq{}identifierType\PYGZdq{}}\PYG{p}{:} \PYG{l+s+s2}{\PYGZdq{}dti:21.T11148/015dc79a77940fb65aa4\PYGZdq{}}\PYG{p}{,}
      \PYG{n+nt}{\PYGZdq{}identifierValue\PYGZdq{}}\PYG{p}{:} \PYG{l+s+s2}{\PYGZdq{}dti:21.T11148/f1627ce85386d8d75078\PYGZdq{}}\PYG{p}{,}
      \PYG{n+nt}{\PYGZdq{}instrumentTypeIdentifier\PYGZdq{}}\PYG{p}{:} \PYG{l+s+s2}{\PYGZdq{}dti:21.T11148/f9bdfd1810b999e3b11e\PYGZdq{}}\PYG{p}{,}
      \PYG{n+nt}{\PYGZdq{}instrumentTypeIdentifierType\PYGZdq{}}\PYG{p}{:} \PYG{l+s+s2}{\PYGZdq{}dti:21.T11148/015dc79a77940fb65aa4\PYGZdq{}}\PYG{p}{,}
      \PYG{n+nt}{\PYGZdq{}instrumentTypeIdentifierValue\PYGZdq{}}\PYG{p}{:} \PYG{l+s+s2}{\PYGZdq{}dti:21.T11148/f1627ce85386d8d75078\PYGZdq{}}\PYG{p}{,}
      \PYG{n+nt}{\PYGZdq{}instrumentTypeName\PYGZdq{}}\PYG{p}{:} \PYG{l+s+s2}{\PYGZdq{}dti:21.T11148/f1627ce85386d8d75078\PYGZdq{}}\PYG{p}{,}
      \PYG{n+nt}{\PYGZdq{}manufacturerIdentifier\PYGZdq{}}\PYG{p}{:} \PYG{l+s+s2}{\PYGZdq{}dti:21.T11148/5b240e16ea32ea25cf65\PYGZdq{}}\PYG{p}{,}
      \PYG{n+nt}{\PYGZdq{}manufacturerIdentifierType\PYGZdq{}}\PYG{p}{:} \PYG{l+s+s2}{\PYGZdq{}dti:21.T11148/015dc79a77940fb65aa4\PYGZdq{}}\PYG{p}{,}
      \PYG{n+nt}{\PYGZdq{}manufacturerIdentifierValue\PYGZdq{}}\PYG{p}{:} \PYG{l+s+s2}{\PYGZdq{}dti:21.T11148/f1627ce85386d8d75078\PYGZdq{}}\PYG{p}{,}
      \PYG{n+nt}{\PYGZdq{}manufacturerName\PYGZdq{}}\PYG{p}{:} \PYG{l+s+s2}{\PYGZdq{}dti:21.T11148/798588c5a1ec532f737b\PYGZdq{}}\PYG{p}{,}
      \PYG{n+nt}{\PYGZdq{}modelIdentifier\PYGZdq{}}\PYG{p}{:} \PYG{l+s+s2}{\PYGZdq{}dti:21.T11148/7e86a0b84960d0992fdf\PYGZdq{}}\PYG{p}{,}
      \PYG{n+nt}{\PYGZdq{}modelIdentifierType\PYGZdq{}}\PYG{p}{:} \PYG{l+s+s2}{\PYGZdq{}dti:21.T11148/015dc79a77940fb65aa4\PYGZdq{}}\PYG{p}{,}
      \PYG{n+nt}{\PYGZdq{}modelIdentifierValue\PYGZdq{}}\PYG{p}{:} \PYG{l+s+s2}{\PYGZdq{}dti:21.T11148/f1627ce85386d8d75078\PYGZdq{}}\PYG{p}{,}
      \PYG{n+nt}{\PYGZdq{}modelName\PYGZdq{}}\PYG{p}{:} \PYG{l+s+s2}{\PYGZdq{}dti:21.T11148/f1627ce85386d8d75078\PYGZdq{}}\PYG{p}{,}
      \PYG{n+nt}{\PYGZdq{}ownerContact\PYGZdq{}}\PYG{p}{:} \PYG{l+s+s2}{\PYGZdq{}dti:21.T11148/a88b7dcd1a9e3e17770b\PYGZdq{}}\PYG{p}{,}
      \PYG{n+nt}{\PYGZdq{}ownerIdentifier\PYGZdq{}}\PYG{p}{:} \PYG{l+s+s2}{\PYGZdq{}dti:21.T11148/1e3c17ac2a3e7ebf466a\PYGZdq{}}\PYG{p}{,}
      \PYG{n+nt}{\PYGZdq{}ownerIdentifierType\PYGZdq{}}\PYG{p}{:} \PYG{l+s+s2}{\PYGZdq{}dti:21.T11148/f1627ce85386d8d75078\PYGZdq{}}\PYG{p}{,}
      \PYG{n+nt}{\PYGZdq{}ownerIdentifierValue\PYGZdq{}}\PYG{p}{:} \PYG{l+s+s2}{\PYGZdq{}dti:221.T11148/38330bcc6a40ca85e5b4\PYGZdq{}}\PYG{p}{,}
      \PYG{n+nt}{\PYGZdq{}ownerName\PYGZdq{}}\PYG{p}{:} \PYG{l+s+s2}{\PYGZdq{}dti:21.T11148/798588c5a1ec532f737b\PYGZdq{}}\PYG{p}{,}
      \PYG{n+nt}{\PYGZdq{}relatedIdentifierName\PYGZdq{}}\PYG{p}{:} \PYG{l+s+s2}{\PYGZdq{}dti:21.T11148/f1627ce85386d8d75078\PYGZdq{}}\PYG{p}{,}
      \PYG{n+nt}{\PYGZdq{}relatedIdentifierType\PYGZdq{}}\PYG{p}{:} \PYG{l+s+s2}{\PYGZdq{}dti:21.T11148/015dc79a77940fb65aa4\PYGZdq{}}\PYG{p}{,}
      \PYG{n+nt}{\PYGZdq{}relatedIdentifierValue\PYGZdq{}}\PYG{p}{:} \PYG{l+s+s2}{\PYGZdq{}dti:21.T11148/f1627ce85386d8d75078\PYGZdq{}}\PYG{p}{,}
      \PYG{n+nt}{\PYGZdq{}relationType\PYGZdq{}}\PYG{p}{:} \PYG{l+s+s2}{\PYGZdq{}dti:21.T11148/292a53bd9ee27a242082\PYGZdq{}}
    \PYG{p}{\PYGZcb{}}\PYG{p}{,}
    \PYG{n+nt}{\PYGZdq{}@id\PYGZdq{}}\PYG{p}{:} \PYG{l+s+s2}{\PYGZdq{}dti:21.T11998/0000\PYGZhy{}001A\PYGZhy{}3905\PYGZhy{}F\PYGZdq{}}\PYG{p}{,}
    \PYG{n+nt}{\PYGZdq{}AlternateIdentifiers\PYGZdq{}}\PYG{p}{:} \PYG{p}{[}
      \PYG{p}{\PYGZob{}}
        \PYG{n+nt}{\PYGZdq{}AlternateIdentifier\PYGZdq{}}\PYG{p}{:} \PYG{p}{\PYGZob{}}
          \PYG{n+nt}{\PYGZdq{}alternateIdentifierValue\PYGZdq{}}\PYG{p}{:} \PYG{l+s+s2}{\PYGZdq{}2490\PYGZdq{}}\PYG{p}{,}
          \PYG{n+nt}{\PYGZdq{}alternateIdentifierType\PYGZdq{}}\PYG{p}{:} \PYG{l+s+s2}{\PYGZdq{}serialNumber\PYGZdq{}}
        \PYG{p}{\PYGZcb{}}
      \PYG{p}{\PYGZcb{}}
    \PYG{p}{]}\PYG{p}{,}
    \PYG{n+nt}{\PYGZdq{}Dates\PYGZdq{}}\PYG{p}{:} \PYG{p}{[}
      \PYG{p}{\PYGZob{}}
        \PYG{n+nt}{\PYGZdq{}date\PYGZdq{}}\PYG{p}{:} \PYG{p}{\PYGZob{}}
          \PYG{n+nt}{\PYGZdq{}date\PYGZdq{}}\PYG{p}{:} \PYG{l+s+s2}{\PYGZdq{}1999\PYGZhy{}11\PYGZhy{}01\PYGZdq{}}\PYG{p}{,}
          \PYG{n+nt}{\PYGZdq{}dateType\PYGZdq{}}\PYG{p}{:} \PYG{l+s+s2}{\PYGZdq{}Commissioned\PYGZdq{}}
        \PYG{p}{\PYGZcb{}}
      \PYG{p}{\PYGZcb{}}
    \PYG{p}{]}\PYG{p}{,}
    \PYG{n+nt}{\PYGZdq{}Description\PYGZdq{}}\PYG{p}{:} \PYG{l+s+s2}{\PYGZdq{}A high accuracy conductivity and temperature recorder with an optional pressure sensor designed for deployment on moorings. The IM model has an inductive modem for real\PYGZhy{}time data transmission plus internal flash memory data storage.\PYGZdq{}}\PYG{p}{,}
    \PYG{n+nt}{\PYGZdq{}InstrumentTypes\PYGZdq{}}\PYG{p}{:} \PYG{p}{[}
      \PYG{p}{\PYGZob{}}
        \PYG{n+nt}{\PYGZdq{}InstrumentType\PYGZdq{}}\PYG{p}{:} \PYG{p}{\PYGZob{}}
          \PYG{n+nt}{\PYGZdq{}instrumentTypeName\PYGZdq{}}\PYG{p}{:} \PYG{l+s+s2}{\PYGZdq{}water temperature sensor\PYGZdq{}}\PYG{p}{,}
          \PYG{n+nt}{\PYGZdq{}instrumentTypeIdentifier\PYGZdq{}}\PYG{p}{:} \PYG{p}{\PYGZob{}}
            \PYG{n+nt}{\PYGZdq{}instrumentTypeIdentifierValue\PYGZdq{}}\PYG{p}{:} \PYG{l+s+s2}{\PYGZdq{}http://vocab.nerc.ac.uk/collection/L05/current/134/\PYGZdq{}}\PYG{p}{,}
            \PYG{n+nt}{\PYGZdq{}instrumentTypeIdentifierType\PYGZdq{}}\PYG{p}{:} \PYG{l+s+s2}{\PYGZdq{}URL\PYGZdq{}}
          \PYG{p}{\PYGZcb{}}
        \PYG{p}{\PYGZcb{}}
      \PYG{p}{\PYGZcb{}}\PYG{p}{,}
      \PYG{p}{\PYGZob{}}
        \PYG{n+nt}{\PYGZdq{}InstrumentType\PYGZdq{}}\PYG{p}{:} \PYG{p}{\PYGZob{}}
          \PYG{n+nt}{\PYGZdq{}instrumentTypeName\PYGZdq{}}\PYG{p}{:} \PYG{l+s+s2}{\PYGZdq{}salinity sensor\PYGZdq{}}\PYG{p}{,}
          \PYG{n+nt}{\PYGZdq{}instrumentTypeIdentifier\PYGZdq{}}\PYG{p}{:} \PYG{p}{\PYGZob{}}
            \PYG{n+nt}{\PYGZdq{}instrumentTypeIdentifierValue\PYGZdq{}}\PYG{p}{:} \PYG{l+s+s2}{\PYGZdq{}http://vocab.nerc.ac.uk/collection/L05/current/350/\PYGZdq{}}\PYG{p}{,}
            \PYG{n+nt}{\PYGZdq{}instrumentTypeIdentifierType\PYGZdq{}}\PYG{p}{:} \PYG{l+s+s2}{\PYGZdq{}URL\PYGZdq{}}
          \PYG{p}{\PYGZcb{}}
        \PYG{p}{\PYGZcb{}}
      \PYG{p}{\PYGZcb{}}
    \PYG{p}{]}\PYG{p}{,}
    \PYG{n+nt}{\PYGZdq{}LandingPage\PYGZdq{}}\PYG{p}{:} \PYG{l+s+s2}{\PYGZdq{}https://linkedsystems.uk/system/instance/TOOL0022\PYGZus{}2490/current/\PYGZdq{}}\PYG{p}{,}
    \PYG{n+nt}{\PYGZdq{}Manufacturers\PYGZdq{}}\PYG{p}{:} \PYG{p}{[}
      \PYG{p}{\PYGZob{}}
        \PYG{n+nt}{\PYGZdq{}Manufacturer\PYGZdq{}}\PYG{p}{:} \PYG{p}{\PYGZob{}}
          \PYG{n+nt}{\PYGZdq{}manufacturerIdentifier\PYGZdq{}}\PYG{p}{:} \PYG{p}{\PYGZob{}}
            \PYG{n+nt}{\PYGZdq{}manufacturerIdentifierType\PYGZdq{}}\PYG{p}{:} \PYG{l+s+s2}{\PYGZdq{}URL\PYGZdq{}}\PYG{p}{,}
            \PYG{n+nt}{\PYGZdq{}manufacturerIdentifierValue\PYGZdq{}}\PYG{p}{:} \PYG{l+s+s2}{\PYGZdq{}http://vocab.nerc.ac.uk/collection/L35/current/MAN0013/\PYGZdq{}}
          \PYG{p}{\PYGZcb{}}\PYG{p}{,}
          \PYG{n+nt}{\PYGZdq{}manufacturerName\PYGZdq{}}\PYG{p}{:} \PYG{l+s+s2}{\PYGZdq{}Sea\PYGZhy{}Bird Scientific\PYGZdq{}}
        \PYG{p}{\PYGZcb{}}
      \PYG{p}{\PYGZcb{}}
    \PYG{p}{]}\PYG{p}{,}
    \PYG{n+nt}{\PYGZdq{}MeasuredVariables\PYGZdq{}}\PYG{p}{:} \PYG{p}{[}
      \PYG{p}{\PYGZob{}}
        \PYG{n+nt}{\PYGZdq{}MeasuredVariable\PYGZdq{}}\PYG{p}{:} \PYG{l+s+s2}{\PYGZdq{}http://vocab.nerc.ac.uk/collection/P01/current/CNDCPR01/\PYGZdq{}}
      \PYG{p}{\PYGZcb{}}\PYG{p}{,}
      \PYG{p}{\PYGZob{}}
        \PYG{n+nt}{\PYGZdq{}MeasuredVariable\PYGZdq{}}\PYG{p}{:} \PYG{l+s+s2}{\PYGZdq{}http://vocab.nerc.ac.uk/collection/P01/current/PSALPR01/\PYGZdq{}}
      \PYG{p}{\PYGZcb{}}\PYG{p}{,}
      \PYG{p}{\PYGZob{}}
        \PYG{n+nt}{\PYGZdq{}MeasuredVariable\PYGZdq{}}\PYG{p}{:} \PYG{l+s+s2}{\PYGZdq{}http://vocab.nerc.ac.uk/collection/P01/current/TEMPPR01/\PYGZdq{}}
      \PYG{p}{\PYGZcb{}}\PYG{p}{,}
      \PYG{p}{\PYGZob{}}
        \PYG{n+nt}{\PYGZdq{}MeasuredVariable\PYGZdq{}}\PYG{p}{:} \PYG{l+s+s2}{\PYGZdq{}http://vocab.nerc.ac.uk/collection/P01/current/PREXMCAT/\PYGZdq{}}
      \PYG{p}{\PYGZcb{}}
    \PYG{p}{]}\PYG{p}{,}
    \PYG{n+nt}{\PYGZdq{}Model\PYGZdq{}}\PYG{p}{:} \PYG{p}{[}
      \PYG{p}{\PYGZob{}}
        \PYG{n+nt}{\PYGZdq{}modelName\PYGZdq{}}\PYG{p}{:} \PYG{l+s+s2}{\PYGZdq{}Sea\PYGZhy{}Bird SBE 37 MicroCat IM\PYGZhy{}CT with optional pressure (submersible) CTD sensor series\PYGZdq{}}\PYG{p}{,}
        \PYG{n+nt}{\PYGZdq{}modelIdentifier\PYGZdq{}}\PYG{p}{:} \PYG{p}{\PYGZob{}}
          \PYG{n+nt}{\PYGZdq{}modelIdentifierValue\PYGZdq{}}\PYG{p}{:} \PYG{l+s+s2}{\PYGZdq{}http://vocab.nerc.ac.uk/collection/L22/current/TOOL0022/\PYGZdq{}}\PYG{p}{,}
          \PYG{n+nt}{\PYGZdq{}modelIdentifierType\PYGZdq{}}\PYG{p}{:} \PYG{l+s+s2}{\PYGZdq{}URL\PYGZdq{}}
        \PYG{p}{\PYGZcb{}}
      \PYG{p}{\PYGZcb{}}
    \PYG{p}{]}\PYG{p}{,}
    \PYG{n+nt}{\PYGZdq{}Name\PYGZdq{}}\PYG{p}{:} \PYG{l+s+s2}{\PYGZdq{}Sea\PYGZhy{}Bird SBE 37\PYGZhy{}IM MicroCAT C\PYGZhy{}T Sensor\PYGZdq{}}\PYG{p}{,}
    \PYG{n+nt}{\PYGZdq{}Owners\PYGZdq{}}\PYG{p}{:} \PYG{p}{[}
      \PYG{p}{\PYGZob{}}
        \PYG{n+nt}{\PYGZdq{}Owner\PYGZdq{}}\PYG{p}{:} \PYG{p}{\PYGZob{}}
          \PYG{n+nt}{\PYGZdq{}ownerContact\PYGZdq{}}\PYG{p}{:} \PYG{l+s+s2}{\PYGZdq{}someone@example.org\PYGZdq{}}\PYG{p}{,}
          \PYG{n+nt}{\PYGZdq{}ownerIdentifier\PYGZdq{}}\PYG{p}{:} \PYG{p}{\PYGZob{}}
            \PYG{n+nt}{\PYGZdq{}ownerIdentifierType\PYGZdq{}}\PYG{p}{:} \PYG{l+s+s2}{\PYGZdq{}URL\PYGZdq{}}\PYG{p}{,}
            \PYG{n+nt}{\PYGZdq{}ownerIdentifierValue\PYGZdq{}}\PYG{p}{:} \PYG{l+s+s2}{\PYGZdq{}http://vocab.nerc.ac.uk/collection/B75/current/ORG00009/\PYGZdq{}}
          \PYG{p}{\PYGZcb{}}\PYG{p}{,}
          \PYG{n+nt}{\PYGZdq{}ownerName\PYGZdq{}}\PYG{p}{:} \PYG{l+s+s2}{\PYGZdq{}National Oceanography Centre\PYGZdq{}}
        \PYG{p}{\PYGZcb{}}
      \PYG{p}{\PYGZcb{}}
    \PYG{p}{]}\PYG{p}{,}
    \PYG{n+nt}{\PYGZdq{}RelatedIdentifiers\PYGZdq{}}\PYG{p}{:} \PYG{p}{[}
      \PYG{p}{\PYGZob{}}
        \PYG{n+nt}{\PYGZdq{}RelatedIdentifier\PYGZdq{}}\PYG{p}{:} \PYG{p}{\PYGZob{}}
          \PYG{n+nt}{\PYGZdq{}relatedIdentifierType\PYGZdq{}}\PYG{p}{:} \PYG{l+s+s2}{\PYGZdq{}URL\PYGZdq{}}\PYG{p}{,}
          \PYG{n+nt}{\PYGZdq{}relatedIdentifierValue\PYGZdq{}}\PYG{p}{:} \PYG{l+s+s2}{\PYGZdq{}https://www.bodc.ac.uk/data/documents/nodb/pdf/37imbrochurejul08.pdf\PYGZdq{}}\PYG{p}{,}
          \PYG{n+nt}{\PYGZdq{}relationType\PYGZdq{}}\PYG{p}{:} \PYG{l+s+s2}{\PYGZdq{}IsDescribedBy \PYGZdq{}}
        \PYG{p}{\PYGZcb{}}
      \PYG{p}{\PYGZcb{}}
    \PYG{p}{]}\PYG{p}{,}
    \PYG{n+nt}{\PYGZdq{}SchemaVersion\PYGZdq{}}\PYG{p}{:} \PYG{l+m+mf}{1.0}\PYG{p}{,}
    \PYG{n+nt}{\PYGZdq{}identifier\PYGZhy{}general\PYGZhy{}with\PYGZhy{}type\PYGZdq{}}\PYG{p}{:} \PYG{p}{\PYGZob{}}
      \PYG{n+nt}{\PYGZdq{}identiferType\PYGZdq{}}\PYG{p}{:} \PYG{l+s+s2}{\PYGZdq{}Handle\PYGZdq{}}\PYG{p}{,}
      \PYG{n+nt}{\PYGZdq{}identifierValue\PYGZdq{}}\PYG{p}{:} \PYG{l+s+s2}{\PYGZdq{}http://hdl.handle.net/21.T11998/0000\PYGZhy{}001A\PYGZhy{}3905\PYGZhy{}F\PYGZdq{}}
    \PYG{p}{\PYGZcb{}}
  \PYG{p}{\PYGZcb{}}
\end{sphinxVerbatim}

\sphinxAtStartPar
As one can see in this result the context is over complete in the sense
that all possible sub types are resolved and referred in @context, but
not all of them are actually used by the types occuring in the PID. This
could be pruned by an additional step of the algorithm to a version
reduced to the necessary and sufficient sub types. Such a pruning is
also automatically done by LD converters%
\begin{footnote}[1]\sphinxAtStartFootnote
as for instance: \sphinxurl{http://www.easyrdf.org/converter}
%
\end{footnote} as one can
see in the following snippet with a conversion into Turtle Terse RDF
that results into the following serialization
(\hyperref[\detokenize{white-paper/landing-page-encoding:snip-landing-encoding-turtle}]{Snippet \ref{\detokenize{white-paper/landing-page-encoding:snip-landing-encoding-turtle}}}), where only the values remain
and the names used in the type definitions are replaced by their type
PID suffixes:
\sphinxSetupCaptionForVerbatim{representation in Turtle Terse RDF of the NERC example
of \hyperref[\detokenize{white-paper/metadata-schema-recommendations:tab-schema-handle-record}]{Table \ref{\detokenize{white-paper/metadata-schema-recommendations:tab-schema-handle-record}}} that was generated
by a JSON\sphinxhyphen{}LD to RDF converter from the JSON\sphinxhyphen{}LD in
\hyperref[\detokenize{white-paper/landing-page-encoding:snip-landing-encoding-json-ld}]{Snippet \ref{\detokenize{white-paper/landing-page-encoding:snip-landing-encoding-json-ld}}}.}
\def\sphinxLiteralBlockLabel{\label{\detokenize{white-paper/landing-page-encoding:snip-landing-encoding-turtle}}}
\begin{sphinxVerbatim}[commandchars=\\\{\}]
  @prefix ns0: \PYGZlt{}http://hdl.handle.net/21.T11148/\PYGZgt{} .
  @prefix xsd: \PYGZlt{}http://www.w3.org/2001/XMLSchema\PYGZsh{}\PYGZgt{} .
  @prefix ns1: \PYGZlt{}http://hdl.handle.net/221.T11148/\PYGZgt{} .

  \PYGZlt{}http://hdl.handle.net/21.T11998/0000\PYGZhy{}001A\PYGZhy{}3905\PYGZhy{}F\PYGZgt{}
    ns0:178fb558abc755ca7046 [ ns0:ec9f00af0761a065dbd0 [
        ns0:015dc79a77940fb65aa4 \PYGZdq{}URL\PYGZdq{}\PYGZca{}\PYGZca{}xsd:string ;
        ns0:292a53bd9ee27a242082 \PYGZdq{}IsDescribedBy \PYGZdq{}\PYGZca{}\PYGZca{}xsd:string ;
        ns0:f1627ce85386d8d75078 \PYGZdq{}https://www.bodc.ac.uk/data/documents/nodb/pdf/37imbrochurejul08.pdf\PYGZdq{}\PYGZca{}\PYGZca{}xsd:string
      ] ] ;
    ns0:1f3e82ddf0697a497432 [ ns0:7adfcd13b3b01de0d875 [
        ns0:5b240e16ea32ea25cf65 [
          ns0:015dc79a77940fb65aa4 \PYGZdq{}URL\PYGZdq{}\PYGZca{}\PYGZca{}xsd:string ;
          ns0:f1627ce85386d8d75078 \PYGZdq{}http://vocab.nerc.ac.uk/collection/L35/current/MAN0013/\PYGZdq{}\PYGZca{}\PYGZca{}xsd:string
        ] ;
        ns0:798588c5a1ec532f737b \PYGZdq{}Sea\PYGZhy{}Bird Scientific\PYGZdq{}\PYGZca{}\PYGZca{}xsd:string
      ] ] ;
    ns0:22c62082a4d2d9ae2602 [ ns0:eb9a4bc1c0c153e4e4b0 [
        ns0:2f0e608b621a5a97e0d9 \PYGZdq{}Commissioned\PYGZdq{}\PYGZca{}\PYGZca{}xsd:string ;
        ns0:eb9a4bc1c0c153e4e4b0 \PYGZdq{}1999\PYGZhy{}11\PYGZhy{}01\PYGZdq{}\PYGZca{}\PYGZca{}xsd:string
      ] ] ;
    ns0:4eaec4bc0f1df68ab2a7 [ ns0:89ff31225c5f042fff61 [
        ns0:1e3c17ac2a3e7ebf466a [
          ns0:f1627ce85386d8d75078 \PYGZdq{}URL\PYGZdq{}\PYGZca{}\PYGZca{}xsd:string ;
          ns1:38330bcc6a40ca85e5b4 \PYGZdq{}http://vocab.nerc.ac.uk/collection/B75/current/ORG00009/\PYGZdq{}\PYGZca{}\PYGZca{}xsd:string
        ] ;
        ns0:798588c5a1ec532f737b \PYGZdq{}National Oceanography Centre\PYGZdq{}\PYGZca{}\PYGZca{}xsd:string ;
        ns0:a88b7dcd1a9e3e17770b \PYGZdq{}someone@example.org\PYGZdq{}\PYGZca{}\PYGZca{}xsd:string
      ] ] ;
    ns0:72928b84e060d491ee41 [ ns0:f1627ce85386d8d75078 \PYGZdq{}http://vocab.nerc.ac.uk/collection/P01/current/CNDCPR01/\PYGZdq{}\PYGZca{}\PYGZca{}xsd:string ], [ ns0:f1627ce85386d8d75078 \PYGZdq{}http://vocab.nerc.ac.uk/collection/P01/current/PSALPR01/\PYGZdq{}\PYGZca{}\PYGZca{}xsd:string ], [ ns0:f1627ce85386d8d75078 \PYGZdq{}http://vocab.nerc.ac.uk/collection/P01/current/TEMPPR01/\PYGZdq{}\PYGZca{}\PYGZca{}xsd:string ], [ ns0:f1627ce85386d8d75078 \PYGZdq{}http://vocab.nerc.ac.uk/collection/P01/current/PREXMCAT/\PYGZdq{}\PYGZca{}\PYGZca{}xsd:string ] ;
    ns0:8eb858ee0b12e8e463a5 [ ns0:f1627ce85386d8d75078 \PYGZdq{}http://hdl.handle.net/21.T11998/0000\PYGZhy{}001A\PYGZhy{}3905\PYGZhy{}F\PYGZdq{}\PYGZca{}\PYGZca{}xsd:string ] ;
    ns0:aa24da8ba845c23ea75c 1 ;
    ns0:ab8d232261b9b60ba559 \PYGZdq{}Sea\PYGZhy{}Bird SBE 37\PYGZhy{}IM MicroCAT C\PYGZhy{}T Sensor\PYGZdq{}\PYGZca{}\PYGZca{}xsd:string ;
    ns0:c1a0ec5ad347427f25d6 [
      ns0:7e86a0b84960d0992fdf [
        ns0:015dc79a77940fb65aa4 \PYGZdq{}URL\PYGZdq{}\PYGZca{}\PYGZca{}xsd:string ;
        ns0:f1627ce85386d8d75078 \PYGZdq{}http://vocab.nerc.ac.uk/collection/L22/current/TOOL0022/\PYGZdq{}\PYGZca{}\PYGZca{}xsd:string
      ] ;
      ns0:f1627ce85386d8d75078 \PYGZdq{}Sea\PYGZhy{}Bird SBE 37 MicroCat IM\PYGZhy{}CT with optional pressure (submersible) CTD sensor series\PYGZdq{}\PYGZca{}\PYGZca{}xsd:string
    ] ;
    ns0:c60c8da7fff2ef4f98ce [ ns0:f76ad9d0324302fc47dd [
        ns0:f1627ce85386d8d75078 \PYGZdq{}water temperature sensor\PYGZdq{}\PYGZca{}\PYGZca{}xsd:string ;
        ns0:f9bdfd1810b999e3b11e [
          ns0:015dc79a77940fb65aa4 \PYGZdq{}URL\PYGZdq{}\PYGZca{}\PYGZca{}xsd:string ;
          ns0:f1627ce85386d8d75078 \PYGZdq{}http://vocab.nerc.ac.uk/collection/L05/current/134/\PYGZdq{}\PYGZca{}\PYGZca{}xsd:string
        ]
      ] ], [ ns0:f76ad9d0324302fc47dd [ ns0:f1627ce85386d8d75078 \PYGZdq{}salinity sensor\PYGZdq{}\PYGZca{}\PYGZca{}xsd:string ] ] ;
    ns0:e0efc41346cda4ba84ca \PYGZdq{}https://linkedsystems.uk/system/instance/TOOL0022\PYGZus{}2490/current/\PYGZdq{}\PYGZca{}\PYGZca{}xsd:string ;
    ns0:eb3c713572f681e6c4c3 [ ns0:d87a75c52c68b06e9a18 [
        ns0:015dc79a77940fb65aa4 \PYGZdq{}serialNumber\PYGZdq{}\PYGZca{}\PYGZca{}xsd:string ;
        ns0:f1627ce85386d8d75078 \PYGZdq{}2490\PYGZdq{}\PYGZca{}\PYGZca{}xsd:string
      ] ] ;
    ns0:f1627ce85386d8d75078 \PYGZdq{}A high accuracy conductivity and temperature recorder with an optional pressure sensor designed for deployment on moorings. The IM model has an inductive modem for real\PYGZhy{}time data transmission plus internal flash memory data storage.\PYGZdq{}\PYGZca{}\PYGZca{}xsd:string .
\end{sphinxVerbatim}


\paragraph{Sensor web enablement (SWE)}
\label{\detokenize{white-paper/landing-page-encoding:sensor-web-enablement-swe}}\label{\detokenize{white-paper/landing-page-encoding:landing-page-encoding-swe}}
\sphinxAtStartPar
Global standards have been developed which enable the web\sphinxhyphen{}based
discovery, exchange and processing of sensors and their observations.
Many developers using standards, such as the Open Geospatial
Consortium’s (OGC) Sensor Web Enablement (SWE), publish formal,
machine\sphinxhyphen{}readable descriptions of sensors and their technical information
as web resources using URLs, identifying the instrument locally.
Web\sphinxhyphen{}based sensor descriptions published using SensorML, part of the SWE
specifications, and may be used as a URL to the landing page of the
instrument registered at a PID provider. A SensorML landing page example
has been published at the British Oceanographic Data Centre (BODC) via
the ePIC PID provider service
(\sphinxurl{http://hdl.handle.net/21.T11998/0000-001A-3905-F}). To view the Handle
record directly see
\sphinxurl{http://hdl.handle.net/21.T11998/0000-001A-3905-F?noredirect} or
\hyperref[\detokenize{white-paper/metadata-schema-recommendations:tab-schema-handle-record}]{Table \ref{\detokenize{white-paper/metadata-schema-recommendations:tab-schema-handle-record}}} in this document.

\sphinxAtStartPar
In SensorML (version 2.0), sensors are identified using a unique ID
within the \sphinxstyleemphasis{gml:identifier} element and institutions may choose to use
an instrument PID to assure uniqueness. Alternatively, an instrument PID
may be declared as metadata within a SensorML description using the
\sphinxstyleemphasis{sml:identifier} property (\hyperref[\detokenize{white-paper/landing-page-encoding:snip-landing-encoding-sensorml}]{Snippet \ref{\detokenize{white-paper/landing-page-encoding:snip-landing-encoding-sensorml}}}).
While the latter is simpler to implement, it may limit the global
discoverability of sensors and their observations within the Sensor
Observation Service (SOS) web Application Programming Interface (API),
part of the SWE standard. Web\sphinxhyphen{}based enquiries, requests or
transactions made for sensors using this service are typically based
on \sphinxstyleemphasis{gml:identifier} element in SensorML (expressed as a \sphinxstyleemphasis{procedure}),
thus identifying sensors using local identifiers rather than global
instrument PIDs directly. The link between local identifiers and
instrument PIDs can be found indirectly using a combination of
\sphinxstyleemphasis{GetCapabilities} and \sphinxstyleemphasis{DescribeSensor} operational requests to a SOS
server.
\sphinxSetupCaptionForVerbatim{An example of expressing an instrument PID
(\sphinxurl{http://hdl.handle.net/21.T11998/0000-001A-3905-F}) as
identifying metadata within a SensorML technical
description using the \sphinxstyleemphasis{sml:identifier} property for a
SeaBird Scientific SBE 37 Conductivity, temperature and
depth sensor.}
\def\sphinxLiteralBlockLabel{\label{\detokenize{white-paper/landing-page-encoding:snip-landing-encoding-sensorml}}}
\begin{sphinxVerbatim}[commandchars=\\\{\}]
  \PYG{n+nt}{\PYGZlt{}sml:identifier}\PYG{n+nt}{\PYGZgt{}}
    \PYG{n+nt}{\PYGZlt{}sml:Term} \PYG{n+na}{definition=}\PYG{l+s}{\PYGZdq{}http://www.example.com/definitions/pidinst/\PYGZdq{}}\PYG{n+nt}{\PYGZgt{}}
       \PYG{n+nt}{\PYGZlt{}sml:label}\PYG{n+nt}{\PYGZgt{}}Instrument persistent identifier\PYG{n+nt}{\PYGZlt{}/sml:label\PYGZgt{}}
       \PYG{n+nt}{\PYGZlt{}sml:value}\PYG{n+nt}{\PYGZgt{}}http://hdl.handle.net/21.T11998/0000\PYGZhy{}001A\PYGZhy{}3905\PYGZhy{}F\PYG{n+nt}{\PYGZlt{}/sml:value\PYGZgt{}}
    \PYG{n+nt}{\PYGZlt{}/sml:Term\PYGZgt{}}
  \PYG{n+nt}{\PYGZlt{}/sml:identifier\PYGZgt{}}
\end{sphinxVerbatim}

\sphinxAtStartPar
The list of properties that can be expressed in SensorML to describe
sensors is not particularly restrictive and it is recommended that
institutional instrument providers follow the PIDINST guidance on
landing page content (see {\hyperref[\detokenize{white-paper/landing-page-content:landing-page-content}]{\sphinxcrossref{\DUrole{std,std-ref}{Landing page content}}}}).  Recently, the
\sphinxhref{https://github.com/ODIP/MarineProfilesForSWE/blob/master/docs/02\_SensorML.md}{Marine SWE Profiles} initiative specified a comprehensive metadata
profile to improve the semantic interoperability of SensorML in the
Earth Science marine domain by developing sets of sensor specific
terminologies.


\subsubsection{Content negotiation}
\label{\detokenize{white-paper/landing-page-encoding:content-negotiation}}
\sphinxAtStartPar
We recommend using content negotiation where instrument landing pages
are not easily consumed for human reading (such as XML schemas). PIDINST
does not specify the form of negotiation that produces human\sphinxhyphen{}readable
content from machine\sphinxhyphen{}readable representations. Other groups, such as the
W3C Dataset Exchange Working Group (DXWG) are currently drafting
recommendations on content negotiation from different information
models.%
\begin{footnote}[2]\sphinxAtStartFootnote
\sphinxurl{https://www.w3.org/TR/dx-prof-conneg/\#dfn-data-profile}
%
\end{footnote}


\subsection{Linking instrument PIDs to datasets}
\label{\detokenize{white-paper/linking-datasets:linking-instrument-pids-to-datasets}}\label{\detokenize{white-paper/linking-datasets::doc}}
\sphinxAtStartPar
One major purpose of PIDINST is to ease tracking the scientific output
of the instrument.  In order to benefit from this, it is important to
establish the relation between the datasets and the instrument being
used to collect the data in a machine readable way.


\subsubsection{DataCite metadata}
\label{\detokenize{white-paper/linking-datasets:datacite-metadata}}
\sphinxAtStartPar
Datasets are usually published with a DataCite DOI.  The \sphinxhref{https://schema.datacite.org/}{DataCite
Metadata Schema} allows to link the instrument from the metadata
registered with that DOI for a data publication using the
\sphinxstyleemphasis{RelatedIdentifier} and \sphinxstyleemphasis{RelatedItem} properties.  The recommended
\sphinxstyleemphasis{relationType} is \sphinxstyleemphasis{IsCollectedBy} in this case.  \hyperref[\detokenize{white-paper/linking-datasets:fig-link-hzb}]{Figure \ref{\detokenize{white-paper/linking-datasets:fig-link-hzb}}}
shows an example for a dataset published by HZB
(\sphinxurl{https://doi.org/10.5442/ND000001}).  The data has been collected using
neutron diffraction with the E2 \sphinxhyphen{} Flat\sphinxhyphen{}Cone Diffractometer beamline at
BER II.  The image show a screenshot of the data publication landing
page which links the PID of the instrument.
\hyperref[\detokenize{white-paper/linking-datasets:snip-link-dataset-datacite-xml-relidentifier}]{Snippet \ref{\detokenize{white-paper/linking-datasets:snip-link-dataset-datacite-xml-relidentifier}}} and
\hyperref[\detokenize{white-paper/linking-datasets:snip-link-dataset-datacite-xml-relitem}]{Snippet \ref{\detokenize{white-paper/linking-datasets:snip-link-dataset-datacite-xml-relitem}}} show sections from the
DOI metadata from the same data publication containing this link.

\begin{figure}[htbp]
\centering
\capstart

\noindent\sphinxincludegraphics{{ND000001-landing}.png}
\caption{Landing page of a dataset published by HZB which links the PID of
the instrument.}\label{\detokenize{white-paper/linking-datasets:fig-link-hzb}}\end{figure}
\sphinxSetupCaptionForVerbatim{Use of the RelatedIdentifier property in the DOI
metadata from a data publication.  The third entry links
the PID of the instrument.}
\def\sphinxLiteralBlockLabel{\label{\detokenize{white-paper/linking-datasets:snip-link-dataset-datacite-xml-relidentifier}}}
\begin{sphinxVerbatim}[commandchars=\\\{\}]
  \PYG{n+nt}{\PYGZlt{}relatedIdentifiers}\PYG{n+nt}{\PYGZgt{}}
    \PYG{n+nt}{\PYGZlt{}relatedIdentifier} \PYG{n+na}{relatedIdentifierType=}\PYG{l+s}{\PYGZdq{}DOI\PYGZdq{}} \PYG{n+na}{relationType=}\PYG{l+s}{\PYGZdq{}IsCitedBy\PYGZdq{}}\PYG{n+nt}{\PYGZgt{}}10.1103/physrevb.99.174111\PYG{n+nt}{\PYGZlt{}/relatedIdentifier\PYGZgt{}}
    \PYG{n+nt}{\PYGZlt{}relatedIdentifier} \PYG{n+na}{relatedIdentifierType=}\PYG{l+s}{\PYGZdq{}DOI\PYGZdq{}} \PYG{n+na}{relationType=}\PYG{l+s}{\PYGZdq{}References\PYGZdq{}}\PYG{n+nt}{\PYGZgt{}}10.17815/jlsrf\PYGZhy{}4\PYGZhy{}110\PYG{n+nt}{\PYGZlt{}/relatedIdentifier\PYGZgt{}}
    \PYG{n+nt}{\PYGZlt{}relatedIdentifier} \PYG{n+na}{relatedIdentifierType=}\PYG{l+s}{\PYGZdq{}DOI\PYGZdq{}} \PYG{n+na}{relationType=}\PYG{l+s}{\PYGZdq{}IsCollectedBy\PYGZdq{}}\PYG{n+nt}{\PYGZgt{}}10.5442/NI000001\PYG{n+nt}{\PYGZlt{}/relatedIdentifier\PYGZgt{}}
  \PYG{n+nt}{\PYGZlt{}/relatedIdentifiers\PYGZgt{}}
\end{sphinxVerbatim}
\sphinxSetupCaptionForVerbatim{Use of the RelatedItem property in the DOI metadata from
a data publication to link the PID of the instrument.}
\def\sphinxLiteralBlockLabel{\label{\detokenize{white-paper/linking-datasets:snip-link-dataset-datacite-xml-relitem}}}
\begin{sphinxVerbatim}[commandchars=\\\{\}]
  \PYG{n+nt}{\PYGZlt{}relatedItems}\PYG{n+nt}{\PYGZgt{}}
    \PYG{c}{\PYGZlt{}!\PYGZhy{}\PYGZhy{}}\PYG{c}{ ... }\PYG{c}{\PYGZhy{}\PYGZhy{}\PYGZgt{}}
    \PYG{n+nt}{\PYGZlt{}relatedItem} \PYG{n+na}{relatedItemType=}\PYG{l+s}{\PYGZdq{}Instrument\PYGZdq{}} \PYG{n+na}{relationType=}\PYG{l+s}{\PYGZdq{}IsCollectedBy\PYGZdq{}}\PYG{n+nt}{\PYGZgt{}}
      \PYG{n+nt}{\PYGZlt{}relatedItemIdentifier} \PYG{n+na}{relatedItemIdentifierType=}\PYG{l+s}{\PYGZdq{}DOI\PYGZdq{}}\PYG{n+nt}{\PYGZgt{}}10.5442/NI000001\PYG{n+nt}{\PYGZlt{}/relatedItemIdentifier\PYGZgt{}}
      \PYG{n+nt}{\PYGZlt{}titles}\PYG{n+nt}{\PYGZgt{}}
        \PYG{n+nt}{\PYGZlt{}title}\PYG{n+nt}{\PYGZgt{}}E2 \PYGZhy{} Flat\PYGZhy{}Cone Diffractometer\PYG{n+nt}{\PYGZlt{}/title\PYGZgt{}}
      \PYG{n+nt}{\PYGZlt{}/titles\PYGZgt{}}
    \PYG{n+nt}{\PYGZlt{}/relatedItem\PYGZgt{}}
  \PYG{n+nt}{\PYGZlt{}/relatedItems\PYGZgt{}}
\end{sphinxVerbatim}


\subsubsection{schema.org}
\label{\detokenize{white-paper/linking-datasets:schema-org}}
\sphinxAtStartPar
\hyperref[\detokenize{white-paper/linking-datasets:fig-link-pangea}]{Figure \ref{\detokenize{white-paper/linking-datasets:fig-link-pangea}}} shows an example of marine dataset
(\sphinxurl{https://doi.org/10.1594/PANGAEA.887579}) published through PANGAEA. The
metadata of the dataset includes descriptive information about the
dataset and its related entities (e.g., scholarly article, project). The
dataset was gathered through sensors attached to an autonomous
underwater vehicle (AWI AUV Polar Autonomous Underwater Laboratory),
which was deployed as part of a cruise campaign (MSM29). The vehicle is
identified through a persistent identifier assigned by
\sphinxurl{https://sensor.awi.de/}. The landing page of the instrument contains
metadata of the instrument such as description, manufacturer, model,
contact, calibration information. \hyperref[\detokenize{white-paper/linking-datasets:fig-link-model}]{Figure \ref{\detokenize{white-paper/linking-datasets:fig-link-model}}} depicts
schema.org types and properties that may be used to model the
dataset’s observation event (e.g., cruise campaign) and instrument
deployed (AUV). \hyperref[\detokenize{white-paper/linking-datasets:fig-link-schema-org}]{Figure \ref{\detokenize{white-paper/linking-datasets:fig-link-schema-org}}} shows the snippet of
actual schema.org representation. External vocabularies (NERC SeaVoX
Platform Categories and GeoLink Schema) are used to indicate the
additional type for Event and Vehicle. In Schema.org, ‘Event’ refers
to an occurrence at a specific time and location, for example a social
event. As such, new types and properties are required to support the
description of observation events and related scientific instruments
to ensure full compliance with Schema.org functionality.

\begin{figure}[htbp]
\centering
\capstart

\noindent\sphinxincludegraphics{{image2}.png}
\caption{An example of a dataset published by PANGAEA which includes its
instrument identifier
(\sphinxurl{https://doi.pangaea.de/10013/sensor.664525cf-45b9-4969-bb88-91a1c5e97a5b})}\label{\detokenize{white-paper/linking-datasets:fig-link-pangea}}\end{figure}

\begin{figure}[htbp]
\centering
\capstart

\noindent\sphinxincludegraphics{{image1}.png}
\caption{Conceptual model of Event and Specific Instrument Type (Vehicle)}\label{\detokenize{white-paper/linking-datasets:fig-link-model}}\end{figure}

\begin{figure}[htbp]
\centering
\capstart

\noindent\sphinxincludegraphics{{image3}.png}
\caption{Snippet of schema.org representation of event and instrument
associated with the dataset in \hyperref[\detokenize{white-paper/linking-datasets:fig-link-pangea}]{Figure \ref{\detokenize{white-paper/linking-datasets:fig-link-pangea}}}.}\label{\detokenize{white-paper/linking-datasets:fig-link-schema-org}}\end{figure}


\subsubsection{NetCDF4}
\label{\detokenize{white-paper/linking-datasets:netcdf4}}\label{\detokenize{white-paper/linking-datasets:section-1}}
\sphinxAtStartPar
State\sphinxhyphen{}of\sphinxhyphen{}the\sphinxhyphen{}art research ships are multimillion\sphinxhyphen{}pound floating
laboratories which operate diverse arrays of high\sphinxhyphen{}powered,
high\sphinxhyphen{}resolution sensors around\sphinxhyphen{}the\sphinxhyphen{}clock (e.g. sea\sphinxhyphen{}floor depth,
weather, ocean current velocity and hydrography etc.). The National
Oceanography Centre (NOC)%
\begin{footnote}[1]\sphinxAtStartFootnote
British Oceanographic Data Centre (BODC) and National Marine
Facilities (NMF) divisions
%
\end{footnote} and British Antarctic Survey
(BAS)%
\begin{footnote}[2]\sphinxAtStartFootnote
Uk Polar Data Centre division
%
\end{footnote} are currently working together to improve the
integrity of the data management workflow from these sensor systems to
end\sphinxhyphen{}users across the UK National Environment Research Council (NERC)
large research vessel fleet, as part of the initiative, I/Ocean. In
doing so, we can make cost effective use of vessel time while
improving the FAIRness,%
\begin{footnote}[3]\sphinxAtStartFootnote
Wilkinson, M., Dumontier, M., Aalbersberg, I. \sphinxstyleemphasis{et al.} The FAIR
Guiding Principles for scientific data management and stewardship.
\sphinxstyleemphasis{Sci Data} 3, 160018 (2016). \sphinxurl{https://doi.org/10.1038/sdata.2016.18}
%
\end{footnote} and in turn, access of data
from these sensor arrays. The initial phase of the solution
implements common NetCDF formats enabling harmonised access to data
for researchers across ships. The formats are based on NetCDF4 and
comply with Climate Forecast conventions. It has currently been
proposed that NetCDF4 groups could be used to identify instruments and
associated metadata in a similar way to the SONAR\sphinxhyphen{}netCDF4 convention
for sonar data%
\begin{footnote}[5]\sphinxAtStartFootnote
Macaulay, Gavin; Peña, Hector (2018). The SONAR\sphinxhyphen{}netCDF4 convention for
sonar data, Version 1.0. ICES Cooperative Research Reports (CRR).
Report. \sphinxurl{https://doi.org/10.17895/ices.pub.4392}
%
\end{footnote}. In doing so, the instrument PID is
implemented as the data of a geophysical variable within a group that
has an applicable date range (\hyperref[\detokenize{white-paper/linking-datasets:snip-link-netcdf-cdl}]{Snippet \ref{\detokenize{white-paper/linking-datasets:snip-link-netcdf-cdl}}}). For
example, when the sensor was installed. Data streams are then linked
to the instruments which produced them using the variable attribute
\sphinxstyleemphasis{instrument} from Attribute Convention for Data Discovery (ACDD) 1\sphinxhyphen{}3.
Through groups, other variables or attributes could hold more detailed
information relating to an instrument. Additionally, groups may
potentially offer a way to store other information with valid date
ranges, such as calibrations, instrument reference frames and
instrument orientations (e.g. the reference point of an anemometer).
\sphinxSetupCaptionForVerbatim{Truncated CF\sphinxhyphen{}NetCDF4 CDL file. Note some terminologies
are in development.}
\def\sphinxLiteralBlockLabel{\label{\detokenize{white-paper/linking-datasets:snip-link-netcdf-cdl}}}
\begin{sphinxVerbatim}[commandchars=\\\{\}]
  \PYG{n}{netcdf} \PYG{n}{iocean\PYGZus{}example} \PYG{p}{\PYGZob{}}
  \PYG{n}{dimensions}\PYG{p}{:}
     \PYG{n}{INSTANCE} \PYG{o}{=} \PYG{n}{UNLIMITED} \PYG{p}{;} \PYG{o}{/}\PYG{o}{/} \PYG{p}{(}\PYG{l+m+mi}{1} \PYG{n}{currently}\PYG{p}{)}
     \PYG{n}{MAXT} \PYG{o}{=} \PYG{l+m+mi}{6} \PYG{p}{;}
  \PYG{n}{variables}\PYG{p}{:}
     \PYG{n+nb}{float} \PYG{n}{seatemp}\PYG{p}{(}\PYG{n}{INSTANCE}\PYG{p}{,} \PYG{n}{MAXT}\PYG{p}{)} \PYG{p}{;}
        \PYG{n}{seatemp}\PYG{p}{:}\PYG{n}{\PYGZus{}FillValue} \PYG{o}{=} \PYG{o}{\PYGZhy{}}\PYG{l+m+mf}{9.}\PYG{n}{f} \PYG{p}{;}
        \PYG{n}{seatemp}\PYG{p}{:}\PYG{n}{long\PYGZus{}name} \PYG{o}{=} \PYG{l+s+s2}{\PYGZdq{}}\PYG{l+s+s2}{sea surface temperature}\PYG{l+s+s2}{\PYGZdq{}} \PYG{p}{;}
        \PYG{n}{seatemp}\PYG{p}{:}\PYG{n}{standard\PYGZus{}name} \PYG{o}{=} \PYG{l+s+s2}{\PYGZdq{}}\PYG{l+s+s2}{sea\PYGZus{}surface\PYGZus{}temperature}\PYG{l+s+s2}{\PYGZdq{}} \PYG{p}{;}
        \PYG{n}{seatemp}\PYG{p}{:}\PYG{n}{units} \PYG{o}{=} \PYG{l+s+s2}{\PYGZdq{}}\PYG{l+s+s2}{degC}\PYG{l+s+s2}{\PYGZdq{}} \PYG{p}{;}
        \PYG{n}{seatemp}\PYG{p}{:}\PYG{n}{sdn\PYGZus{}parameter\PYGZus{}urn} \PYG{o}{=} \PYG{l+s+s2}{\PYGZdq{}}\PYG{l+s+s2}{SDN:P01::TEMPHU01}\PYG{l+s+s2}{\PYGZdq{}} \PYG{p}{;}
        \PYG{n}{seatemp}\PYG{p}{:}\PYG{n}{sdn\PYGZus{}uom\PYGZus{}urn} \PYG{o}{=} \PYG{l+s+s2}{\PYGZdq{}}\PYG{l+s+s2}{SDN:P06::UPAA}\PYG{l+s+s2}{\PYGZdq{}} \PYG{p}{;}
        \PYG{n}{seatemp}\PYG{p}{:}\PYG{n}{sdn\PYGZus{}parameter\PYGZus{}name} \PYG{o}{=} \PYG{l+s+s2}{\PYGZdq{}}\PYG{l+s+s2}{Temperature of the water body by thermosalinograph hull sensor and NO verification against independent measurements}\PYG{l+s+s2}{\PYGZdq{}} \PYG{p}{;}
        \PYG{n}{seatemp}\PYG{p}{:}\PYG{n}{sdn\PYGZus{}uom\PYGZus{}name} \PYG{o}{=} \PYG{l+s+s2}{\PYGZdq{}}\PYG{l+s+s2}{Degrees Celsius}\PYG{l+s+s2}{\PYGZdq{}} \PYG{p}{;}
        \PYG{n}{seatemp}\PYG{p}{:}\PYG{n}{instrument} \PYG{o}{=} \PYG{l+s+s2}{\PYGZdq{}}\PYG{l+s+s2}{/instruments/SBE\PYGZus{}2490}\PYG{l+s+s2}{\PYGZdq{}} \PYG{p}{;}

  \PYG{o}{/}\PYG{o}{/} \PYG{k}{global} \PYG{n}{attributes}\PYG{p}{:}
        \PYG{p}{:}\PYG{n}{\PYGZus{}NCProperties} \PYG{o}{=} \PYG{l+s+s2}{\PYGZdq{}}\PYG{l+s+s2}{version=2,netcdf=4.7.2,hdf5=1.10.5}\PYG{l+s+s2}{\PYGZdq{}} \PYG{p}{;}
  \PYG{n}{data}\PYG{p}{:}

   \PYG{n}{seatemp} \PYG{o}{=}
    \PYG{l+m+mf}{7.4809}\PYG{p}{,} \PYG{l+m+mf}{7.439}\PYG{p}{,} \PYG{n}{\PYGZus{}}\PYG{p}{,} \PYG{l+m+mf}{7.403}\PYG{p}{,} \PYG{l+m+mf}{7.3647}\PYG{p}{,} \PYG{l+m+mf}{7.3497} \PYG{p}{;}

  \PYG{n}{group}\PYG{p}{:} \PYG{n}{instruments} \PYG{p}{\PYGZob{}}
    \PYG{n}{dimensions}\PYG{p}{:}
     \PYG{n}{NCOLUMNS} \PYG{o}{=} \PYG{l+m+mi}{1} \PYG{p}{;}

    \PYG{n}{group}\PYG{p}{:} \PYG{n}{SBE\PYGZus{}2490} \PYG{p}{\PYGZob{}}
      \PYG{n}{variables}\PYG{p}{:}
        \PYG{n}{string} \PYG{n}{instrument\PYGZus{}pid}\PYG{p}{(}\PYG{n}{NCOLUMNS}\PYG{p}{)} \PYG{p}{;}
           \PYG{n}{instrument\PYGZus{}pid}\PYG{p}{:}\PYG{n}{long\PYGZus{}name} \PYG{o}{=} \PYG{l+s+s2}{\PYGZdq{}}\PYG{l+s+s2}{Instrument identifier}\PYG{l+s+s2}{\PYGZdq{}} \PYG{p}{;}

      \PYG{o}{/}\PYG{o}{/} \PYG{n}{group} \PYG{n}{attributes}\PYG{p}{:}
           \PYG{p}{:}\PYG{n}{date\PYGZus{}valid\PYGZus{}from} \PYG{o}{=} \PYG{l+s+s2}{\PYGZdq{}}\PYG{l+s+s2}{2020\PYGZhy{}01\PYGZhy{}31T00:00:00Z}\PYG{l+s+s2}{\PYGZdq{}} \PYG{p}{;}
           \PYG{p}{:}\PYG{n}{date\PYGZus{}valid\PYGZus{}to} \PYG{o}{=} \PYG{l+s+s2}{\PYGZdq{}}\PYG{l+s+s2}{2020\PYGZhy{}08\PYGZhy{}16T00:00:00Z}\PYG{l+s+s2}{\PYGZdq{}} \PYG{p}{;}

      \PYG{n}{data}\PYG{p}{:}

       \PYG{n}{instrument\PYGZus{}pid} \PYG{o}{=} \PYG{l+s+s2}{\PYGZdq{}}\PYG{l+s+s2}{http://hdl.handle.net/21.T11998/0000\PYGZhy{}001A\PYGZhy{}3905\PYGZhy{}F}\PYG{l+s+s2}{\PYGZdq{}} \PYG{p}{;}

      \PYG{p}{\PYGZcb{}} \PYG{o}{/}\PYG{o}{/} \PYG{n}{group} \PYG{n}{SBE\PYGZus{}2490}
    \PYG{p}{\PYGZcb{}} \PYG{o}{/}\PYG{o}{/} \PYG{n}{group} \PYG{n}{instruments}
  \PYG{p}{\PYGZcb{}}
\end{sphinxVerbatim}

\sphinxAtStartPar
The National Centres for Environmental Information (NCEI) at the
National Oceanic and Atmospheric Administration (NOAA) in the US,
report instruments using a CF\sphinxhyphen{}NetCDF specification%
\begin{footnote}[4]\sphinxAtStartFootnote
\sphinxurl{https://www.ncei.noaa.gov/data/oceans/ncei/formats/netcdf/v2.0/index.html}
%
\end{footnote}. These
are either global attributes specified using the \sphinxstyleemphasis{instrument}
attribute from the Attribute Convention for Data Discovery (ACDD)
1\sphinxhyphen{}3. Alternatively they are defined as empty geophysical variables
within the root group of the NetCDF file. In the latter case,
the instrument PID may be expressed as an attribute \sphinxstyleemphasis{instrument\_pid}
within the recommended variable attributes as shown in
\hyperref[\detokenize{white-paper/linking-datasets:snip-link-pidinst-netcdf}]{Snippet \ref{\detokenize{white-paper/linking-datasets:snip-link-pidinst-netcdf}}}. Alternatively, an \sphinxstyleemphasis{instrument\_pid}
attribute could be added to the set of global attributes.
\sphinxSetupCaptionForVerbatim{Addition of an instrument PID attribute to NCEI CF\sphinxhyphen{}NetCDF
files v2.0.}
\def\sphinxLiteralBlockLabel{\label{\detokenize{white-paper/linking-datasets:snip-link-pidinst-netcdf}}}
\begin{sphinxVerbatim}[commandchars=\\\{\}]
    \PYG{n}{char} \PYG{n}{instrument1} \PYG{p}{;}
  \PYG{n}{instrument1}\PYG{p}{:}\PYG{n}{instrument\PYGZus{}pid} \PYG{o}{=} \PYG{l+s+s2}{\PYGZdq{}}\PYG{l+s+s2}{http://hdl.handle.net/21.T11998/0000\PYGZhy{}001A\PYGZhy{}3905\PYGZhy{}F}\PYG{l+s+s2}{\PYGZdq{}} \PYG{p}{;}
            \PYG{n}{instrument1}\PYG{p}{:}\PYG{n}{long\PYGZus{}name} \PYG{o}{=} \PYG{l+s+s2}{\PYGZdq{}}\PYG{l+s+s2}{Seabird 37 Microcat}\PYG{l+s+s2}{\PYGZdq{}} \PYG{p}{;}
            \PYG{n}{instrument1}\PYG{p}{:}\PYG{n}{ncei\PYGZus{}name} \PYG{o}{=} \PYG{l+s+s2}{\PYGZdq{}}\PYG{l+s+s2}{CTD}\PYG{l+s+s2}{\PYGZdq{}} \PYG{p}{;}
            \PYG{n}{instrument1}\PYG{p}{:}\PYG{n}{make\PYGZus{}model} \PYG{o}{=} \PYG{l+s+s2}{\PYGZdq{}}\PYG{l+s+s2}{SBE\PYGZhy{}37}\PYG{l+s+s2}{\PYGZdq{}} \PYG{p}{;}
            \PYG{n}{instrument1}\PYG{p}{:}\PYG{n}{serial\PYGZus{}number} \PYG{o}{=} \PYG{l+s+s2}{\PYGZdq{}}\PYG{l+s+s2}{1859723}\PYG{l+s+s2}{\PYGZdq{}} \PYG{p}{;}
            \PYG{n}{instrument1}\PYG{p}{:}\PYG{n}{calibration\PYGZus{}date} \PYG{o}{=} \PYG{l+s+s2}{\PYGZdq{}}\PYG{l+s+s2}{2016\PYGZhy{}03\PYGZhy{}25}\PYG{l+s+s2}{\PYGZdq{}} \PYG{p}{;}
            \PYG{n}{instrument1}\PYG{p}{:}\PYG{n}{accuracy} \PYG{o}{=} \PYG{l+s+s2}{\PYGZdq{}}\PYG{l+s+s2}{\PYGZdq{}} \PYG{p}{;}
            \PYG{n}{instrument1}\PYG{p}{:}\PYG{n}{precision} \PYG{o}{=} \PYG{l+s+s2}{\PYGZdq{}}\PYG{l+s+s2}{\PYGZdq{}} \PYG{p}{;}
            \PYG{n}{instrument1}\PYG{p}{:}\PYG{n}{comment} \PYG{o}{=} \PYG{l+s+s2}{\PYGZdq{}}\PYG{l+s+s2}{serial number and calibration dates are bogus}\PYG{l+s+s2}{\PYGZdq{}} \PYG{p}{;}
\end{sphinxVerbatim}


\subsubsection{OpenAIRE CERIF metadata}
\label{\detokenize{white-paper/linking-datasets:openaire-cerif-metadata}}
\sphinxAtStartPar
The \sphinxstyleemphasis{OpenAIRE Guidelines for CRIS Managers} %
\begin{footnote}[6]\sphinxAtStartFootnote
Dvořák, Jan, Czerniak, Andreas, \& Ivanović, Dragan. (2023). OpenAIRE
Guidelines for CRIS Managers 1.2 (1.2.0). \sphinxstyleemphasis{Zenodo}.
\sphinxurl{https://doi.org/10.5281/zenodo.8050936}
%
\end{footnote}
provide orientation for Research Information System (CRIS) managers to
expose their metadata in a way that is compatible with the OpenAIRE
infrastructure as well as the European Open Science Cloud (EOSC).
These Guidelines also serve as an example of a CERIF\sphinxhyphen{}based (Common
European Research Information Format) standard for information
interchange between individual CRISs and other research
e\sphinxhyphen{}Infrastructures.

\sphinxAtStartPar
The metadata format described by the Guidelines are includes Equipment
which could contain Instruments as well via the \sphinxhref{https://openaire-guidelines-for-cris-managers.readthedocs.io/en/v1.2.0/cerif\_xml\_product\_entity.html\#generatedby}{GeneratedBy property}.
\sphinxSetupCaptionForVerbatim{Use of the equipment entity for an instrument in
exposed in a product (dataset) metadata record.
Detailed \sphinxhref{https://github.com/openaire/guidelines-cris-managers/blob/cb96b925159655adfd97fb11c4a93f3d20c8cbef/samples/openaire\_cerif\_xml\_example\_products.xml\#L30}{product (dataset) example} at \sphinxstyleemphasis{OpenAIRE
Guidelines for CRIS Managers repository on GitHub}.}
\def\sphinxLiteralBlockLabel{\label{\detokenize{white-paper/linking-datasets:id7}}\label{\detokenize{white-paper/linking-datasets:snip-link-product-oaire-cerif-xml}}}
\begin{sphinxVerbatim}[commandchars=\\\{\}]
  \PYG{n+nt}{\PYGZlt{}GeneratedBy}\PYG{n+nt}{\PYGZgt{}}
    \PYG{n+nt}{\PYGZlt{}Equipment} \PYG{n+na}{id=}\PYG{l+s}{\PYGZdq{}82394876\PYGZdq{}}\PYG{n+nt}{\PYGZgt{}}
        \PYG{n+nt}{\PYGZlt{}Name} \PYG{n+na}{xml:lang=}\PYG{l+s}{\PYGZdq{}en\PYGZdq{}}\PYG{n+nt}{\PYGZgt{}}E2 \PYGZhy{} Flat\PYGZhy{}Cone Diffractometer\PYG{n+nt}{\PYGZlt{}/Name\PYGZgt{}}
        \PYG{n+nt}{\PYGZlt{}Identifier} \PYG{n+na}{type=}\PYG{l+s}{\PYGZdq{}DOI\PYGZdq{}}\PYG{n+nt}{\PYGZgt{}}https://doi.org/10.5442/NI000001\PYG{n+nt}{\PYGZlt{}/Identifier\PYGZgt{}}
        \PYG{n+nt}{\PYGZlt{}Description} \PYG{n+na}{xml:lang=}\PYG{l+s}{\PYGZdq{}en\PYGZdq{}}\PYG{n+nt}{\PYGZgt{}}A 3\PYGZhy{}dimensional part of the reciprocal space can be scanned in less then five steps by combining the “off\PYGZhy{}plane Bragg\PYGZhy{}scattering” and the flat\PYGZhy{}cone layer concept while using a new computer\PYGZhy{}controlled tilting axis of the detector bank. Parasitic scattering from cryostat or furnace walls is reduced by an oscillating \PYGZbs{}\PYGZdq{}radial\PYGZbs{}\PYGZdq{} collimator. The datasets and all connected information is stored in one independent NeXus file format for each measurement and can be easily archived. The software package TVneXus deals with the raw data sets, the transformed physical spaces and the usual data analysis tools (e.g. MatLab). TVneXus can convert to various data sets e.g. into powder diffractograms, linear detector projections, rotation crystal pictures or the 2D/3D reciprocal space.\PYG{n+nt}{\PYGZlt{}/Description\PYGZgt{}}
    \PYG{n+nt}{\PYGZlt{}/Equipment\PYGZgt{}}
  \PYG{n+nt}{\PYGZlt{}/GeneratedBy\PYGZgt{}}
\end{sphinxVerbatim}

\sphinxAtStartPar
The products (dataset) relates internal to the Equipment record via
the \sphinxstyleemphasis{id} attribute, eg. 82394874.  The metadata for the equipment
itself is exposed via equipment metadata record and described in the
\sphinxhref{https://openaire-guidelines-for-cris-managers.readthedocs.io/en/v1.2.0/cerif\_xml\_equipment\_entity.html}{Equipment entity}.
\sphinxSetupCaptionForVerbatim{Use of the equipment entity for an instrument in
exposed in a product (dataset) metadata record.
Detailed \sphinxhref{https://github.com/openaire/guidelines-cris-managers/blob/cb96b925159655adfd97fb11c4a93f3d20c8cbef/samples/openaire\_cerif\_xml\_example\_equipments.xml\#L18C1-L29C17}{equipment example} at \sphinxstyleemphasis{OpenAIRE Guidelines for
CRIS Managers repository on GitHub}.}
\def\sphinxLiteralBlockLabel{\label{\detokenize{white-paper/linking-datasets:id8}}\label{\detokenize{white-paper/linking-datasets:snip-link-equipment-oaire-cerif-xml}}}
\begin{sphinxVerbatim}[commandchars=\\\{\}]
  \PYG{n+nt}{\PYGZlt{}Equipment} \PYG{n+na}{xmlns=}\PYG{l+s}{\PYGZdq{}https://www.openaire.eu/cerif\PYGZhy{}profile/1.2/\PYGZdq{}} \PYG{n+na}{id=}\PYG{l+s}{\PYGZdq{}82394876\PYGZdq{}}\PYG{n+nt}{\PYGZgt{}}
    \PYG{n+nt}{\PYGZlt{}Name} \PYG{n+na}{xml:lang=}\PYG{l+s}{\PYGZdq{}en\PYGZdq{}}\PYG{n+nt}{\PYGZgt{}}E2 \PYGZhy{} Flat\PYGZhy{}Cone Diffractometer\PYG{n+nt}{\PYGZlt{}/Name\PYGZgt{}}
    \PYG{n+nt}{\PYGZlt{}Identifier} \PYG{n+na}{type=}\PYG{l+s}{\PYGZdq{}DOI\PYGZdq{}}\PYG{n+nt}{\PYGZgt{}}https://doi.org/10.5442/NI000001\PYG{n+nt}{\PYGZlt{}/Identifier\PYGZgt{}}
    \PYG{n+nt}{\PYGZlt{}Description} \PYG{n+na}{xml:lang=}\PYG{l+s}{\PYGZdq{}en\PYGZdq{}}\PYG{n+nt}{\PYGZgt{}}A 3\PYGZhy{}dimensional part of the reciprocal space can be scanned in less then five steps by combining the “off\PYGZhy{}plane Bragg\PYGZhy{}scattering” and the flat\PYGZhy{}cone layer concept while using a new computer\PYGZhy{}controlled tilting axis of the detector bank. Parasitic scattering from cryostat or furnace walls is reduced by an oscillating \PYGZbs{}\PYGZdq{}radial\PYGZbs{}\PYGZdq{} collimator. The datasets and all connected information is stored in one independent NeXus file format for each measurement and can be easily archived. The software package TVneXus deals with the raw data sets, the transformed physical spaces and the usual data analysis tools (e.g. MatLab). TVneXus can convert to various data sets e.g. into powder diffractograms, linear detector projections, rotation crystal pictures or the 2D/3D reciprocal space.\PYG{n+nt}{\PYGZlt{}/Description\PYGZgt{}}
    \PYG{n+nt}{\PYGZlt{}Owner}\PYG{n+nt}{\PYGZgt{}}
      \PYG{n+nt}{\PYGZlt{}OrgUnit} \PYG{n+na}{id=}\PYG{l+s}{\PYGZdq{}OrgUnits/350002\PYGZdq{}}\PYG{n+nt}{\PYGZgt{}}
        \PYG{n+nt}{\PYGZlt{}Acronym}\PYG{n+nt}{\PYGZgt{}}HZB\PYG{n+nt}{\PYGZlt{}/Acronym\PYGZgt{}}
        \PYG{n+nt}{\PYGZlt{}Name} \PYG{n+na}{xml:lang=}\PYG{l+s}{\PYGZdq{}de\PYGZdq{}}\PYG{n+nt}{\PYGZgt{}}Helmholtz\PYGZhy{}Zentrum Berlin Für Materialien Und Energie\PYG{n+nt}{\PYGZlt{}/Name\PYGZgt{}}
        \PYG{n+nt}{\PYGZlt{}Name} \PYG{n+na}{xml:lang=}\PYG{l+s}{\PYGZdq{}en\PYGZdq{}}\PYG{n+nt}{\PYGZgt{}}Helmholtz\PYGZhy{}Zentrum Berlin\PYG{n+nt}{\PYGZlt{}/Name\PYGZgt{}}
        \PYG{n+nt}{\PYGZlt{}RORID}\PYG{n+nt}{\PYGZgt{}}https://ror.org/02aj13c28\PYG{n+nt}{\PYGZlt{}/RORID\PYGZgt{}}
      \PYG{n+nt}{\PYGZlt{}/OrgUnit\PYGZgt{}}
    \PYG{n+nt}{\PYGZlt{}/Owner\PYGZgt{}}
  \PYG{n+nt}{\PYGZlt{}/Equipment\PYGZgt{}}
\end{sphinxVerbatim}


\subsection{Current, planned and potential adoption}
\label{\detokenize{white-paper/adoption:current-planned-and-potential-adoption}}\label{\detokenize{white-paper/adoption::doc}}

\subsubsection{Helmholtz\sphinxhyphen{}Zentrum Berlin für Materialien und Energie (HZB)}
\label{\detokenize{white-paper/adoption:helmholtz-zentrum-berlin-fur-materialien-und-energie-hzb}}
\sphinxAtStartPar
HZB minted four DOIs with DataCite for HZB instruments: two beamlines
at the neutron source BER II;%
\begin{footnote}[1]\sphinxAtStartFootnote
\sphinxurl{https://doi.org/10.5442/NI000001}
%
\end{footnote}$^{\text{,}}$%
\begin{footnote}[2]\sphinxAtStartFootnote
\sphinxurl{https://doi.org/10.5442/NI000002}
%
\end{footnote} one
beamline at the synchrotron light source BESSY II;%
\begin{footnote}[3]\sphinxAtStartFootnote
\sphinxurl{https://doi.org/10.5442/NI000003}
%
\end{footnote} and
one experimental station at BESSY II.%
\begin{footnote}[4]\sphinxAtStartFootnote
\sphinxurl{https://doi.org/10.5442/NI000004}
%
\end{footnote} The DOIs resolve
to the respective instrument page from the HZB instrument database
that did already exist before and was thus not created for this
purpose.  One particularity with these instruments is that they are
custom built by HZB.  Thus, in the metadata HZB appears as \sphinxtitleref{Creator}
as well as \sphinxtitleref{Contributor} with property \sphinxtitleref{contributorType} value
\sphinxtitleref{HostingInstitution}.  It is noteworthy that one of the DOIs uses the
additional property \sphinxtitleref{fundingReference} from the DataCite schema to
acknowledge external funding that HZB received for upgrading the
instrument.  This property was not considered in the PIDINST schema,
or in the DataCite mapping.  HZB plans to continue the adoption and to
mint DOIs for all its beamlines and experimental stations that are in
user operation in the near future.


\subsubsection{British Oceanographic Data Centre (BODC)}
\label{\detokenize{white-paper/adoption:british-oceanographic-data-centre-bodc}}
\sphinxAtStartPar
The British Oceanographic Data Centre (BODC) is a national facility
for preserving and distributing oceanographic and marine data.  BODC
tested the ePIC implementation in web\sphinxhyphen{}published, sensor technical
metadata descriptions encoded in the Open Geospatial Consortium’s
(OGC) \sphinxhref{https://www.opengeospatial.org/standards/sensorml}{SensorML} open standards for conceptualising and integrating
real\sphinxhyphen{}world sensors.  In an initial test case, a PID was minted for a
Sea\sphinxhyphen{}Bird Scientific SBE37 Microcat regularly deployed on fixed\sphinxhyphen{}point
moorings in the \sphinxhref{https://projects.noc.ac.uk/pap/}{Porcupine Abyssal Plain Sustained Observatory
(PAP\sphinxhyphen{}SO)} in the north Atlantic.  For further details see
{\hyperref[\detokenize{white-paper/landing-page-encoding:landing-page-encoding-swe}]{\sphinxcrossref{\DUrole{std,std-ref}{Sensor web enablement (SWE)}}}}.  BODC plan to continue adoption
identifying sensors on large research vessels owned by the Natural
Environment Research Council (NERC) and managed by the National
Oceanography Centre (NOC) and British Antarctic Survey (BAS).  PIDs
will be used to manage sensor data and metadata workflows from ‘deck
to desktop’ as part of a UK initiative, I/Ocean.


\subsubsection{EISCAT3D}
\label{\detokenize{white-paper/adoption:eiscat3d}}
\sphinxAtStartPar
\sphinxhref{https://eiscat.se/business/eiscat3d7/}{EISCAT3D} will be an international research infrastructure, using
radar observations and the incoherent scatter technique for studies of
the atmosphere and near\sphinxhyphen{}Earth space environment above the
Fenno\sphinxhyphen{}Scandinavian Arctic as well as for the support of the solar
system and radio astronomy sciences.  EISCAT3D will implement
persistent identification for instruments following the
recommendations by PIDINST.  The radar is complex, more digital than
previous radars, and is roughly divided into a number of separate
units.  While software is a substantial constituent of these units,
they can be regarded as hardware units, each persistently identified.
Updates to the units will be primarily to software and result in new
unit versions with own PIDs.  The radar itself can also be
persistently identified and the relation type HasComponent can be used
to relate to the persistently identified units.


\subsubsection{SENSOR.awi.de and PANGAEA}
\label{\detokenize{white-paper/adoption:sensor-awi-de-and-pangaea}}
\sphinxAtStartPar
The Alfred Wegener Institute Helmholtz Centre for Polar and Marine
Research (AWI) has been continuously committed to develop and sustain
an eResearch infrastructure for coherent discovery, view,
dissemination, and archival of scientific data and related information
in polar and marine regions.  In order to address the increasing
heterogeneity of research platforms and respective devices and sensors
along with varying project\sphinxhyphen{}driven requirements, a generic and modular
framework has been built intended to support the flow of sensor
observations to archives (O2A).%
\begin{footnote}[5]\sphinxAtStartFootnote
Koppe, R., Gerchow, P., Macario, A., Haas, A., Schäfer\sphinxhyphen{}Neth, C.
and Pfeiffenberger, H. (2015): O2A: A Generic Framework for Enabling
the Flow of Sensor Observations to Archives and Publications, OCEANS
2015 Genova. doi: 10.1109/OCEANS\sphinxhyphen{}Genova.2015.7271657
%
\end{footnote} In this context,
SENSOR.awi.de, available since 2015, is an O2A component dedicated to
the registry of research platforms, devices and sensors and in the
meantime in use by several international partners (e.g. MOSAiC
project).  SENSOR.awi.de has been built using OGC SensorML standard
and all individual records, to date over 4000, are assigned a
persistent identifier using UUIDs in the handle syntax along with
automated generation of a record citation.  Terminologies (e.g.,
controlled vocabularies) are used to define sensor categories (NERC
L05) as well as sensor types and models (NERC L22).  The data model of
SENSOR.awi.de is compliant with the PIDINST schema and the additional
implementation of Datacite DOIS for sensors is to date under
evaluation.  The ultimate goal of SENSOR.awi.de is to enhance the
quality of published and archived data in PANGAEA by providing
complete metadata and persistent identifiers on instruments and
sensors used in the data acquisition process
(\hyperref[\detokenize{white-paper/linking-datasets:fig-link-pangea}]{Figure \ref{\detokenize{white-paper/linking-datasets:fig-link-pangea}}}).  Given that platforms and sensors evolve
in time (sensors are being calibrated, instrument payload changes,
etc), SENSOR.awi.de also supports record versioning by maintaining an
audit trail of changes in the XML record.

\sphinxAtStartPar
PANGAEA is a digital repository for environmental research data and
the dedicated long term archive within the O2A framework jointly
operated by the AWI and MARUM (University Bremen).  Each dataset is
made available with its descriptive metadata, including the relations
with research resources (e.g., articles, funder, instrument and
specimen, if applicable).  As a data provider, PANGAEA only curates
limited information of a device such as device name, identifier and
type.  As an effort to standardize device type and name, currently the
repository applies external terminologies, in particular the NERC L05
device category vocabulary and the L22 device catalogue.  The
repository has developed tailor\sphinxhyphen{}made client applications to import
these terminologies in a periodic, incremental manner.  For both the
persistent identification as well as for the detailed description of
instruments, PANGAEA thus relies on institutional instrument
registries such as SENSOR.awi.de and uses their issued PIDs to
uniquely identify instruments which have been used to acquire data
archived at PANGAEA.  Since AWI and PANGAEA use the same
vocabularies/terminologies as well as PIDs to represent devices, they
facilitate easy integration of datasets in particular during transfer
of near real time data from O2A raw data staging areas via data
quality control services etc to their final destination, the PANGAEA
data archive.\sphinxfootnotemark[5]


\subsubsection{ICOS}
\label{\detokenize{white-paper/adoption:icos}}
\sphinxAtStartPar
The Integrated Carbon Observation System (ICOS) is a pan\sphinxhyphen{}european
research infrastructure for quantifying and understanding the
greenhouse gas balance of the European continent.  It conducts many
continuous in\sphinxhyphen{}situ measurements like gas concentrations, wind speed
and direction, humidity, temperature, etc.  To deliver high quality
measurement data, ICOS considers the adoption of a persistent
identifier for instruments a must for documenting data provenance and
tracking calibration history.


\subsubsection{B2INST}
\label{\detokenize{white-paper/adoption:b2inst}}
\sphinxAtStartPar
B2INST is a service for registering, persistently identifying and
describing instruments.  The B2INST service fills the gap especially
for research groups or smaller communities, who might lack the
capability to operate a registry.

\sphinxAtStartPar
Communities and organisations can make use of the service to FAIRify
their instrument by registering their metadata and assigning the
instrument a PID.  This instrument\sphinxhyphen{}PID can then be added to research
outputs, such as journal articles and datasets.

\sphinxAtStartPar
B2INST provides several generic features, like assigning PIDs and DOIs
to the metadata, as well as presenting a landing page of the
instrument based on the registered metadata.  It also provides
additional features, like the possibility to upload data to the
registered instruments (such additional data can be almost everything
that supports the description of the instrument, e.g. calibration
protocols, pictures of the instruments, technical manuals, etc).  In
B2INST, the registered information is publicly available for everyone.
Creating or maintaining information requires authorization \sphinxhyphen{} for that
B2INST supports federated identity management, so users can use their
home accounts to log in to the system.

\sphinxAtStartPar
The identified use cases showed that communities have different
requirements for instrument metadata.  The PIDINST schema covers a
minimum set of metadata to describe instruments only.  B2INST provides
community extensions; thus, it is possible to add broader descriptions
of instruments and to support the requirements of different
communities.  Based on the PIDINST schema, communities can add
metadata extensions to better support their community needs.

\sphinxAtStartPar
The current plan foresees that B2INST will be offered as a public
service by EUDAT.  The initial proof\sphinxhyphen{}of\sphinxhyphen{}concept was set up by SURF.
It was further developed by the GWDG, which will operate the service
in a production mode.


\subsubsection{National Institute of Standards and Technology (NIST)}
\label{\detokenize{white-paper/adoption:national-institute-of-standards-and-technology-nist}}
\sphinxAtStartPar
Recently, a group of researchers collaborating with the Office of Data
and Informatics at NIST had deployed a proto\sphinxhyphen{}instance of an instrument
database.  It is our hope that this database will becomes a living
record of all instruments from NIST.  We view the persistent
identification of research instruments as an essential attribute
anyone engaged in the production of high\sphinxhyphen{}quality and reproducible
science.

\sphinxAtStartPar
We chose the SharePoint platform as our evaluation portal for the ease
of setting up an internal facing web interface.  We made a SharePoint
List object based on the RDA PIDINST schema v1.0 release.  Next, we
populated the columns using two data sources.  The first is the NIST
electron microscopy Nexus microscope inventory.  The second the NIST
Sunflower property databases.  For the Sunflower property database, we
limited our results to only instruments listed under a single division
with NIST so that our feasibility study stayed manageable.  In all,
there are 600+ instruments in the proto\sphinxhyphen{}database.

\sphinxAtStartPar
From the two data sources, we were able to populate many of the
required columns but not all.  For the \sphinxtitleref{LandingPage} requirement,
since the vast majority of our instruments do not have landing pages,
we programmatically generated these with the Pelican static site
generator using Sunflower property data.  The pages with
internally\sphinxhyphen{}resolvable IP addresses are added to a local web host with
the understanding that some of these can be made public at NIST’s
discretion.  For the \sphinxtitleref{Identifier} requirement, we minted these using
Handle, and then imported them into SharePoint.  For fields like
\sphinxtitleref{Owner} and \sphinxtitleref{ownerName}, NIST is identified as the top\sphinxhyphen{}level owner.
However, it is our hope that instrument custodians will self\sphinxhyphen{}report as
second or third\sphinxhyphen{}line owners as the practice of persistent identifier
for instruments take root.  It is worth mentioning that SharePoint
List objects are not capable of having nested (1\sphinxhyphen{}n) objects.
Therefore, our instrument database remains in prototype stage until a
suitable database is identified.  This work is on\sphinxhyphen{}going.


\subsubsection{Natural Environmental Data Service (EDS)}
\label{\detokenize{white-paper/adoption:natural-environmental-data-service-eds}}
\sphinxAtStartPar
The \sphinxhref{https://eds.ukri.org/}{NERC Environmental Data Service (EDS)} is a trusted UK facility
providing data stewardship services for environmental data across
all environmental science domains. The EDS is made up of five data
centres with domain specific expertise; the British Oceanographic
Data Centre (BODC), Centre for Environmental Data Analysis (CEDA),
Environmental Information Data Centre (EIDC), National Geoscience
Data Centre (NGDS), and UK Polar Data Centre (PDC). The NERC\sphinxhyphen{}supported
EDS brings the data centres together to provide an integrated data
service across all environmental science domains. As part of the
Research Data Cloud Pilot project funded by the UK Research and
Innovation Council (UKRI), the EDS will prototype digital
infrastructure to cite graphs of all the PIDs used to generate
formal environmental data collections derived from sensors.
These graphs (or ‘reliquaries’ of complex citations) will include
persistent identifiers for instruments following the PIDINST
recommendations.


\section{Contributors}
\label{\detokenize{white-paper/index:contributors}}
\sphinxAtStartPar
\sphinxstyleemphasis{Louise Darroch} (\sphinxhref{mailto:lorr@noc.ac.uk}{lorr@noc.ac.uk}, \sphinxurl{https://orcid.org/0000-0003-4163-9575}),
British Oceanographic Data Centre, National Oceanography Centre,
Liverpool, L3 5DA, United Kingdom

\sphinxAtStartPar
\sphinxstyleemphasis{Robert Huber} (\sphinxhref{mailto:rhuber@uni-bremen.de}{rhuber@uni\sphinxhyphen{}bremen.de}, \sphinxurl{https://orcid.org/0000-0003-3000-0020}),
MARUM \sphinxhyphen{} Center for Marine Environmental Sciences, University of Bremen,
Leobener Str. 8, 28359 Bremen, Germany

\sphinxAtStartPar
\sphinxstyleemphasis{Anusuriya Devaraju} (\sphinxhref{mailto:anusuriya.devaraju@csiro.au}{anusuriya.devaraju@csiro.au}, \sphinxurl{https://orcid.org/0000-0003-0870-3192}),
CSIRO Mineral Resources, 26 Dick Perry Avenue, Kensington WA 6151, Australia

\sphinxAtStartPar
\sphinxstyleemphasis{Ulrich Schwardmann} (\sphinxhref{mailto:ulrich.schwardmann@gwdg.de}{ulrich.schwardmann@gwdg.de}, \sphinxurl{https://orcid.org/0000-0001-6337-8674}),
GWDG, Gesellschaft für wissenschaftliche Datenverarbeitung Göttingen,
Burckhardtweg 4, 37077 Göttingen, Germany

\sphinxAtStartPar
\sphinxstyleemphasis{Rolf Krahl} (\sphinxhref{mailto:rolf.krahl@helmholtz-berlin.de}{rolf.krahl@helmholtz\sphinxhyphen{}berlin.de}, \sphinxurl{https://orcid.org/0000-0002-1266-3819}),
Helmholtz\sphinxhyphen{}Zentrum Berlin für Materialien und Energie,
Albert\sphinxhyphen{}Einstein\sphinxhyphen{}Str. 15, 12489 Berlin, Germany

\sphinxAtStartPar
\sphinxstyleemphasis{Sven Bingert} (\sphinxhref{mailto:sven.bingert@gwdg.de}{sven.bingert@gwdg.de}, \sphinxurl{https://orcid.org/0000-0001-9547-1582}),
GWDG, Gesellschaft für wissenschaftliche Datenverarbeitung Göttingen,
Burckhardtweg 4, 37077 Göttingen, Germany

\sphinxAtStartPar
\sphinxstyleemphasis{Philipp Wieder} (\sphinxhref{mailto:philipp.wieder@gwdg.de}{philipp.wieder@gwdg.de}, \sphinxurl{https://orcid.org/0000-0002-6992-1866}),
GWDG, Gesellschaft für wissenschaftliche Datenverarbeitung Göttingen,
Burckhardtweg 4, 37077 Göttingen, Germany

\sphinxAtStartPar
\sphinxstyleemphasis{Tibor Kálmán} (\sphinxhref{mailto:tibor.kalman@gwdg.de}{tibor.kalman@gwdg.de}, \sphinxurl{https://orcid.org/0000-0001-5194-5053}),
GWDG, Gesellschaft für wissenschaftliche Datenverarbeitung Göttingen,
Burckhardtweg 4, 37077 Göttingen, Germany

\sphinxAtStartPar
\sphinxstyleemphasis{Anita Bandrowski} (\sphinxhref{mailto:abandrowski@ucsd.edu}{abandrowski@ucsd.edu}, \sphinxurl{https://orcid.org/0000-0002-5497-0243}),
Department of Neuroscience, University of California at San Diego,
9500 Gilman Drive La Jolla, CA 92093\sphinxhyphen{}0662 and SciCrunch Inc, 9500
Gilman Drive La Jolla, CA 92093\sphinxhyphen{}0662

\sphinxAtStartPar
\sphinxstyleemphasis{Ted Habermann} (\sphinxhref{mailto:ted@tedhabermann.com}{ted@tedhabermann.com}, \sphinxurl{http://orcid.org/0000-0003-3585-6733}),
Metadata Game Changers, 3980 Broadway, Suite 103\sphinxhyphen{}185, Boulder,
Colorado, USA 80304

\sphinxAtStartPar
\sphinxstyleemphasis{Mark van de Sanden} (\sphinxhref{mailto:mark.vandesanden@surf.nl}{mark.vandesanden@surf.nl}, \sphinxurl{https://orcid.org/0000-0002-2718-8918}),
SURF, Science Park 140, 1098 XG Amsterdam, The Netherlands

\sphinxAtStartPar
\sphinxstyleemphasis{Claudio D’Onofrio} (\sphinxhref{mailto:claudio.donofrio@nateko.lu.se}{claudio.donofrio@nateko.lu.se}, \sphinxurl{https://orcid.org/0000-0002-1982-3889}),
ICOS Carbon Portal, Lund University, Physical Geography \& Ecosystem
Science, Sölvegatan 12, 223 62 Lund, Sweden

\sphinxAtStartPar
\sphinxstyleemphasis{Margareta Hellström} (\sphinxhref{mailto:margareta.hellstrom@nateko.lu.se}{margareta.hellstrom@nateko.lu.se}, \sphinxurl{https://orcid.org/0000-0002-4154-2610}),
ICOS Carbon Portal, Physical Geography \& Ecosystem Science, Lund
University, Sölvegatan 12, 223 62 Lund, Sweden

\sphinxAtStartPar
\sphinxstyleemphasis{Ingemar Häggström} (\sphinxhref{mailto:ingemar.haggstrom@eiscat.se}{ingemar.haggstrom@eiscat.se}, \sphinxurl{https://orcid.org/0000-0003-1070-6915}),
EISCAT Scientific Association, Box 812, 98128 Kiruna, Sweden

\sphinxAtStartPar
\sphinxstyleemphasis{Roland Koppe} (\sphinxhref{mailto:roland.koppe@awi.de}{roland.koppe@awi.de}, \sphinxurl{https://orcid.org/0000-0002-2826-3932}),
Computing and Data Centre, Alfred Wegener Institute Helmholtz Centre for
Polar and Marine Research. Am Handelshafen 12, 27570, Bremerhaven,
Germany

\sphinxAtStartPar
\sphinxstyleemphasis{Ana Macario} (\sphinxhref{mailto:ana.macario@awi.de}{ana.macario@awi.de}, \sphinxurl{https://orcid.org/0000-0003-3747-793X}),
Computing and Data Centre, Alfred Wegener Institute Helmholtz Centre for
Polar and Marine Research. Am Handelshafen 12, 27570, Bremerhaven,
Germany

\sphinxAtStartPar
\sphinxstyleemphasis{Tina Dohna} (\sphinxhref{mailto:tdohna@marum.de}{tdohna@marum.de}, \sphinxurl{https://orcid.org/0000-0002-5948-0980}),
MARUM \sphinxhyphen{} Center for Marine Environmental Sciences, University of Bremen,
Leobener Str. 8, 28199 Bremen, Germany

\sphinxAtStartPar
\sphinxstyleemphasis{Jens Klump} (\sphinxhref{mailto:jens.klump@csiro.au}{jens.klump@csiro.au}, \sphinxurl{https://orcid.org/0000-0001-5911-6022}),
CSIRO Mineral Resources, 26 Dick Perry Avenue, Kensington WA 6151, Australia

\sphinxAtStartPar
\sphinxstyleemphasis{Markus Stocker} (\sphinxhref{mailto:markus.stocker@tib.eu}{markus.stocker@tib.eu}, \sphinxurl{https://orcid.org/0000-0001-5492-3212}),
TIB \textendash{} Leibniz Information Centre for Science and Technology,
Welfengarten 1 B, 30167 Hannover, Germany and Leibniz University
Hannover, Welfengarten 1, 30167 Hannover, Germany

\sphinxAtStartPar
\sphinxstyleemphasis{Andreas Czerniak} (\sphinxhref{mailto:andreas.czerniak@uni-bielefeld.de}{andreas.czerniak@uni\sphinxhyphen{}bielefeld.de}, \sphinxurl{https://orcid.org/0000-0003-3883-4169}),
Bielefeld University Library, Universitätsstr. 25, 33615 Bielefeld, Germany

\sphinxAtStartPar
\sphinxstyleemphasis{Robert J. Hanisch} (\sphinxhref{mailto:robert.hanisch@nist.gov}{robert.hanisch@nist.gov}, \sphinxurl{https://orcid.org/0000-0002-6853-4602})
Office of Data and Informatics, Material Measurement Laboratory,
National Institute of Standards and Technology, Gaithersburg, MD, USA

\sphinxAtStartPar
\sphinxstyleemphasis{June W. Lau} (\sphinxhref{mailto:june.lau@nist.gov}{june.lau@nist.gov}, \sphinxurl{https://orcid.org/0000-0002-5233-4956})
Office of Data and Informatics, Material Measurement Laboratory,
National Institute of Standards and Technology, Gaithersburg, MD, USA

\sphinxAtStartPar
\sphinxstyleemphasis{The Research Data Alliance Persistent Identification of Instruments
Working Group members} (\sphinxurl{https://www.rd-alliance.org/node/57186/members})


\section{Competing interests}
\label{\detokenize{white-paper/index:competing-interests}}
\sphinxAtStartPar
Anita Bandrowski is a founder and CEO of SciCrunch Inc, a company
devoted to working with publishers to improve the scientific
literature.  All other authors declare that they have no competing
interests.


\chapter{ePIC Cookbook}
\label{\detokenize{epic-cookbook/index:epic-cookbook}}\label{\detokenize{epic-cookbook/index:id1}}\label{\detokenize{epic-cookbook/index::doc}}

\begin{savenotes}\sphinxattablestart
\centering
\begin{tabular}[t]{|*{2}{\X{1}{2}|}}
\hline

\sphinxAtStartPar
Document type
&
\begin{DUlineblock}{0em}
\item[] Research Data Alliance (RDA)
\item[] Persistent Identification of Instruments (PIDINST)
\item[] working group output report
\end{DUlineblock}
\\
\hline
\end{tabular}
\par
\sphinxattableend\end{savenotes}


\section{Introduction}
\label{\detokenize{epic-cookbook/intro:introduction}}\label{\detokenize{epic-cookbook/intro::doc}}
\sphinxAtStartPar
This cookbook enables instrument providers to create persistent
identifiers (PID) for instruments using the \sphinxhref{https://www.pidconsortium.net/}{ePIC infrastructure}.
ePIC is an international consortium that provides PID services for the
worldwide research community, allowing them to allocate and resolve
PIDs based on the \sphinxhref{https://www.handle.net/}{handle system}.  In 2019, ePIC published a
metadata schema for citing instruments, as part of the recommendations
resulting from the Research Data Alliance working group for the
persistent identification of instruments (\sphinxhref{https://www.rd-alliance.org/groups/persistent-identification-instruments-wg}{PIDINST}) referred to as
the \sphinxhref{https://github.com/rdawg-pidinst/schema}{PIDINST metadata schema}.  This document provides technical
guidance for publishing instrument PIDs through ePIC.


\section{PIDINST handles at ePIC}
\label{\detokenize{epic-cookbook/handles:pidinst-handles-at-epic}}\label{\detokenize{epic-cookbook/handles::doc}}
\sphinxAtStartPar
Properties, sub\sphinxhyphen{}properties and attributes of the PIDINST metadata
schema used in ePIC PID handle records can be viewed as follows.  Use
‘\#’ before ‘objects’ to interchange between human\sphinxhyphen{}readable and JSON
representations.  Each property, sub\sphinxhyphen{}property and attribute is
resolvable through a unique handle record:


\begin{savenotes}\sphinxattablestart
\centering
\begin{tabulary}{\linewidth}[t]{|T|T|}
\hline

\sphinxAtStartPar
Human\sphinxhyphen{}readable
&
\sphinxAtStartPar
\sphinxurl{http://dtr-test.pidconsortium.eu/\#objects/21.T11148/17ce618137e697852ea6}
\\
\hline
\sphinxAtStartPar
JSON representation
&
\sphinxAtStartPar
\sphinxurl{http://dtr-test.pidconsortium.eu/objects/21.T11148/17ce618137e697852ea6}
\\
\hline
\end{tabulary}
\par
\sphinxattableend\end{savenotes}


\subsection{Generating a new instrument PID}
\label{\detokenize{epic-cookbook/handles:generating-a-new-instrument-pid}}
\sphinxAtStartPar
ePIC handles are accessed and managed via a RESTful web service, using
the HTTP application protocol.  This service or Application
Programming Interface (API) uses JSON as the primary exchange format.
Available are the generic and more basic Handle API or, if implemented
at the PID service used, the ePIC API that comes with its own business
logic and additional services.  In our following examples PIDINST
handles are created via the ePIC API in the ePIC API test environment
with a view to moving the architecture to the production environment
in the future.  There is also an overview for the basic CRUD
operations on PIDs for either the ePIC API or the Handle API at the
end.

\sphinxAtStartPar
In order to generate new PIDs and assign them to your instruments, it
is necessary to become an ePIC member (provider), or work with one of
their current members or repositories that has their own ePIC API
endpoint.  To create a PID using the test environment you will need to
obtain credentials (username and password) for authentication using
the test environment prefix 21.T11998.  These can be obtained from
ePIC by emailing \sphinxhref{mailto:support@pidconsortium.net}{support@pidconsortium.net}.

\sphinxAtStartPar
PIDs are typically created using POST/PUT methods.  Using the POST
method will automatically generate a Universally Unique Identifier
(UUID) within the suffix of a handle record.  Alternatively a suffix
can be manually created via PUT method using a local identifier (see
\sphinxurl{https://doc.pidconsortium.eu/guides/api-create/}).

\sphinxAtStartPar
All examples below use cURL requests at the command line (in Linux).
Requests can also use PHP, Perl and Python (see
\sphinxurl{https://doc.pidconsortium.eu/guides/api-create/}).  Examples also use
the test API endpoint \sphinxurl{http://vm04.pid.gwdg.de:8081/handles/}.  Each
ePIC member may use their own API end\sphinxhyphen{}point.

\sphinxAtStartPar
To generate a PID handle record automatically generating a UUID for
the suffix:

\begin{sphinxVerbatim}[commandchars=\\\{\}]
\PYG{n}{curl} \PYG{o}{\PYGZhy{}}\PYG{n}{v} \PYG{o}{\PYGZhy{}}\PYG{n}{u} \PYG{l+s+s2}{\PYGZdq{}}\PYG{l+s+s2}{username:password}\PYG{l+s+s2}{\PYGZdq{}} \PYG{o}{\PYGZhy{}}\PYG{n}{H} \PYG{l+s+s2}{\PYGZdq{}}\PYG{l+s+s2}{Accept:application/json}\PYG{l+s+s2}{\PYGZdq{}} \PYG{o}{\PYGZhy{}}\PYG{n}{H} \PYG{l+s+s2}{\PYGZdq{}}\PYG{l+s+s2}{Content\PYGZhy{}Type:application/json}\PYG{l+s+s2}{\PYGZdq{}} \PYG{o}{\PYGZhy{}}\PYG{n}{X} \PYG{n}{POST} \PYG{o}{\PYGZhy{}}\PYG{o}{\PYGZhy{}}\PYG{n}{data} \PYG{l+s+s1}{\PYGZsq{}}\PYG{l+s+s1}{[}\PYG{l+s+s1}{\PYGZob{}}\PYG{l+s+s1}{\PYGZdq{}}\PYG{l+s+s1}{type}\PYG{l+s+s1}{\PYGZdq{}}\PYG{l+s+s1}{:}\PYG{l+s+s1}{\PYGZdq{}}\PYG{l+s+s1}{URL}\PYG{l+s+s1}{\PYGZdq{}}\PYG{l+s+s1}{,}\PYG{l+s+s1}{\PYGZdq{}}\PYG{l+s+s1}{parsed\PYGZus{}data}\PYG{l+s+s1}{\PYGZdq{}}\PYG{l+s+s1}{:}\PYG{l+s+s1}{\PYGZdq{}}\PYG{l+s+s1}{https://linkedsystems.uk/system/instance/TOOL0022\PYGZus{}2490/current/}\PYG{l+s+s1}{\PYGZdq{}}\PYG{l+s+s1}{\PYGZcb{}]}\PYG{l+s+s1}{\PYGZsq{}} \PYG{n}{http}\PYG{p}{:}\PYG{o}{/}\PYG{o}{/}\PYG{n}{vm04}\PYG{o}{.}\PYG{n}{pid}\PYG{o}{.}\PYG{n}{gwdg}\PYG{o}{.}\PYG{n}{de}\PYG{p}{:}\PYG{l+m+mi}{8081}\PYG{o}{/}\PYG{n}{handles}\PYG{o}{/}\PYG{l+m+mf}{21.}\PYG{n}{T11998}\PYG{o}{/}
\end{sphinxVerbatim}

\sphinxAtStartPar
\sphinxcode{\sphinxupquote{Result:}} \sphinxurl{https://vm04.pid.gwdg.de:8081/handles/21.T11998/0000-001A-64A4-A}

\sphinxAtStartPar
To generate a PID handle record automatically generating a UUID within
the suffix:

\begin{sphinxVerbatim}[commandchars=\\\{\}]
curl \PYGZhy{}v \PYGZhy{}u \PYGZdq{}username:password\PYGZdq{} \PYGZhy{}H \PYGZdq{}Accept:application/json\PYGZdq{} \PYGZhy{}H \PYGZdq{}Content\PYGZhy{}Type:application/json\PYGZdq{} \PYGZhy{}X POST \PYGZhy{}\PYGZhy{}data \PYGZsq{}[\PYGZob{}\PYGZdq{}type\PYGZdq{}:\PYGZdq{}URL\PYGZdq{},\PYGZdq{}parsed\PYGZus{}data\PYGZdq{}:\PYGZdq{}https://linkedsystems.uk/system/instance/TOOL0022\PYGZus{}2490/current/\PYGZdq{}\PYGZcb{}]\PYGZsq{} http://vm04.pid.gwdg.de:8081/handles/21.T11998/\PYGZbs{}?prefix=BODC\PYGZbs{}\PYGZam{}suffix=TEST
\end{sphinxVerbatim}

\sphinxAtStartPar
\sphinxcode{\sphinxupquote{Result:}} \sphinxurl{https://vm04.pid.gwdg.de:8081/handles/21.T11998/BODC-0000-001A-64A3-B-TEST}

\sphinxAtStartPar
To manually generate a suffix using PUT method:

\begin{sphinxVerbatim}[commandchars=\\\{\}]
\PYG{n}{curl} \PYG{o}{\PYGZhy{}}\PYG{n}{v} \PYG{o}{\PYGZhy{}}\PYG{n}{u} \PYG{l+s+s2}{\PYGZdq{}}\PYG{l+s+s2}{username:password}\PYG{l+s+s2}{\PYGZdq{}} \PYG{o}{\PYGZhy{}}\PYG{n}{H} \PYG{l+s+s2}{\PYGZdq{}}\PYG{l+s+s2}{Accept:application/json}\PYG{l+s+s2}{\PYGZdq{}} \PYG{o}{\PYGZhy{}}\PYG{n}{H} \PYG{l+s+s2}{\PYGZdq{}}\PYG{l+s+s2}{Content\PYGZhy{}Type:application/json}\PYG{l+s+s2}{\PYGZdq{}} \PYG{o}{\PYGZhy{}}\PYG{n}{X} \PYG{n}{PUT} \PYG{o}{\PYGZhy{}}\PYG{o}{\PYGZhy{}}\PYG{n}{data} \PYG{l+s+s1}{\PYGZsq{}}\PYG{l+s+s1}{[}\PYG{l+s+s1}{\PYGZob{}}\PYG{l+s+s1}{\PYGZdq{}}\PYG{l+s+s1}{type}\PYG{l+s+s1}{\PYGZdq{}}\PYG{l+s+s1}{:}\PYG{l+s+s1}{\PYGZdq{}}\PYG{l+s+s1}{URL}\PYG{l+s+s1}{\PYGZdq{}}\PYG{l+s+s1}{,}\PYG{l+s+s1}{\PYGZdq{}}\PYG{l+s+s1}{parsed\PYGZus{}data}\PYG{l+s+s1}{\PYGZdq{}}\PYG{l+s+s1}{:}\PYG{l+s+s1}{\PYGZdq{}}\PYG{l+s+s1}{https://linkedsystems.uk/system/instance/TOOL0022\PYGZus{}2490/current/}\PYG{l+s+s1}{\PYGZdq{}}\PYG{l+s+s1}{\PYGZcb{}]}\PYG{l+s+s1}{\PYGZsq{}} \PYG{n}{http}\PYG{p}{:}\PYG{o}{/}\PYG{o}{/}\PYG{n}{vm04}\PYG{o}{.}\PYG{n}{pid}\PYG{o}{.}\PYG{n}{gwdg}\PYG{o}{.}\PYG{n}{de}\PYG{p}{:}\PYG{l+m+mi}{8081}\PYG{o}{/}\PYG{n}{handles}\PYG{o}{/}\PYG{l+m+mf}{21.}\PYG{n}{T11998}\PYG{o}{/}\PYG{l+m+mi}{564987}\PYG{o}{\PYGZhy{}}\PYG{l+m+mi}{8865544}\PYG{o}{\PYGZhy{}}\PYG{l+m+mi}{9998}
\end{sphinxVerbatim}

\sphinxAtStartPar
\sphinxcode{\sphinxupquote{Result:}} \sphinxurl{https://vm04.pid.gwdg.de:8081/handles/21.T11998/564987-8865544-9998}


\subsection{Viewing PID handle records}
\label{\detokenize{epic-cookbook/handles:viewing-pid-handle-records}}
\sphinxAtStartPar
If you have specified the URL property in the handle record it will
automatically redirect you to it when you view the handle record:

\sphinxAtStartPar
\sphinxurl{http://hdl.handle.net/21.T11998/0000-001A-3905-F}

\sphinxAtStartPar
If you want to see the handle record then use:

\sphinxAtStartPar
\sphinxurl{http://hdl.handle.net/21.T11998/0000-001A-3905-F?noredirect}

\sphinxAtStartPar
The REST API calls can also yield JSON responses:

\sphinxAtStartPar
\sphinxurl{http://hdl.handle.net/api/handles/21.T11998/0000-001A-3905-F}


\subsection{Updating the description of a PID handle record}
\label{\detokenize{epic-cookbook/handles:updating-the-description-of-a-pid-handle-record}}
\sphinxAtStartPar
Properties are updated using the PUT method by either specifying the
JSON properties directly in the cURL request or parsing them via a
JSON file (see \sphinxcode{\sphinxupquote{JSON example}}).

\sphinxAtStartPar
Directly specifying properties within the cURL request:

\begin{sphinxVerbatim}[commandchars=\\\{\}]
\PYG{n}{curl} \PYG{o}{\PYGZhy{}}\PYG{n}{v} \PYG{o}{\PYGZhy{}}\PYG{n}{u} \PYG{l+s+s2}{\PYGZdq{}}\PYG{l+s+s2}{username:password}\PYG{l+s+s2}{\PYGZdq{}} \PYG{o}{\PYGZhy{}}\PYG{n}{H} \PYG{l+s+s2}{\PYGZdq{}}\PYG{l+s+s2}{Accept:application/json}\PYG{l+s+s2}{\PYGZdq{}} \PYG{o}{\PYGZhy{}}\PYG{n}{H} \PYG{l+s+s2}{\PYGZdq{}}\PYG{l+s+s2}{Content\PYGZhy{}Type:application/json}\PYG{l+s+s2}{\PYGZdq{}} \PYG{o}{\PYGZhy{}}\PYG{n}{X} \PYG{n}{PUT} \PYG{o}{\PYGZhy{}}\PYG{o}{\PYGZhy{}}\PYG{n}{data} \PYG{l+s+s1}{\PYGZsq{}}\PYG{l+s+s1}{[}\PYG{l+s+s1}{\PYGZob{}}\PYG{l+s+s1}{\PYGZdq{}}\PYG{l+s+s1}{type}\PYG{l+s+s1}{\PYGZdq{}}\PYG{l+s+s1}{: }\PYG{l+s+s1}{\PYGZdq{}}\PYG{l+s+s1}{21.T11148/8eb858ee0b12e8e463a5}\PYG{l+s+s1}{\PYGZdq{}}\PYG{l+s+s1}{,}\PYG{l+s+s1}{\PYGZdq{}}\PYG{l+s+s1}{parsed\PYGZus{}data}\PYG{l+s+s1}{\PYGZdq{}}\PYG{l+s+s1}{: }\PYG{l+s+s1}{\PYGZdq{}}\PYG{l+s+s1}{\PYGZob{}}\PYG{l+s+se}{\PYGZbs{}\PYGZdq{}}\PYG{l+s+s1}{identifierValue}\PYG{l+s+se}{\PYGZbs{}\PYGZdq{}}\PYG{l+s+s1}{:}\PYG{l+s+se}{\PYGZbs{}\PYGZdq{}}\PYG{l+s+s1}{http://hdl.handle.net/21.T11998/BODC\PYGZhy{}0000\PYGZhy{}001A\PYGZhy{}64A3\PYGZhy{}B\PYGZhy{}TEST}\PYG{l+s+se}{\PYGZbs{}\PYGZdq{}}\PYG{l+s+s1}{,}\PYG{l+s+se}{\PYGZbs{}\PYGZdq{}}\PYG{l+s+s1}{identiferType}\PYG{l+s+se}{\PYGZbs{}\PYGZdq{}}\PYG{l+s+s1}{:}\PYG{l+s+se}{\PYGZbs{}\PYGZdq{}}\PYG{l+s+s1}{MeasuringInstrument}\PYG{l+s+se}{\PYGZbs{}\PYGZdq{}}\PYG{l+s+s1}{\PYGZcb{}}\PYG{l+s+s1}{\PYGZdq{}}\PYG{l+s+s1}{\PYGZcb{},}\PYG{l+s+s1}{\PYGZob{}}\PYG{l+s+s1}{\PYGZdq{}}\PYG{l+s+s1}{type}\PYG{l+s+s1}{\PYGZdq{}}\PYG{l+s+s1}{: }\PYG{l+s+s1}{\PYGZdq{}}\PYG{l+s+s1}{21.T11148/4eaec4bc0f1df68ab2a7}\PYG{l+s+s1}{\PYGZdq{}}\PYG{l+s+s1}{,}\PYG{l+s+s1}{\PYGZdq{}}\PYG{l+s+s1}{parsed\PYGZus{}data}\PYG{l+s+s1}{\PYGZdq{}}\PYG{l+s+s1}{: }\PYG{l+s+s1}{\PYGZdq{}}\PYG{l+s+s1}{[}\PYG{l+s+s1}{\PYGZob{}}\PYG{l+s+se}{\PYGZbs{}\PYGZdq{}}\PYG{l+s+s1}{Owner}\PYG{l+s+se}{\PYGZbs{}\PYGZdq{}}\PYG{l+s+s1}{: }\PYG{l+s+s1}{\PYGZob{}}\PYG{l+s+se}{\PYGZbs{}\PYGZdq{}}\PYG{l+s+s1}{ownerName}\PYG{l+s+se}{\PYGZbs{}\PYGZdq{}}\PYG{l+s+s1}{:}\PYG{l+s+se}{\PYGZbs{}\PYGZdq{}}\PYG{l+s+s1}{National Oceanography Centre}\PYG{l+s+se}{\PYGZbs{}\PYGZdq{}}\PYG{l+s+s1}{,}\PYG{l+s+se}{\PYGZbs{}\PYGZdq{}}\PYG{l+s+s1}{ownerContact}\PYG{l+s+se}{\PYGZbs{}\PYGZdq{}}\PYG{l+s+s1}{:}\PYG{l+s+se}{\PYGZbs{}\PYGZdq{}}\PYG{l+s+s1}{louise.darroch@bodc.ac.uk}\PYG{l+s+se}{\PYGZbs{}\PYGZdq{}}\PYG{l+s+s1}{,}\PYG{l+s+se}{\PYGZbs{}\PYGZdq{}}\PYG{l+s+s1}{ownerIdentifier}\PYG{l+s+se}{\PYGZbs{}\PYGZdq{}}\PYG{l+s+s1}{:}\PYG{l+s+s1}{\PYGZob{}}\PYG{l+s+se}{\PYGZbs{}\PYGZdq{}}\PYG{l+s+s1}{ownerIdentifierValue}\PYG{l+s+se}{\PYGZbs{}\PYGZdq{}}\PYG{l+s+s1}{:}\PYG{l+s+se}{\PYGZbs{}\PYGZdq{}}\PYG{l+s+s1}{http://vocab.nerc.ac.uk/collection/B75/current/ORG00009/}\PYG{l+s+se}{\PYGZbs{}\PYGZdq{}}\PYG{l+s+s1}{,}\PYG{l+s+se}{\PYGZbs{}\PYGZdq{}}\PYG{l+s+s1}{ownerIdentifierType}\PYG{l+s+se}{\PYGZbs{}\PYGZdq{}}\PYG{l+s+s1}{:}\PYG{l+s+se}{\PYGZbs{}\PYGZdq{}}\PYG{l+s+s1}{URL}\PYG{l+s+se}{\PYGZbs{}\PYGZdq{}}\PYG{l+s+s1}{\PYGZcb{}\PYGZcb{}\PYGZcb{}]}\PYG{l+s+s1}{\PYGZdq{}}\PYG{l+s+s1}{\PYGZcb{}]}\PYG{l+s+s1}{\PYGZsq{}} \PYG{n}{http}\PYG{p}{:}\PYG{o}{/}\PYG{o}{/}\PYG{n}{vm04}\PYG{o}{.}\PYG{n}{pid}\PYG{o}{.}\PYG{n}{gwdg}\PYG{o}{.}\PYG{n}{de}\PYG{p}{:}\PYG{l+m+mi}{8081}\PYG{o}{/}\PYG{n}{handles}\PYG{o}{/}\PYG{l+m+mf}{21.}\PYG{n}{T11998}\PYG{o}{/}\PYG{n}{BODC}\PYG{o}{\PYGZhy{}}\PYG{l+m+mi}{0000}\PYG{o}{\PYGZhy{}}\PYG{l+m+mi}{001}\PYG{n}{A}\PYG{o}{\PYGZhy{}}\PYG{l+m+mi}{64}\PYG{n}{A3}\PYG{o}{\PYGZhy{}}\PYG{n}{B}\PYG{o}{\PYGZhy{}}\PYG{n}{TEST}
\end{sphinxVerbatim}

\sphinxAtStartPar
\sphinxstyleemphasis{Note: Double quotes must be escaped with a backslash (\textbackslash{}) within the
JSON parsed\_data string}

\sphinxAtStartPar
Specifying properties with a JSON file:

\begin{sphinxVerbatim}[commandchars=\\\{\}]
\PYG{n}{curl} \PYG{o}{\PYGZhy{}}\PYG{n}{v} \PYG{o}{\PYGZhy{}}\PYG{n}{u} \PYG{l+s+s2}{\PYGZdq{}}\PYG{l+s+s2}{username:password}\PYG{l+s+s2}{\PYGZdq{}} \PYG{o}{\PYGZhy{}}\PYG{n}{H} \PYG{l+s+s2}{\PYGZdq{}}\PYG{l+s+s2}{Accept:application/json}\PYG{l+s+s2}{\PYGZdq{}} \PYG{o}{\PYGZhy{}}\PYG{n}{H} \PYG{l+s+s2}{\PYGZdq{}}\PYG{l+s+s2}{Content\PYGZhy{}Type:application/json}\PYG{l+s+s2}{\PYGZdq{}} \PYG{o}{\PYGZhy{}}\PYG{n}{X} \PYG{n}{PUT} \PYG{o}{\PYGZhy{}}\PYG{o}{\PYGZhy{}}\PYG{n}{data} \PYG{o}{@}\PYG{o}{/}\PYG{n}{users}\PYG{o}{/}\PYG{o}{.}\PYG{o}{.}\PYG{o}{.}\PYG{o}{/}\PYG{n}{ePIC\PYGZus{}json\PYGZus{}example}\PYG{o}{.}\PYG{n}{json} \PYG{n}{http}\PYG{p}{:}\PYG{o}{/}\PYG{o}{/}\PYG{n}{vm04}\PYG{o}{.}\PYG{n}{pid}\PYG{o}{.}\PYG{n}{gwdg}\PYG{o}{.}\PYG{n}{de}\PYG{p}{:}\PYG{l+m+mi}{8081}\PYG{o}{/}\PYG{n}{handles}\PYG{o}{/}\PYG{l+m+mf}{21.}\PYG{n}{T11998}\PYG{o}{/}\PYG{n}{BODC}\PYG{o}{\PYGZhy{}}\PYG{l+m+mi}{0000}\PYG{o}{\PYGZhy{}}\PYG{l+m+mi}{001}\PYG{n}{A}\PYG{o}{\PYGZhy{}}\PYG{l+m+mi}{64}\PYG{n}{A3}\PYG{o}{\PYGZhy{}}\PYG{n}{B}\PYG{o}{\PYGZhy{}}\PYG{n}{TEST}
\end{sphinxVerbatim}


\subsection{Managing PIDs}
\label{\detokenize{epic-cookbook/handles:managing-pids}}

\subsubsection{Using the ePIC API}
\label{\detokenize{epic-cookbook/handles:using-the-epic-api}}
\sphinxAtStartPar
The following HTTP protocol methods enable users to manage their PID
handle records using the ePIC API based on username\sphinxhyphen{}password.
Server: \sphinxcode{\sphinxupquote{vm04.pid.gwdg.de}}, Port: \sphinxcode{\sphinxupquote{8081}}, Resources: \sphinxcode{\sphinxupquote{handles/}}

\sphinxAtStartPar
\sphinxstylestrong{Get a PID:}

\begin{sphinxVerbatim}[commandchars=\\\{\}]
\PYG{n}{curl} \PYG{o}{\PYGZhy{}}\PYG{n}{D}\PYG{o}{\PYGZhy{}} \PYG{o}{\PYGZhy{}}\PYG{n}{u} \PYG{l+s+s2}{\PYGZdq{}}\PYG{l+s+s2}{username:password}\PYG{l+s+s2}{\PYGZdq{}} \PYG{o}{\PYGZhy{}}\PYG{n}{X} \PYG{n}{GET} \PYG{o}{\PYGZhy{}}\PYG{n}{H} \PYG{l+s+s2}{\PYGZdq{}}\PYG{l+s+s2}{Content\PYGZhy{}Type: application/json}\PYG{l+s+s2}{\PYGZdq{}} \PYG{n}{http}\PYG{p}{:}\PYG{o}{/}\PYG{o}{/}\PYG{n}{vm04}\PYG{o}{.}\PYG{n}{pid}\PYG{o}{.}\PYG{n}{gwdg}\PYG{o}{.}\PYG{n}{de}\PYG{p}{:}\PYG{l+m+mi}{8081}\PYG{o}{/}\PYG{n}{handles}\PYG{o}{/}\PYG{l+m+mf}{21.}\PYG{n}{T11998}\PYG{o}{/}\PYG{n}{BODC}\PYG{o}{\PYGZhy{}}\PYG{l+m+mi}{0000}\PYG{o}{\PYGZhy{}}\PYG{l+m+mi}{001}\PYG{n}{A}\PYG{o}{\PYGZhy{}}\PYG{l+m+mi}{64}\PYG{n}{A3}\PYG{o}{\PYGZhy{}}\PYG{n}{B}\PYG{o}{\PYGZhy{}}\PYG{n}{TEST}
\end{sphinxVerbatim}

\sphinxAtStartPar
\sphinxstylestrong{Delete a PID (not allowed for production Handles):}

\begin{sphinxVerbatim}[commandchars=\\\{\}]
\PYG{n}{curl} \PYG{o}{\PYGZhy{}}\PYG{n}{v} \PYG{o}{\PYGZhy{}}\PYG{n}{u} \PYG{l+s+s2}{\PYGZdq{}}\PYG{l+s+s2}{username:password}\PYG{l+s+s2}{\PYGZdq{}} \PYG{o}{\PYGZhy{}}\PYG{n}{H} \PYG{l+s+s2}{\PYGZdq{}}\PYG{l+s+s2}{Accept:application/json}\PYG{l+s+s2}{\PYGZdq{}} \PYG{o}{\PYGZhy{}}\PYG{n}{H} \PYG{l+s+s2}{\PYGZdq{}}\PYG{l+s+s2}{Content\PYGZhy{}Type:application/json}\PYG{l+s+s2}{\PYGZdq{}} \PYG{o}{\PYGZhy{}}\PYG{n}{X} \PYG{n}{DELETE} \PYG{n}{http}\PYG{p}{:}\PYG{o}{/}\PYG{o}{/}\PYG{n}{vm04}\PYG{o}{.}\PYG{n}{pid}\PYG{o}{.}\PYG{n}{gwdg}\PYG{o}{.}\PYG{n}{de}\PYG{p}{:}\PYG{l+m+mi}{8081}\PYG{o}{/}\PYG{n}{handles}\PYG{o}{/}\PYG{l+m+mf}{21.}\PYG{n}{T11998}\PYG{o}{/}\PYG{n}{BODC}\PYG{o}{\PYGZhy{}}\PYG{l+m+mi}{0000}\PYG{o}{\PYGZhy{}}\PYG{l+m+mi}{001}\PYG{n}{A}\PYG{o}{\PYGZhy{}}\PYG{l+m+mi}{64}\PYG{n}{A3}\PYG{o}{\PYGZhy{}}\PYG{n}{B}\PYG{o}{\PYGZhy{}}\PYG{n}{TEST}
\end{sphinxVerbatim}

\sphinxAtStartPar
\sphinxstylestrong{Update a PID:}

\begin{sphinxVerbatim}[commandchars=\\\{\}]
\PYG{n}{curl} \PYG{o}{\PYGZhy{}}\PYG{n}{v} \PYG{o}{\PYGZhy{}}\PYG{n}{u} \PYG{l+s+s2}{\PYGZdq{}}\PYG{l+s+s2}{username:password}\PYG{l+s+s2}{\PYGZdq{}} \PYG{o}{\PYGZhy{}}\PYG{n}{H} \PYG{l+s+s2}{\PYGZdq{}}\PYG{l+s+s2}{Accept:application/json}\PYG{l+s+s2}{\PYGZdq{}} \PYG{o}{\PYGZhy{}}\PYG{n}{H} \PYG{l+s+s2}{\PYGZdq{}}\PYG{l+s+s2}{Content\PYGZhy{}Type:application/json}\PYG{l+s+s2}{\PYGZdq{}} \PYG{o}{\PYGZhy{}}\PYG{n}{X} \PYG{n}{PUT} \PYG{o}{\PYGZhy{}}\PYG{o}{\PYGZhy{}}\PYG{n}{data} \PYG{l+s+s1}{\PYGZsq{}}\PYG{l+s+s1}{[}\PYG{l+s+s1}{\PYGZob{}}\PYG{l+s+s1}{\PYGZdq{}}\PYG{l+s+s1}{type}\PYG{l+s+s1}{\PYGZdq{}}\PYG{l+s+s1}{:}\PYG{l+s+s1}{\PYGZdq{}}\PYG{l+s+s1}{21.T11148/8eb858ee0b12e8e463a5}\PYG{l+s+s1}{\PYGZdq{}}\PYG{l+s+s1}{,}\PYG{l+s+s1}{\PYGZdq{}}\PYG{l+s+s1}{parsed\PYGZus{}data}\PYG{l+s+s1}{\PYGZdq{}}\PYG{l+s+s1}{:}\PYG{l+s+s1}{\PYGZdq{}}\PYG{l+s+s1}{\PYGZob{}}\PYG{l+s+se}{\PYGZbs{}\PYGZdq{}}\PYG{l+s+s1}{identifierValue}\PYG{l+s+se}{\PYGZbs{}\PYGZdq{}}\PYG{l+s+s1}{:}\PYG{l+s+se}{\PYGZbs{}\PYGZdq{}}\PYG{l+s+s1}{http://hdl.handle.net/21.T11998/BODC\PYGZhy{}0000\PYGZhy{}001A\PYGZhy{}64A3\PYGZhy{}B\PYGZhy{}TEST}\PYG{l+s+se}{\PYGZbs{}\PYGZdq{}}\PYG{l+s+s1}{,}\PYG{l+s+se}{\PYGZbs{}\PYGZdq{}}\PYG{l+s+s1}{identiferType}\PYG{l+s+se}{\PYGZbs{}\PYGZdq{}}\PYG{l+s+s1}{:}\PYG{l+s+se}{\PYGZbs{}\PYGZdq{}}\PYG{l+s+s1}{MeasuringInstrument}\PYG{l+s+se}{\PYGZbs{}\PYGZdq{}}\PYG{l+s+s1}{\PYGZcb{}}\PYG{l+s+s1}{\PYGZdq{}}\PYG{l+s+s1}{\PYGZcb{}]}\PYG{l+s+s1}{\PYGZsq{}} \PYG{n}{http}\PYG{p}{:}\PYG{o}{/}\PYG{o}{/}\PYG{n}{vm04}\PYG{o}{.}\PYG{n}{pid}\PYG{o}{.}\PYG{n}{gwdg}\PYG{o}{.}\PYG{n}{de}\PYG{p}{:}\PYG{l+m+mi}{8081}\PYG{o}{/}\PYG{n}{handles}\PYG{o}{/}\PYG{l+m+mf}{21.}\PYG{n}{T11998}\PYG{o}{/}\PYG{n}{BODC}\PYG{o}{\PYGZhy{}}\PYG{l+m+mi}{0000}\PYG{o}{\PYGZhy{}}\PYG{l+m+mi}{001}\PYG{n}{A}\PYG{o}{\PYGZhy{}}\PYG{l+m+mi}{64}\PYG{n}{A3}\PYG{o}{\PYGZhy{}}\PYG{n}{B}\PYG{o}{\PYGZhy{}}\PYG{n}{TEST}
\end{sphinxVerbatim}


\subsubsection{Using the Handle API}
\label{\detokenize{epic-cookbook/handles:using-the-handle-api}}
\sphinxAtStartPar
The following HTTP protocol methods enable users to manage their PID
handle records using the generic Handle API based on Certificates.
Server: \sphinxcode{\sphinxupquote{vm04.pid.gwdg.de}}, Port: \sphinxcode{\sphinxupquote{8081}}, Resources: \sphinxcode{\sphinxupquote{handles/}}

\sphinxAtStartPar
The process to derive the \sphinxcode{\sphinxupquote{privkey.pem}} and \sphinxcode{\sphinxupquote{certificate\_only.pem}}
from a is described for instance at:
\sphinxurl{http://eudat-b2safe.github.io/B2HANDLE/creatingclientcertificates.html}

\sphinxAtStartPar
The Handle API does not have an internal suffix generator.  The suffix
needs to be provided by the user.

\sphinxAtStartPar
The Handle API only knows POST, GET and DELETE methods, which means
that, if the Credentials are sufficient, an existing PID could be
accidentally overwritten by a request intended for creation.  This has
to be detected by the user in advance.

\sphinxAtStartPar
\sphinxstylestrong{Access parameters:}

\sphinxAtStartPar
For given username, index, where the public key HS\_PUBKEY is stored,
and prefix the certificate files are stored here with the naming
convention \$\{INDEX\}\_\$\{PREFIX\}\_\$\{USER\}\_???.pem.

\begin{sphinxVerbatim}[commandchars=\\\{\}]
\PYG{n}{PATH}\PYG{o}{=}\PYG{l+s+s2}{\PYGZdq{}}\PYG{l+s+s2}{/SomePath2Certs}\PYG{l+s+s2}{\PYGZdq{}}
\PYG{n}{PREFIX}\PYG{o}{=}\PYG{l+s+s2}{\PYGZdq{}}\PYG{l+s+s2}{21.T11998}\PYG{l+s+s2}{\PYGZdq{}} \PYG{c+c1}{\PYGZsh{} prefix of the PID service}
\PYG{n}{USER}\PYG{o}{=}\PYG{l+s+s2}{\PYGZdq{}}\PYG{l+s+s2}{USER21}\PYG{l+s+s2}{\PYGZdq{}} \PYG{c+c1}{\PYGZsh{} USER that has access to PIDs under \PYGZdl{}PREFIX}
\PYG{n}{INDEX}\PYG{o}{=}\PYG{l+s+s2}{\PYGZdq{}}\PYG{l+s+s2}{300}\PYG{l+s+s2}{\PYGZdq{}}  \PYG{c+c1}{\PYGZsh{} index where HS\PYGZus{}PUBKEY is stored for \PYGZdl{}USER}
\PYG{n}{SERVPORT}\PYG{o}{=}\PYG{l+s+s2}{\PYGZdq{}}\PYG{l+s+s2}{vm04.pid.gwdg.de:8001}\PYG{l+s+s2}{\PYGZdq{}} \PYG{c+c1}{\PYGZsh{} PID service and port}
\PYG{n}{VERBOSE}\PYG{o}{=}\PYG{l+s+s2}{\PYGZdq{}}\PYG{l+s+s2}{\PYGZdq{}} \PYG{c+c1}{\PYGZsh{} optional “ \PYGZhy{}v \PYGZdq{}}
\PYG{c+c1}{\PYGZsh{} Certificates}
\PYG{n}{USERKEY}\PYG{o}{=}\PYG{l+s+s2}{\PYGZdq{}}\PYG{l+s+s2}{\PYGZdl{}}\PYG{l+s+si}{\PYGZob{}PATH\PYGZcb{}}\PYG{l+s+s2}{/Certificates/\PYGZdl{}}\PYG{l+s+si}{\PYGZob{}INDEX\PYGZcb{}}\PYG{l+s+s2}{\PYGZus{}\PYGZdl{}}\PYG{l+s+si}{\PYGZob{}PREFIX\PYGZcb{}}\PYG{l+s+s2}{\PYGZus{}\PYGZdl{}}\PYG{l+s+si}{\PYGZob{}USER\PYGZcb{}}\PYG{l+s+s2}{\PYGZus{}privkey.pem}\PYG{l+s+s2}{\PYGZdq{}}
\PYG{n}{USERCERT}\PYG{o}{=}\PYG{l+s+s2}{\PYGZdq{}}\PYG{l+s+s2}{\PYGZdl{}}\PYG{l+s+si}{\PYGZob{}PATH\PYGZcb{}}\PYG{l+s+s2}{/Certificates/\PYGZdl{}}\PYG{l+s+si}{\PYGZob{}INDEX\PYGZcb{}}\PYG{l+s+s2}{\PYGZus{}\PYGZdl{}}\PYG{l+s+si}{\PYGZob{}PREFIX\PYGZcb{}}\PYG{l+s+s2}{\PYGZus{}\PYGZdl{}}\PYG{l+s+si}{\PYGZob{}USER\PYGZcb{}}\PYG{l+s+s2}{\PYGZus{}certificate\PYGZus{}only.pem}\PYG{l+s+s2}{\PYGZdq{}}
\end{sphinxVerbatim}

\sphinxAtStartPar
\sphinxstylestrong{Create Handle:}

\begin{sphinxVerbatim}[commandchars=\\\{\}]
curl \PYGZhy{}s \PYGZhy{}\PYGZhy{}insecure \PYGZdl{}\PYGZob{}VERBOSE\PYGZcb{} \PYGZhy{}\PYGZhy{}key \PYGZdl{}\PYGZob{}USERKEY\PYGZcb{} \PYGZhy{}\PYGZhy{}cert \PYGZdl{}\PYGZob{}USERCERT\PYGZcb{} \PYGZhy{}H \PYGZdq{}Content\PYGZhy{}Type:application/json\PYGZdq{} \PYGZhy{}H \PYGZsq{}Authorization: Handle clientCert=\PYGZdq{}true\PYGZdq{}\PYGZsq{} \PYGZhy{}X PUT \PYGZhy{}\PYGZhy{}data  \PYGZsq{}\PYGZob{}\PYGZdq{}values\PYGZdq{}:[\PYGZob{}\PYGZdq{}index\PYGZdq{}:100,\PYGZdq{}type\PYGZdq{}:\PYGZdq{}HS\PYGZus{}ADMIN\PYGZdq{},\PYGZdq{}data\PYGZdq{}:\PYGZob{}\PYGZdq{}value\PYGZdq{}:\PYGZob{}\PYGZdq{}index\PYGZdq{}:\PYGZsq{}\PYGZdl{}\PYGZob{}INDEX\PYGZcb{}\PYGZsq{},\PYGZdq{}handle\PYGZdq{}:\PYGZdq{}\PYGZsq{}\PYGZdl{}\PYGZob{}PREFIX\PYGZcb{}\PYGZsq{}\PYGZbs{}/\PYGZsq{}\PYGZdl{}\PYGZob{}USER\PYGZcb{}\PYGZsq{}\PYGZdq{},\PYGZdq{}permissions\PYGZdq{}:\PYGZdq{}011111110011\PYGZdq{},\PYGZdq{}format\PYGZdq{}:\PYGZdq{}admin\PYGZdq{}\PYGZcb{},\PYGZdq{}format\PYGZdq{}:\PYGZdq{}admin\PYGZdq{}\PYGZcb{}\PYGZcb{},\PYGZob{}\PYGZdq{}index\PYGZdq{}:1,\PYGZdq{}type\PYGZdq{}:\PYGZdq{}URL\PYGZdq{},\PYGZdq{}data\PYGZdq{}:\PYGZdq{}www.gwdg.de\PYGZdq{}\PYGZcb{}]\PYGZcb{}\PYGZsq{} https://\PYGZdl{}\PYGZob{}SERVPORT\PYGZcb{}/api/handles/\PYGZdl{}\PYGZob{}PREFIX\PYGZcb{}/test\PYGZus{}epic3\PYGZus{}1234
\end{sphinxVerbatim}

\sphinxAtStartPar
\sphinxstylestrong{Get Handle created:}

\begin{sphinxVerbatim}[commandchars=\\\{\}]
curl \PYGZhy{}s \PYGZhy{}\PYGZhy{}insecure \PYGZdl{}\PYGZob{}VERBOSE\PYGZcb{} \PYGZhy{}\PYGZhy{}key \PYGZdl{}\PYGZob{}USERKEY\PYGZcb{} \PYGZhy{}\PYGZhy{}cert \PYGZdl{}\PYGZob{}USERCERT\PYGZcb{} \PYGZhy{}H \PYGZdq{}Content\PYGZhy{}Type:application/json\PYGZdq{} \PYGZhy{}H \PYGZsq{}Authorization: Handle clientCert=\PYGZdq{}true\PYGZdq{}\PYGZsq{} \PYGZhy{}q https://\PYGZdl{}\PYGZob{}SERVPORT\PYGZcb{}/api/handles/test\PYGZus{}epic3\PYGZus{}1234
\end{sphinxVerbatim}

\sphinxAtStartPar
\sphinxstylestrong{Modify Handle created:}

\begin{sphinxVerbatim}[commandchars=\\\{\}]
curl \PYGZhy{}s \PYGZhy{}\PYGZhy{}insecure \PYGZdl{}\PYGZob{}VERBOSE\PYGZcb{} \PYGZhy{}\PYGZhy{}key \PYGZdl{}\PYGZob{}USERKEY\PYGZcb{} \PYGZhy{}\PYGZhy{}cert \PYGZdl{}\PYGZob{}USERCERT\PYGZcb{} \PYGZhy{}H \PYGZdq{}Content\PYGZhy{}Type:application/json\PYGZdq{} \PYGZhy{}H \PYGZsq{}Authorization: Handle clientCert=\PYGZdq{}true\PYGZdq{}\PYGZsq{} \PYGZhy{}X PUT \PYGZhy{}\PYGZhy{}data  \PYGZsq{}\PYGZob{}\PYGZdq{}values\PYGZdq{}:[\PYGZob{}\PYGZdq{}index\PYGZdq{}:100,\PYGZdq{}type\PYGZdq{}:\PYGZdq{}HS\PYGZus{}ADMIN\PYGZdq{},\PYGZdq{}data\PYGZdq{}:\PYGZob{}\PYGZdq{}value\PYGZdq{}:\PYGZob{}\PYGZdq{}index\PYGZdq{}:\PYGZsq{}\PYGZdl{}\PYGZob{}INDEX\PYGZcb{}\PYGZsq{},\PYGZdq{}handle\PYGZdq{}:\PYGZdq{}\PYGZsq{}\PYGZdl{}\PYGZob{}PREFIX\PYGZcb{}\PYGZsq{}\PYGZbs{}/\PYGZsq{}\PYGZdl{}\PYGZob{}USER\PYGZcb{}\PYGZsq{}\PYGZdq{},\PYGZdq{}permissions\PYGZdq{}:\PYGZdq{}011111110011\PYGZdq{},\PYGZdq{}format\PYGZdq{}:\PYGZdq{}admin\PYGZdq{}\PYGZcb{},\PYGZdq{}format\PYGZdq{}:\PYGZdq{}admin\PYGZdq{}\PYGZcb{}\PYGZcb{},\PYGZob{}\PYGZdq{}index\PYGZdq{}:1,\PYGZdq{}type\PYGZdq{}:\PYGZdq{}URL\PYGZdq{},\PYGZdq{}data\PYGZdq{}:\PYGZdq{}pid.gwdg.de\PYGZdq{}\PYGZcb{}]\PYGZcb{}\PYGZsq{} https://\PYGZdl{}\PYGZob{}SERVPORT\PYGZcb{}/api/handles/\PYGZdl{}\PYGZob{}PREFIX\PYGZcb{}/test\PYGZus{}epic3\PYGZus{}1234
\end{sphinxVerbatim}

\sphinxAtStartPar
\sphinxstylestrong{Delete Handle created:}

\begin{sphinxVerbatim}[commandchars=\\\{\}]
curl \PYGZhy{}s \PYGZhy{}\PYGZhy{}insecure \PYGZdl{}\PYGZob{}VERBOSE\PYGZcb{} \PYGZhy{}\PYGZhy{}key \PYGZdl{}\PYGZob{}USERKEY\PYGZcb{} \PYGZhy{}\PYGZhy{}cert \PYGZdl{}\PYGZob{}USERCERT\PYGZcb{} \PYGZhy{}H \PYGZdq{}Content\PYGZhy{}Type:application/json\PYGZdq{} \PYGZhy{}H \PYGZsq{}Authorization: Handle clientCert=\PYGZdq{}true\PYGZdq{}\PYGZsq{} \PYGZhy{}X DELETE  https://\PYGZdl{}\PYGZob{}SERVPORT\PYGZcb{}/api/handles/test\PYGZus{}epic3\PYGZus{}1234
\end{sphinxVerbatim}


\chapter{DataCite Cookbook}
\label{\detokenize{datacite-cookbook/index:datacite-cookbook}}\label{\detokenize{datacite-cookbook/index:id1}}\label{\detokenize{datacite-cookbook/index::doc}}

\begin{savenotes}\sphinxattablestart
\centering
\begin{tabular}[t]{|*{2}{\X{1}{2}|}}
\hline

\sphinxAtStartPar
Document type
&
\begin{DUlineblock}{0em}
\item[] Research Data Alliance (RDA)
\item[] Persistent Identification of Instruments (PIDINST)
\item[] working group output report
\end{DUlineblock}
\\
\hline
\end{tabular}
\par
\sphinxattableend\end{savenotes}


\section{Introduction}
\label{\detokenize{datacite-cookbook/intro:introduction}}\label{\detokenize{datacite-cookbook/intro::doc}}
\sphinxAtStartPar
This cookbook enables instrument providers to create \sphinxhref{https://datacite.org/}{DataCite} DOIs
as persistent identifiers (PID) for instruments.  DataCite is a global
not\sphinxhyphen{}for\sphinxhyphen{}profit organization that provides DOIs for research data and
other research outputs.  The \sphinxhref{https://www.rd-alliance.org/groups/persistent-identification-instruments-wg}{Persistent Identification of Instruments
WG (PIDINST)} has worked with DataCite on the mapping of
the PIDINST metadata schema onto the DataCite schema.

\sphinxAtStartPar
This cookbook is not intended to replace the \sphinxhref{https://support.datacite.org/}{DataCite
documentation}.  It will rather point to the relevant steps for
minting an instrument DOI and mostly concentrate on the instrument PID
metadata.  The cookbook assumes some basic knowledge about DOIs.  The
\sphinxhref{https://en.wikipedia.org/wiki/Digital\_object\_identifier}{Wikipedia article on DOI} is a useful primer containing
the relevant information.


\section{Prerequisites for minting DOIs}
\label{\detokenize{datacite-cookbook/prereq:prerequisites-for-minting-dois}}\label{\detokenize{datacite-cookbook/prereq::doc}}
\sphinxAtStartPar
As a prerequisite to issue DataCite DOIs, your organization needs to
be a DataCite member or work with a DataCite member.  This will
provide you with your own DOI prefix and the access to
\sphinxhref{https://doi.datacite.org/}{DataCite Fabrica} that you will need for
{\hyperref[\detokenize{datacite-cookbook/minting:datacite-cookbook-minting}]{\sphinxcrossref{\DUrole{std,std-ref}{Minting the instrument DOI}}}}.


\section{Collecting the metadata}
\label{\detokenize{datacite-cookbook/metadata:collecting-the-metadata}}\label{\detokenize{datacite-cookbook/metadata::doc}}
\sphinxAtStartPar
To create a DOI for an instrument, you need to collect all the
metadata that describe the instrument and that you want to include in
the DOI record.  Section {\hyperref[\detokenize{white-paper/metadata-schema:pidinst-metadata-schema}]{\sphinxcrossref{\DUrole{std,std-ref}{PIDINST metadata schema}}}} in the PIDINST
White Paper describes the metadata that you should consider.

\sphinxAtStartPar
The Persistent Identification of Instruments WG has developed a
PIDINST Metadata Schema.  But since you are going to create a DataCite
DOI, you will be constrained to use the \sphinxhref{https://schema.datacite.org/}{DataCite Metadata Schema}.
With version 4.5 of that schema, DataCite has significantly improved
support for instruments and also provides a \sphinxhref{https://datacite-metadata-schema.readthedocs.io/en/latest/mappings/pidinst/}{Mapping of the
PIDINST Schema onto the DataCite Schema}


\subsection{Mapping of PIDINST metadata onto DataCite}
\label{\detokenize{datacite-cookbook/metadata:mapping-of-pidinst-metadata-onto-datacite}}
\sphinxAtStartPar
Based on the mapping provided by DataCite, we want to give in the
following additional hints and discuss how the metadata describing the
instrument can be best represented in the DataCite Schema:
\begin{description}
\item[{\sphinxtitleref{Identifier}}] \leavevmode
\sphinxAtStartPar
The DOI that you are going to create.  Add as DataCite property
\sphinxtitleref{Identifier} with \sphinxtitleref{identifierType=DOI}.

\item[{\sphinxtitleref{LandingPage}}] \leavevmode
\sphinxAtStartPar
The URL of the landing page that the PID resolves to.  The DataCite
Schema does not have a property for this, but you’ll register the
URL along with the metadata when creating the DOI.

\item[{\sphinxtitleref{Name}}] \leavevmode
\sphinxAtStartPar
The name by which this instrument is known.  Add as DataCite property
\sphinxtitleref{Title}.

\item[{\sphinxtitleref{Owner}}] \leavevmode
\sphinxAtStartPar
The organization or individual that manages the instrument.  Add as
DataCite property \sphinxtitleref{Contributor} with
\sphinxtitleref{contributorType=HostingInstitution}.  Consider also to add an
identifier for the owner in the \sphinxtitleref{nameIdentifier} subproperty of
\sphinxtitleref{Contributor}.

\item[{\sphinxtitleref{Manufacturer}}] \leavevmode
\sphinxAtStartPar
The organization or individual that built the instrument.  Add as
DataCite property \sphinxtitleref{Creator}.  Consider also to add an identifier for
the manufacturer in the \sphinxtitleref{nameIdentifier} subproperty of \sphinxtitleref{Creator}.

\item[{\sphinxtitleref{Model}}] \leavevmode
\sphinxAtStartPar
The name of the model or type of the instrument.  As of this
writing, the DataCite Schema has no specific property for that.  The
mapping provided by DataCite suggest to add it as a \sphinxtitleref{Description}
with \sphinxtitleref{descriptionType=TechnicalInfo}, see note below.

\sphinxAtStartPar
The DataCite property \sphinxtitleref{Description} does not provide a way to
include the \sphinxtitleref{modelIdentifier}.  If the model has a PID and you want
to include that, one option would be to additionally add a
\sphinxtitleref{RelatedIdentifier} with \sphinxtitleref{relationType=References}.

\item[{\sphinxtitleref{Description}}] \leavevmode
\sphinxAtStartPar
A textual description of the device and its capabilities.  Add as
DataCite property \sphinxtitleref{Description} with \sphinxtitleref{descriptionType=Abstract}.

\item[{\sphinxtitleref{InstrumentType}}] \leavevmode
\sphinxAtStartPar
A classification of the type of the instrument.  As of this writing,
the DataCite Schema has no specific property for that.  The mapping
provided by DataCite suggest to add it as a \sphinxtitleref{Description} with
\sphinxtitleref{descriptionType=TechnicalInfo}, see note below.  An alternative
might be to add it as keywords providing such a classification in
the \sphinxtitleref{Subject} property.  The latter might be particularly useful if
the instrument type is using terms from a controlled vocabulary, as
\sphinxtitleref{Subject} allows to link those terms using the \sphinxtitleref{subjectScheme},
\sphinxtitleref{schemeURI}, and \sphinxtitleref{valueURI} subproperties.

\item[{\sphinxtitleref{MeasuredVariable}}] \leavevmode
\sphinxAtStartPar
The variables or physical properties that the instrument measures or
observes.  As of this writing, the DataCite Schema has no specific
property for that.  The mapping provided by DataCite suggest to add
it as a \sphinxtitleref{Description} with \sphinxtitleref{descriptionType=TechnicalInfo}, see note
below.

\item[{\sphinxtitleref{Date}}] \leavevmode
\sphinxAtStartPar
Relevant events pertaining to this instrument instance.  Add as
DataCite property \sphinxtitleref{Date}.  Use \sphinxtitleref{dateType=Available} to indicate the
date that the instrument is or was in operation.  Use a single date
if the instrument is still in operation, to indicate a start date.
Use a date interval to indicate a start and an end date, if the
instrument has already been decommissioned.

\item[{\sphinxtitleref{RelatedIdentifier}}] \leavevmode
\sphinxAtStartPar
This can be used to establish links to related resources.  The
DataCite Schema has a property with the same name, having very
similar subproperties and semantics as the PIDINST Schema.

\item[{\sphinxtitleref{AlternateIdentifier}}] \leavevmode
\sphinxAtStartPar
To be used if this instrument is also registered elsewhere.  Add as
DataCite property \sphinxtitleref{AlternateIdentifier}.  Use
\sphinxtitleref{alternateIdentifierType=SerialNumber} for a serial number
attributed by the manufacturer.  Use
\sphinxtitleref{alternateIdentifierType=InventoryNumber} for an inventory number
used by the owner.

\sphinxAtStartPar
Note that as opposed to the PIDINST schema,
\sphinxtitleref{alternateIdentifierType} is free text in the DataCite schema.
Thus, when adding an alternate identifier that is not a serial
number or an inventory number, you are not forced to use
\sphinxtitleref{alternateIdentifierType=Other}, but may set the appropriate type in
\sphinxtitleref{alternateIdentifierType} right away.

\end{description}


\subsection{Note on Description in the DataCite Schema}
\label{\detokenize{datacite-cookbook/metadata:note-on-description-in-the-datacite-schema}}
\sphinxAtStartPar
The mapping of PIDINST metadata onto DataCite suggest that \sphinxtitleref{Model},
\sphinxtitleref{InstrumentType}, and \sphinxtitleref{MeasuredVariable} should be added as a
\sphinxtitleref{Description} with \sphinxtitleref{descriptionType=TechnicalInfo}.  The value of
\sphinxtitleref{Description} is free text.  There is no structured way to include
subproperties such as \sphinxtitleref{modelIdentifier} here.

\sphinxAtStartPar
Note that \sphinxtitleref{Description} is multivalued, so you may add as many
instances as needed, even using the same \sphinxtitleref{descriptionType}.  We
suggest to use separate \sphinxtitleref{Description} instances for \sphinxtitleref{Model},
\sphinxtitleref{InstrumentType} and \sphinxtitleref{MeasuredVariable} respectively.


\subsection{Additional properties in the DataCite Schema}
\label{\detokenize{datacite-cookbook/metadata:additional-properties-in-the-datacite-schema}}
\sphinxAtStartPar
There are a few more properties in the DataCite Schema that have no
counterpart in the PIDINST Schema and that either need to be set
because they are mandatory in DataCite or that are worth considering.
Of course, any other DataCite property not mentioned here may be
considered as well, if it makes sense for a particular use case.
\begin{description}
\item[{\sphinxtitleref{Publisher}}] \leavevmode
\sphinxAtStartPar
“The name of the entity that holds, archives, publishes, prints,
distributes, releases, issues, or produces the resource” (quote from
the definition in the DataCite Schema).  It’s not quite clear what
that would mean in the case of an instrument and it seem to be a
little redundant with what would be the \sphinxtitleref{Owner} in the PIDINST
Schema.  But it is mandatory in the DataCite Schema, so it needs to
be set.  We recommend to set it to the entity that created the DOI
and is responsible for maintaining the DOI metadata.

\item[{\sphinxtitleref{PublicationYear}}] \leavevmode
\sphinxAtStartPar
Mandatory in the DataCite Schema.  We suggest to set it to the year
of issuing the DOI.

\item[{\sphinxtitleref{ResourceType}}] \leavevmode
\sphinxAtStartPar
DataCite DOIs are for many different types of objects, so there is a
need to indicate the type.  This is mandatory in the DataCite
Schema.  \sphinxtitleref{ResourceType} is free text, but even more relevant is the
subproperty \sphinxtitleref{resourceTypeGeneral} having a controlled list of values
with defined types of resources.  Set
\sphinxtitleref{resourceTypeGeneral=Instrument} here.

\item[{\sphinxtitleref{FundingReference}}] \leavevmode
\sphinxAtStartPar
This is optional in the DataCite Schema, but it may be useful to
acknowledge external funding that supported the purchase or the
creation of the instrument.

\end{description}


\section{Minting the instrument DOI}
\label{\detokenize{datacite-cookbook/minting:minting-the-instrument-doi}}\label{\detokenize{datacite-cookbook/minting:datacite-cookbook-minting}}\label{\detokenize{datacite-cookbook/minting::doc}}
\sphinxAtStartPar
\sphinxhref{https://doi.datacite.org/}{DataCite Fabrica} is the web interface to create and manage your
DOIs and metadata.  Your organization needs to be a DataCite member or
work with a DataCite member.  This will provide you with your own DOI
prefix and the credentials to sign in to Fabrica.

\sphinxAtStartPar
There are basically three different options to create a new or update
an existing DOI that we describe in the following.


\subsection{Create the DOI using the web form}
\label{\detokenize{datacite-cookbook/minting:create-the-doi-using-the-web-form}}
\sphinxAtStartPar
The most interactive way of creating a DOI is to fill a web formular,
indicating the DOI, the URL of the landingpage, and all of the
metadata values.  It provides useful assistence, such as automated
validation or lookup of external identifiers while entering the
values.  See the \sphinxhref{https://support.datacite.org/docs/fabrica-create-doi-form}{DataCite documentation}
for detailed instructions.


\subsection{Create the DOI using file upload}
\label{\detokenize{datacite-cookbook/minting:create-the-doi-using-file-upload}}
\sphinxAtStartPar
If you want to create many instrument DOIs, manually entering all the
metadata values separately may become tedious.  There is another web
formular that allows the metadata to be uploaded as a file.  Different
file formats are supported, but the most fine grained control over the
properties and their values is propably achieved using DataCite XML.
See the \sphinxhref{https://support.datacite.org/docs/fabrica-create-doi-file-upload}{DataCite documentation}
for detailed instructions.


\subsection{Use the DataCite REST API}
\label{\detokenize{datacite-cookbook/minting:use-the-datacite-rest-api}}
\sphinxAtStartPar
The process of creating the DOIs can be fully automated using the
DataCite REST API.  The metadata may be feed into the API as JSON or
DataCite XML.  See the \sphinxhref{https://support.datacite.org/docs/api}{DataCite REST API Guide} for details.



\renewcommand{\indexname}{Index}
\printindex
\end{document}
